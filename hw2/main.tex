\documentclass[letterpaper]{article}

%% Sets page size and margins
\usepackage[letterpaper,top=3cm,bottom=2cm,left=3cm,right=3cm,marginparwidth=1.75cm]{geometry}
 
%% Useful packages
\usepackage{amsmath}
\usepackage{amsfonts}
\usepackage{mathtools}
\usepackage{graphicx}
\usepackage{enumitem}
\usepackage[colorlinks=true, allcolors=blue]{hyperref}
\usepackage{pythontex}
\usepackage{lastpage}
\usepackage{fancyhdr}

\pagestyle{fancy}
\graphicspath{ {images/} }
\newcommand{\true}{$T$}
\newcommand{\false}{$F$}
\newcommand{\ans}{\textit{Answer: }}
\newcommand{\prf}{\textbf{Proof:}}
\newenvironment{question}[2][Question]{\begin{trivlist}
\item[\hskip \labelsep {\bfseries #1}\hskip \labelsep {\bfseries #2.}]}{\end{trivlist}}

% Fancy matrix
\makeatletter
\renewcommand*\env@matrix[1][*\c@MaxMatrixCols c]{%
  \hskip -\arraycolsep
  \let\@ifnextchar\new@ifnextchar
  \array{#1}}
\makeatother

% My scripts setup
\begin{pycode}
import sympy
import numpy as np

import sys
sys.path.append('..')

from utils import Utils
utils = Utils(pytex)
sympify = np.vectorize(sympy.sympify)
\end{pycode}

\newcommand{\printobj}[1]{\py{utils.print.print_object(#1)}}

\title{MA 351, HW 2} 
\lhead{MA 351, HW 2}

\author{Elnard Utiushev}
\rhead{Elnard Utiushev}
\cfoot{\thepage\ of \pageref{LastPage}}

\begin{document}

\maketitle

Section 1.3: True/False: 1.21, 1.23, 1.24; Exercises 1.65, 1.66, 1.69, 1.79, 1.80 \\
Section 1.4: True/False: 1.25, 1.26, 1.31, 1.33, 1.34; Exercises: 1.93, 1.96, 1.102, 1.119

\section{Section 1.3}
\subsection{True/False}

\begin{question}{1.21}
    If the system has three unknowns and R has three nonzero rows, then the
    system can have an infinite number of solutions.

    \ans False, it will have either no solutions, or a single solution.
\end{question}

\begin{question}{1.23}
    The following matrix may be reduced to reduced echelon form using only one
    elementary row operation:

    \begin{pycode}
A = [[1, 0, 0, 0], [0, 1, 0, 3], [0, 0, 1, 1], [0, 0, -1, 1]]
# sol, ops = utils.algorithms.rref(A)
    \end{pycode}

    $$\printobj{A}$$

    \ans False
\end{question}

\begin{question}{1.24}
    The matrix in question 1.23 is the coefficient matrix for a consistent
    system of equations

    \begin{pycode}
A = [[1, 0, 0, 0], [0, 1, 0, 3], [0, 0, 1, 1], [0, 0, -1, 1]]
sol, ops = utils.algorithms.rref(A)
    \end{pycode}

    \ans False, the system is inconsistent
    \begin{align*}
        \py{ops}
    \end{align*}
    
\end{question}

\subsection{Exercises}

\begin{question}{1.65}
    Find the reduced echelon form of each of the following matrices:

    \begin{pycode}
A = [
    [[2, 7, -5, -3, 13], [1, 0, 1, 4, 3], [1, 3, -2, -2, 6]],
    [[1, 1, 1, 1, 1], [2, 2, 1, 1, 1], [1, 0, 2, 3, 2], [4, 3, 2, 1, 0]],
    [[3, 9, 13], [2, 7, 9]],
    [[2, 1, 3, 4, 0, -1], [-2, -1, -3, -4, 5, 6], [4, 2, 7, 6, 1, -1]],
    [[5, 4], [1, 2]],
    [['a', 'b'], ['c', 'd']],
    [[2, 4, 3, 0, 6], [0, 1, 1, 1, 1], [0, 0, 0, 2, 4]],
    [[1, 2, 3, 4], [0, 5, 6, 7], [0, 0, 9, 10], [0, 0, 0, 13]],
    [['a', 'b', 'c', 'd'], [0, 'e', 'f', 'g'], [0, 0, 'h', 'i'], [0, 0, 0, 'j']],
    [[2, 5, 11, 6], [1, 4, 9, 5], [-1, 2, 5, 3], [2, -1, -3, -2]]
]
solutions = map(lambda x: utils.algorithms.rref(x, sep_col=0), A)
output = ""
for sol, ops in solutions:
    output += "\\item \\begin{align*}\n"
    output += ops
    output += "\n\\end{align*} \n"
    \end{pycode}

    \begin{enumerate}[label=\alph*.]
        \py{output}
    \end{enumerate}
    
\end{question}

\begin{question}{1.66}
    Suppose that the matrices in Exercise 1.65 are the augmented matrices for a 
    system of equations. In each case, write the system down and find all 
    solutions (if any) to the system. [(a), (c), (e), (g)]

    \begin{pycode}
A = [
    ('a', [[2, 7, -5, -3, 13], [1, 0, 1, 4, 3], [1, 3, -2, -2, 6]]),
    ('c', [[3, 9, 13], [2, 7, 9]]),
    ('e', [[5, 4], [1, 2]]),
    ('g', [[2, 4, 3, 0, 6], [0, 1, 1, 1, 1], [0, 0, 0, 2, 4]])
]
solutions = map(lambda x: (x[0], utils.algorithms.rref(x[1], sep_col=0)[0]), A)
output = "\\begin{enumerate}[label=\\alph*]"
for letter, sol in solutions:
    output += " \\item[%s.] $$" % letter
    output += utils.print.print_augmented_matrix_num(sol)
    output += "$$ \n"
    output += utils.algorithms.show_solutions(sol) + '\n'
output += "\\end{enumerate}"
    \end{pycode}
   
    \py{output}
\end{question}

\begin{question}{1.69}
    Find conditions on a, b, c, and d for which the following system has solutions:

    \begin{pycode}
A = [
    [2, 4, 1, 3, 'a'],
    [-3, 1, 2, -2, 'b'],
    [13, 5, -4, 12, 'c'],
    [12, 10, -1, 13, 'd']
]
sol, ops = utils.algorithms.rref(A)
    \end{pycode}
    
    \ans 
    $$\printobj{A} \rightarrow \printobj{sol}$$
    \py{utils.algorithms.show_solutions(sol)}
\end{question}

\begin{question}{1.79}
    $$X = \printobj{[1, 2]}, \quad Y = \printobj{[1, -2]}$$
    $$xX + yY = \printobj{['a', 'b']}$$

    \begin{pycode}
A = [
    [1, 1, 'a'],
    [2, -2, 'b']
]
sol, ops = utils.algorithms.rref(A)
    \end{pycode}
    
    \ans 
    \begin{align*}
        \py{ops}
    \end{align*}

    Therefore, it is always possible to solve this equation regardless of 
    $a$ and $b$
    
\end{question}

\begin{question}{1.80}
    Show that the vectors do not span $\mathbb{R}^3$ by finding a vector that cannot be 
    expressed as a linear combination of them

\begin{pycode}
A = {
    'X1': np.array([3, 1, 5]),
    'X2': np.array([2, 1, 4]),
    'X3': np.array([-1, 2, 3])
}
\end{pycode}

    $$\py{utils.print.print_dict_of_arrays(A)}$$

    \ans 
    $$\printobj{A['X1'] + A['X2'] + A['X3'] + np.array([0, 0, 1])}$$
    
\end{question}

\section{Section 1.4}
\subsection{True/False}

\begin{question}{1.25}
    The nullspace of a 3 × 4 matrix cannot consist of only the zero vector.

    \ans True, because there will always be a free variable.
    
\end{question}

\begin{question}{1.26}
    The nullspace of a 4 × 3 matrix cannot consist of only the zero vector.

    \ans False
    $$ \printobj{np.r_[np.eye(3), np.zeros(3).reshape(1, -1)]} $$
    
\end{question}

\begin{question}{1.31}
    The set of 3 × 1 column vectors for which the system below is
    solvable is a plane in $\mathbb{R}^3$.

    \begin{pycode}
A = [
    [1, 2, 3, 'b_1'],
    [2, 4, 6, 'b_2'],
    [3, 6, 9, 'b_3']
]
sol, ops = utils.algorithms.rref(A)
    \end{pycode}

    \ans
    \begin{align*}
        \py{ops}
    \end{align*}
    \py{utils.algorithms.show_solutions(sol)}

\end{question}

\begin{question}{1.33}
    The set of all vectors of the form $\printobj{[1, 'x', 'y']}$, 
    where x and y range over all real numbers, is a subspace of $\mathbb{R}^3$.

    \ans False, for example, vectors $\printobj{[1, 1, 1]}, \printobj{[1, 0, 1]}$
    belong to the subspace, but their difference does not. 
    $$\printobj{[1, 1, 1]} - \printobj{[1, 0, 1]} = \printobj{[0, 1, 0]}$$
    
\end{question}

\begin{question}{1.34}
    The set of all vectors of the form $\printobj{[0, 'x', 'y']}$, 
    where x and y range over all real numbers, is a subspace of $\mathbb{R}^3$.

\begin{pycode}
u = sympify(np.array([0, 'a', 'b'])) 
v = sympify(np.array([0, 'c', 'd']))
\end{pycode}

    \ans True. Let $u = \printobj{u}, v = \printobj{v}$ to be vectors in this
    subspace, then $u + v = \printobj{u + v}$ and 
    $ s * u = \printobj{u * sympy.sympify('s')}$ belong to the subspace.
    
\end{question}

\subsection{Exercises}

\begin{question}{1.93}
    For each matrix A and each vector X, compute AX.

    \begin{pycode}
A = [
    (
        [[1, 0, -3, 2], [2, -2, 1, 1], [3, 2, -2, -3]],
        [[1], [2], [3], [4]]
    ),
    (
        [[-5, 17], [4, 2], [3, 1], [5, -5]],
        [[2], [1]]
    ),
    (
        [[1, 2, 3], [4, 5, 6]],
        [['x'], ['y'], ['z']]
    )
]

output = "\\begin{enumerate}[label=\\alph*.]"
for a, x in A:
    output += "\n \\item $$"
    output += utils.print(a) + " " + utils.print(x) + " = "
    output += utils.print(
        sympy.Matrix(np.array(a)) @ sympy.Matrix(np.array(x))
    )
    output += " $$ \n"

output += "\\end{enumerate}"

    \end{pycode}

    \py{output}

\end{question}

\begin{question}{1.96}
    Find the nullspace for each of the matrices A in Exercise 1.93. 
    Express each answer as a span.


    \begin{pycode}
A = [
    np.array([[1, 0, -3, 2], [2, -2, 1, 1], [3, 2, -2, -3]]),
    np.array([[-5, 17], [4, 2], [3, 1], [5, -5]]),
    np.array([[1, 2, 3], [4, 5, 6]]),
]

output = "\\begin{enumerate}[label=\\alph*.]"
for a in A:
    sol, ops = utils.algorithms.rref(a)
    output += "\n \\item $$"
    output += utils.print(a) + " \\rightarrow " + utils.print(sol) + "$$"
    output += utils.algorithms.show_solutions(
        np.c_[sol, np.zeros((sol.shape[0], 1)).astype(int)]
        ) + '\n'

output += "\\end{enumerate}"
    \end{pycode}

    \py{output}
    
\end{question}

\begin{question}{1.102}
    Create a system of four equations in five unknowns (reader’s choice) 
    such that the solution space is a plane in $\mathbb{R}^5$ (Definition 1.10 on page 38).
    Do not make any coefficients equal 0. Explain why your example works.

    \begin{pycode}
A = np.arange(1, 1 + 6).reshape(1, -1)
B = np.flip(np.arange(7, 7 + 6).reshape(1, -1))
C = np.random.randint(1, 100, (1, 6))

M = np.r_[A, B, C, 2 * B]
sol, ops = utils.algorithms.rref(M)
    \end{pycode}

    \ans 
    $$ \printobj{M} \rightarrow \printobj{sol} $$

    I made one of the rows of the matrix to be a linear combination of another row,
    so the number of free variables at the end will be equal to 2.
    
\end{question}

\begin{question}{1.119}
    A square matrix A is upper triangular if all the entries below the main 
    diagonal are zero. Prove that the set $\mathcal{T}$ of all 3 × 3 upper triangular
    matrices is a subspace of M(3, 3).

    \begin{pycode}
A = sympify(np.array([
    ['a_1', 'a_2', 'a_3'],
    [0, 'a_4', 'a_5'],
    [0, 0, 'a_6']
]))
B = sympify(np.array([
    ['b_1', 'b_2', 'b_3'],
    [0, 'b_4', 'b_5'],
    [0, 0, 'b_6']
]))

s, t = sympy.var('s t')
    \end{pycode}

    \ans Let A and B to be in the set $\mathcal{T}$. Let s and t to be scalars 

    $$A = \printobj{A}, \quad B = \printobj{B}$$

    $$sA + tB = \printobj{s * A} + \printobj{t * B} = \printobj{s * A + t * B}$$

    $sA + tB$ is an upper triangular, therefore $\mathcal{T}$ closed under 
    linear combination and a subspace. 
    
\end{question}

 
\end{document}
