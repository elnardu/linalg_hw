\documentclass[letterpaper]{article}

%% Sets page size and margins
\usepackage[letterpaper,top=3cm,bottom=2cm,left=3cm,right=3cm,marginparwidth=1.75cm]{geometry}
 
%% Useful packages
\usepackage{amsmath}
\usepackage{amsfonts}
\usepackage{mathtools}
\usepackage{graphicx}
\usepackage{enumitem}
\usepackage[colorlinks=true, allcolors=blue]{hyperref}
\usepackage{pythontex}
\usepackage{lastpage}
\usepackage{fancyhdr}

\pagestyle{fancy}
\graphicspath{ {images/} }
\newcommand{\true}{$T$}
\newcommand{\false}{$F$}
\newcommand{\ans}{\textit{Answer: }}
\newcommand{\prf}{\textbf{Proof:}}
\newenvironment{question}[2][Question]{\begin{trivlist}
\item[\hskip \labelsep {\bfseries #1}\hskip \labelsep {\bfseries #2.}]}{\end{trivlist}}

\allowdisplaybreaks

% Fancy matrix
\makeatletter
\renewcommand*\env@matrix[1][*\c@MaxMatrixCols c]{%
  \hskip -\arraycolsep
  \let\@ifnextchar\new@ifnextchar
  \array{#1}}
\makeatother

% My scripts setup
\begin{pycode}
import sympy
import numpy as np

import sys
sys.path.append('..')

from utils import Utils
utils = Utils(pytex)
sympify = np.vectorize(sympy.sympify)

# display det
ddet = utils.print.print_matrix_det


def proj(basis, vector):
  solution = []
  solution_vec = 0
  for b in basis:
    solution.append(f"\\frac{{{utils.print(vector)}\\cdot {utils.print(b)}}}{{{utils.print(b)} \cdot {utils.print(b)}}}{utils.print(b)}")
    solution_vec += (np.dot(vector, b) / np.dot(b, b)) * b

  solution = "+".join(solution)
  solution += f"= {utils.print(solution_vec)}"
  return solution, solution_vec


\end{pycode}

\newcommand{\printobj}[1]{\py{utils.print.print_object(#1)}}

\title{MA 351, HW 10} 
\lhead{MA 351, HW 10}

\author{Elnard Utiushev}
\rhead{Elnard Utiushev}
\cfoot{\thepage\ of \pageref{LastPage}}

\begin{document}

\maketitle

Section 6.2: True/False: 6.11, 6.12, 6.14\\
Exercises: 6.19, 6.20, 6.22, 6.23, 6.24, 6.27, 6.30, 6.32, 6.37

\section{Section 6.2}
\subsection{True/False}

\begin{question}{6.11}
  $[1,1,1]^{t}$ is perpendicular to the span of $[1,-1,0]^{t}$ and 
  $[2,2,-4]^{t}$ in $\mathbb{R}^{3}$

  \ans True, since 
  $[1,1,1]^{t} \cdot [1,-1,0]^{t} = 0$, 
  $[1,1,1]^{t} \cdot [2,2,-4]^{t} = 0$
\end{question}

\begin{question}{6.12}
  Let $\mathcal{W}$ be a two-dimensional subspace of $\mathbb{R}^{5} .$ 
  Suppose that $\left\{Q_{1}, Q_{2}\right\}$ and
  $\left\{P_{1}, P_{2}\right\}$ are two orthogonal bases for $\mathcal{W} .$ 
  Then for all $X \in \mathbb{R}^{5}$ ,

  $$
  \frac{X \cdot Q_{1}}{Q_{1} \cdot Q_{1}} Q_{1}+\frac{X \cdot Q_{2}}{Q_{2} \cdot Q_{2}} Q_{2}=\frac{X \cdot P_{1}}{P_{1} \cdot P_{1}} P_{1}+\frac{X \cdot P_{2}}{P_{2} \cdot P_{2}} P_{2}
  $$

  \ans True, according to Fourier Theorem

  $$
  \frac{X \cdot Q_{1}}{Q_{1} \cdot Q_{1}} Q_{1}+\frac{X \cdot Q_{2}}{Q_{2} \cdot Q_{2}} Q_{2}
  = \text{Proj}_\mathcal{W} (X)
  $$
  $$
  \frac{X \cdot P_{1}}{P_{1} \cdot P_{1}} P_{1}+\frac{X \cdot P_{2}}{P_{2} \cdot P_{2}} P_{2}
  = \text{Proj}_\mathcal{W} (X)
  $$
\end{question}

\begin{question}{6.14}
  Let $\mathcal{W}$ be a subspace of $\mathbb{R}^{n}$ and let 
  $X \in \mathbb{R}^{n} .$ Then $\operatorname{Proj}_{\mathcal{W}}\left(X-\operatorname{Proj}_{\mathcal{W}}(X)\right)=\mathbf{0}$
  
  \ans True, 
  $\operatorname{Proj}_{\mathcal{W}}\left(X-\operatorname{Proj}_{\mathcal{W}}(X)\right)
  =\operatorname{Proj}_{\mathcal{W}}\left(\operatorname{Orth}_{\mathcal{W}}(X)\right)
  =0
  $
\end{question}

\subsection{Exercises}

\begin{question}{6.19}
  In each part let $\mathcal{W}$ be the subspace of $\mathbb{R}^{4}$ spanned by the set $\mathcal{B} .$ Show that $\mathcal{B}$ is
  an orthogonal basis for $\mathcal{W}$ and find $\operatorname{Proj}_{\mathcal{W}}\left([1,2,-1,-3]^{t}\right)$
  
  \begin{enumerate}[label=(\alph*)]
    \item $\mathcal{B}=\left\{[2,-1,-1,-1]^{t},[1,3,3,-4]^{t},[1,1,0,1]^{t}\right\}$
    
    \begin{pycode}
basis = [
  [2,-1,-1,-1],
  [1,3,3,-4],
  [1,1,0,1]
]

basis = list(map(np.array, basis))
vector = np.array([1,2,-1,-3])    
    \end{pycode}

    \ans $\mathcal{B}_1 \cdot \mathcal{B}_2 = \printobj{basis[0].dot(basis[1])}$,
    $\mathcal{B}_2 \cdot \mathcal{B}_3 = \printobj{basis[1].dot(basis[2])}$,
    $\mathcal{B}_1 \cdot \mathcal{B}_3 = \printobj{basis[0].dot(basis[2])}$,
    therefore, the basis is orthogonal.
    $$\operatorname{Proj}_{\mathcal{W}}\left([1,2,-1,-3]^{t}\right)
    = \printobj{proj(basis, vector)[0]}$$

    \item $\mathcal{B}=\left\{[1,1,1,1]^{t},[1,-2,1,0]^{t},[1,1,1,-3]^{t}\right\}$
    
    \begin{pycode}
basis = [
  [1,1,1,1],
  [1,-2,1,0],
  [1,1,1,-3]
]

basis = list(map(np.array, basis))
vector = np.array([1,2,-1,-3])
    \end{pycode}

    \ans $\mathcal{B}_1 \cdot \mathcal{B}_2 = \printobj{basis[0].dot(basis[1])}$,
    $\mathcal{B}_2 \cdot \mathcal{B}_3 = \printobj{basis[1].dot(basis[2])}$,
    $\mathcal{B}_1 \cdot \mathcal{B}_3 = \printobj{basis[0].dot(basis[2])}$,
    therefore, the basis is orthogonal.
    $$\operatorname{Proj}_{\mathcal{W}}\left([1,2,-1,-3]^{t}\right)
    = \printobj{proj(basis, vector)[0]}$$
  \end{enumerate}
\end{question}

\begin{question}{6.20}
  Below, you are given two sets of vectors $\mathcal{B}_{1}$ and $\mathcal{B}_{2}$ in $\mathbb{R}^{2}$ .

  $$
  \begin{array}{l}{\mathcal{B}_{1}=\left\{[-1,1,-1]^{t},[1,3,2]^{t}\right\}} \\ {B_{2}=\left\{[3,1,4]^{t},[-4,16,-1]^{t}\right\}}\end{array}
  $$

  \begin{pycode}
basis1 = [
  [-1,1,-1],
  [1,3,2]
]
basis1 = list(map(np.array, basis1))

basis2 = [
  [3,1,4],
  [-4,16,-1]
]
basis2 = list(map(np.array, basis2))
  \end{pycode}

  \begin{enumerate}[label=(\alph*)]
    \item Show that both $\mathcal{B}_{1}$ and $\mathcal{B}_{2}$ are orthogonal sets.
    
    \begin{enumerate}
      \item $\mathcal{B}_{1_1} \cdot \mathcal{B}_{1_2} = \printobj{basis1[0].dot(basis1[1])}$,
      therefore, basis 1 is orthogonal
      \item $\mathcal{B}_{2_1} \cdot \mathcal{B}_{2_2} = \printobj{basis2[0].dot(basis2[1])}$,
      therefore, basis 2 is orthogonal
    \end{enumerate}

    \item Show that $B_{1}$ and $B_{2}$ both span the same subspace $\mathcal{W}$ of $\mathbb{R}^{3}$ .
    
    \ans 
    $\printobj{utils.algorithms.as_linear_combination(basis1, basis2[0])[1]}$,
    $\printobj{utils.algorithms.as_linear_combination(basis1, basis2[1])[1]}$

    \item Find $\operatorname{Proj}_{w}\left([1,2,2]^{t}\right)$ using formula $(6.20)$ on page 321 and $\mathcal{B}_{1}$
    
    \ans $$\printobj{proj(basis1, np.array([1,2,2]))[0]}$$

    \item Find $\operatorname{Proj}_{w}\left([1,2,2]^{t}\right)$ using formula $(6.20)$ on page 321 and $B_{2} .$ You
    should get the same answer as in part (c). Why?

    \ans $$\printobj{proj(basis2, np.array([1,2,2]))[0]}$$

    The answer is the same since $\mathcal{B}_{1}$ and $\mathcal{B}_{2}$ describe the same subspace according to (b)

    \item Let $X=[x, y, z]^{t} .$ Use formula $(6.20)$ on page 321 to find $\operatorname{Proj}_{w}(X)$ using
    the basis $B_{1} .$ Then find a matrix $R$ such that $\operatorname{Proj}_w(X)=R X .$

    \begin{pycode}
vector = np.array(['x', 'y', 'z'])
vector = sympify(vector)

R = np.array([
  ['17 / 42', '-5 / 42', '10 / 21'],
  ['-5 / 42', '41 / 42', '2 / 21'],
  ['10 / 21', '2 / 21', '13 / 21'],
])
R = sympify(R)
    \end{pycode}

    \ans $$\printobj{proj(basis1, vector)[0]}$$
    $$R = \printobj{R}$$

    \item Repeat part (e) using basis $\mathcal{B}_{2} .$ You should get the same matrix $R$
    
    \ans $$\printobj{proj(basis2, vector)[0]}$$
    $$R = \printobj{R}$$

    \item Show that the matrix $R$ from part (e) satisfies $R^{2}=R .$ Explain the geometric meaning of this equality.
    
    \ans
    $$R^2 = \printobj{R}\printobj{R} = \printobj{sympy.MatMul(sympy.Matrix(R), sympy.Matrix(R)).doit()} = R$$
  \end{enumerate}

\end{question}

\begin{question}{6.22}
  Use the Gram-Schmidt process to find an orthogonal basis for the subspaces
  of $\mathbb{R}^{n}$ spanned by the following ordered sets of vectors for the appropriate $n :$
  
  \begin{pycode}
problems = [
  [[0, 1, 1], [1, 1, 1]],
  [[-2, 1, 3], [2, 1, 2]],
  [[1, 2, 1, 1], [-2, 1, 1, -1], [1, 2, 1, 3]],
  [[1, 1, -1, 1], [1, 1, 0, 0], [1, 1, 1, -1]]
]

def gram_schi(basis):
  solution = "\\begin{itemize}\n"
  p1 = basis[0]
  new_basis = [p1]

  solution += f"\\item $P_1 = X_1 = {utils.print(p1)}$\n"
  for i, b in enumerate(basis[1:]):
    new_p = b - proj(new_basis, b)[1]

    solution += f"\\item $Y_{i+2} = {proj(new_basis, b)[0]}$, "
    solution += f"$P_{i+2} = X_{i+2} - Y_{i+2} = {utils.print(new_p)}$"
    solution += "\n"

    if (new_p == 0).all():
      continue

    new_basis.append(new_p)

  solution += "\\end{itemize}\n\n"
  return solution, new_basis

problem_solution = []

solution = "\\begin{enumerate}[label=(\\alph*)]"
for problem in problems:
  problem = list(map(sympify, problem))
  solution += "\\item "
  solution += f"$${utils.print.print_list_of_arrays(problem)}$$"
  gram_solution, gram_solution_vecs = gram_schi(problem)
  solution += gram_solution
  problem_solution.append(gram_solution_vecs)

solution += "\\end{enumerate}"
  \end{pycode}

  \ans 
  \py{solution}
\end{question}

\begin{question}{6.23}
  Compute Proj $_{\mathcal{W}}\left([1,2,3]^{t}\right),$ where $\mathcal{W}$ is the subspace of $\mathbb{R}^{3}$ spanned by the
  vectors in Exercise $6.22 .$ a. Repeat for the subspace spanned by the vectors in
  Exercise $6.22 . \mathrm{b} .$

  \ans 

  \begin{enumerate}[label=(\alph*)]
    \item 6.22a 
    $$\printobj{proj(problem_solution[0], np.array([1,2,3]))[0]}$$
    \item 6.22b 
    $$\printobj{proj(problem_solution[1], np.array([1,2,3]))[0]}$$
  \end{enumerate}
  
\end{question}

\begin{question}{6.24}
  Compute Proj $w\left([1,2,3,4]^{t}\right),$ where $\mathcal{W}$ is the subspace of $\mathbb{R}^{4}$ spanned by the
  vectors in Exercise $6.22 . \mathrm{c} .$ Repeat for the subspace spanned by the vectors in
  Exercise $6.22 . \mathrm{d} .$
  
  \ans 

  \begin{enumerate}[label=(\alph*)]
    \item 6.22c 
    $$\printobj{proj(problem_solution[2], np.array([1,2,3,4]))[0]}$$
    \item 6.22d 
    $$\printobj{proj(problem_solution[3], np.array([1,2,3,4]))[0]}$$
  \end{enumerate}

\end{question}

\begin{question}{6.27}
  Let $A$ be an $m \times n$ matrix and let $\mathcal{W} \subset \mathbb{R}^{n}$ be the nullspace of $A .$ Prove that for
  all $X \in \mathbb{R}^{n}, A X=A\left(\operatorname{Orth}_{\mathcal{W}} X\right)$
  
  \ans $A\left(\operatorname{Orth}_{\mathcal{W}} X\right) 
  = A(X - \operatorname{Proj}_{\mathcal{W}} X)
  = AX - A(\operatorname{Proj}_{\mathcal{W}} X)$, since $\mathcal{W}$ is the nullspace of A and 
  $\operatorname{Proj}_{\mathcal{W}} X$ is in $\mathcal{W}$, $A(\operatorname{Proj}_{\mathcal{W}} X) = 0$,
  therefore $AX - A(\operatorname{Proj}_{\mathcal{W}} X) = AX$
\end{question}

\begin{question}{6.30}
  Find an orthogonal basis for $S^{\perp}$ for the following sets of vectors.
  

  \begin{enumerate}[label=(\alph*)]
    \begin{pycode}
basis = [
  [1,3,1,-1],
  [2,6,0,1],
  [4,12,2,-1]
]

basis = list(map(np.array, basis))
problem = np.vstack(basis)

rref, _ = utils.algorithms.rref(np.c_[problem, np.zeros((problem.shape[0], 1)).astype(int)])
matrix_sol, sol = utils.algorithms.show_solutions(rref)
    \end{pycode}
    \item $$\printobj{utils.print.print_list_of_arrays(basis)}$$
    $$\printobj{problem}X=0$$
    \py{sol} \\
    Basis: $\printobj{utils.print.print_list_of_arrays([matrix_sol['param_vecs'][0], matrix_sol['param_vecs'][1]])}$ 


    \begin{pycode}
basis = [
  [1,3,1,-1],
  [2,6,0,1]
]

basis = list(map(np.array, basis))
problem = np.vstack(basis)

rref, _ = utils.algorithms.rref(np.c_[problem, np.zeros((problem.shape[0], 1)).astype(int)])
matrix_sol, sol = utils.algorithms.show_solutions(rref)
    \end{pycode}
    \item $$\printobj{utils.print.print_list_of_arrays(basis)}$$
    $$\printobj{problem}X=0$$
    \py{sol} \\
    Basis: $\printobj{utils.print.print_list_of_arrays([matrix_sol['param_vecs'][0], matrix_sol['param_vecs'][1]])}$


    \begin{pycode}
basis = [
  [1,1,1,1,1],
  [1,1,0,1,-1],
  [1,-1,0,0,1]
]

basis = list(map(np.array, basis))
problem = np.vstack(basis)

rref, _ = utils.algorithms.rref(np.c_[problem, np.zeros((problem.shape[0], 1)).astype(int)])
matrix_sol, sol = utils.algorithms.show_solutions(rref)
    \end{pycode}
    \item $$\printobj{utils.print.print_list_of_arrays(basis)}$$
    $$\printobj{problem}X=0$$
    \py{sol} \\
    Basis: $\printobj{utils.print.print_list_of_arrays(matrix_sol['param_vecs'][:-1])}$ 
  \end{enumerate}

\end{question}

\begin{question}{6.32}
  Let $\mathcal{W}$ be a subspace of $\mathbb{R}^{n} .$ Let $\left\{P_{1}, P_{2}, \ldots, P_{k}\right\}$ be an orthogonal basis for $\mathcal{W}$
and let $\left\{X_{1}, X_{2}, \ldots, X_{m}\right\}$ be an orthogonal basis for $\mathcal{W}^{\perp} .$ Show that the set

$$
\mathcal{B}=\left\{P_{1}, P_{2}, \ldots, P_{k}, X_{1}, X_{2}, \ldots, X_{m}\right\}
$$

is an orthogonal set. Prove that this set is a basis for $\mathbb{R}^{n}$ and, hence, $\operatorname{dim}(\mathcal{W})
+\operatorname{dim}\left(\mathcal{W}^{\perp}\right)=n$

  \ans This set is orthogonal since any two vectors $P_i$, $P_k$ and $X_i$, $X_j$
  are orthogonal and any $X_i$ is orthogonal to any $P_j$ since $P_j$ is in $\mathcal{W}$
  and $X_i$ is in $\mathcal{W}^{\perp}$. This set is a basis for $\mathbb{R}^{n}$, since
  any vector $v$ can be written as $v = \operatorname{Proj}_{\mathcal{W}} + 
  \operatorname{Orth}_{\mathcal{W}}$, where $\operatorname{Proj}_{\mathcal{W}} \in \mathcal{W}$
  and $\operatorname{Orth}_{\mathcal{W}} \in \mathcal{W}^{\perp}$. The set is 
  linearly independent, so $\operatorname{dim}(\mathcal{W})
  +\operatorname{dim}\left(\mathcal{W}^{\perp}\right) = k + m =n$
\end{question}

\begin{question}{6.37}
Let $\left\{P_{1}, P_{2}\right\}$ be an ordered orthogonal (but not orthonormal) basis for some
subspace $\mathcal{W}$ of $\mathbb{R}^{n} .$ Let $X$ and $Y$ be elements of $\mathcal{W}$ whose coordinate vectors
with respect to these bases are $X^{\prime}=\left[x_{1}, x_{2}\right]^{t}$ and $Y^{\prime}=\left[y_{1}, y_{2}\right]^{t} .$ Prove that
  
$$
X \cdot Y=x_{1} y_{1}\left|P_{1}\right|^{2}+x_{2} y_{2}\left|P_{2}\right|^{2}
$$

\ans 

$$
X \cdot Y=x_{1}P_{1}y_{1}P_{1}+x_{2}P_{2} y_{2}P_{2}
= x_{1} y_{1}\left|P_{1}\right|^{2}+x_{2} y_{2}\left|P_{2}\right|^{2}
$$

What is the corresponding formula for $|X|^{2} ?$

\ans 
$$|X|^{2} = x_{1}^2\left|P_{1}\right|^{2}+x_{2}^2 \left|P_{2}\right|^{2}$$

\end{question}

\end{document}