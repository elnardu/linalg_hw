\documentclass[letter]{article}

%% Sets page size and margins
\usepackage[a4paper,top=3cm,bottom=2cm,left=3cm,right=3cm,marginparwidth=1.75cm]{geometry}
 
%% Useful packages
\usepackage{amsmath}
\usepackage{mathtools}
\usepackage{graphicx}
\usepackage{enumitem}
\usepackage[colorlinks=true, allcolors=blue]{hyperref}
\usepackage{pythontex}
\usepackage{fancyhdr}

\pagestyle{fancy}
\graphicspath{ {images/} }
\newcommand{\true}{$T$}
\newcommand{\false}{$F$}
\newcommand{\ans}{\textit{Answer: }}
\newcommand{\prf}{\textbf{Proof:}}
\newenvironment{question}[2][Question]{\begin{trivlist}
\item[\hskip \labelsep {\bfseries #1}\hskip \labelsep {\bfseries #2.}]}{\end{trivlist}}

\makeatletter
\renewcommand*\env@matrix[1][*\c@MaxMatrixCols c]{%
  \hskip -\arraycolsep
  \let\@ifnextchar\new@ifnextchar
  \array{#1}}
\makeatother

\begin{pycode}
import sympy
import numpy as np

import sys

sys.path.append('..')
import utils

myprint = utils.printing.MyPrint(pytex) 
myalgo = utils.algorithms.Algorithms(pytex) 
\end{pycode}


\title{MA 351} 
\author{Elnard Utiushev}
\rhead{Elnard Utiushev}
\lhead{MA 351}
% \lfoot{Elnard Utiushev}

\begin{document}

\maketitle

\section{Section 1.1}
\subsection{True/False}

\begin{question}{1.1}
  A subset of a linearly independent set is linearly independent.
  
  \ans True, since none of the elements in the subset are linear combinations. 
\end{question} 

\begin{question}{1.2}
  A subset of a linearly dependent set is linearly dependent. 

  \ans False, for example, lets take this linearly dependent set and remove 
  the last element. The set will no longer be lineary dependant.
  $$S = \Big\{
    \py{myprint([1, 2])},
    \py{myprint([4, 3])},
    \py{myprint([5, 1])},
    \py{myprint([10, 6])}
  \Big\}
  $$
\end{question}

\begin{question}{1.3}
  A set that contains a linearly independent set is linearly independent.

  \ans False
\end{question}

\begin{question}{1.4}
  A set that contains a linearly dependent set is linearly dependent.

  \ans False
\end{question}

\begin{question}{1.5}
  If a set of elements of a vector space is linearly dependent, then each element
  of the set is a linear combination of the other elements of the set.

  \ans True
\end{question}

\begin{question}{1.8}
  If $\{X, A_1, A_2, A_3\}$ is linearly dependent then $X$ is in the span of $A_1$, $A_2$, and $A_3$.

  \ans True
\end{question}

\subsection{Exercises}

\begin{question}{1.2}
  Each of the following sets of matrices is linearly dependent. Demonstrate this 
  by explicitly exhibiting one of the elements of the set as a linear
  combination of the others. You should be able to find the constants by inspection (guessing).

\begin{enumerate}[label=\alph*]
  \item $
    \py{myprint([[1, 1, 2]])} + 2*\py{myprint([[0, 0, 1]])} = \py{myprint([[1, 1, 4]])}
  $

  \item $
    \py{myprint([[1,0,0]])}
    + 2 * \py{myprint([[0,1,0]])}
    + 3 * \py{myprint([[0,0,1]])}
    = \py{myprint([[1,2,3]])}
  $

  \item $
    0 * \py{myprint([[0, 0],
                 [1, 0]])}
    + \py{myprint([[1, 0],
                 [0, 0]])}
    + 2 * \py{myprint([[0, 1],
                       [0, 0]])}
    = \py{myprint([[1, 2],
                   [0, 0]])}
  $

  \item $
  -1 *\py{myprint([[1],
               [2],
               [3]])}
  + \py{myprint([[4],
                   [5],
                   [6]])}
  + 0 * \py{myprint([[9],
                   [12],
                   [15]])}
  = \py{myprint([[3],
               [3],
               [3]])}
  $

  \item $
  2 * \py{myprint([[1, 0], [0, 0]])}
  + \py{myprint([[0, 1], [0, 0]])}
  + 3 * \py{myprint([[0, 0], [1, 0]])}
  -4 * \py{myprint([[0, 0], [0, 1]])}
  = \py{myprint([[2, 1], [3, -4]])}
  $

  \item $
  -3 * \py{myprint([[3, -1, 2], [0, 1, 4]])}
  = \py{myprint([[-9, 3, -6], [0, -3, -12]])}
  $

  \item $
  0 * \py{myprint([[1, 1], [2, 3]])}
  + \py{myprint([[1, 1], [0, 1]])}
  - \py{myprint([[1, 2], [0, 0]])}
  = \py{myprint([[0, -1], [0, 1]])}
  $
\end{enumerate}
\end{question}

\begin{question}{1.8}
  Verify the Remark following Example 1.2 on page 8, that is, show that A1 is not a linear combination of A2, A3, and A4.

  \ans It is not a linear combination since there is no way to get $A_{1_{21}} = 1$ 
  by summing and scaling zeroes.
\end{question}

\begin{question}{1.10}
  Prove that the rows of the following matrix are linearly independent. 
  \begin{pycode}
A = np.array([[1, 2, 3], [0, 5, 6], [0, 0, 8]])
  \end{pycode}

  $$A = \py{myprint(A)}$$

  \ans 
  \begin{align*}
    A_3 &= xA_1 + yA_2 \\
    \py{myprint(A[2, :])} &= x\py{myprint(A[0, :])} + y\py{myprint(A[1, :])} \\
      &= 0 \py{myprint(A[0, :])} + y \py{myprint(A[1, :])} & \text{Since it is the only way to get 0 in } A_{3_1} \\ 
      &= 0 \py{myprint(A[0, :])} + 0 \py{myprint(A[1, :])} & \text{Since it is the only way to get 0 in } A_{3_2}
  \end{align*}

  Same reasoning can be used to prove that $A_2 \neq xA_1 + yA_3$ and $A_1 \neq xA_2 + yA_3$.
\end{question}

\begin{question}{1.19}
  For each of the following sets of functions either find a function $f (x)$ in their
  span such that $f (x) > 0$ for all $x$ or prove that no such function exists.

  \begin{enumerate}[label=\alph*]
    \item $\{\sin{x}, 1\}$ \\
    \ans Since $\forall x, -1 \leq \sin{x} \leq 1$, $f(x) = \sin{x} + 2 * 1 > 0$ 

    \item $\{\cos{x}, 1\}$ \\
    \ans Since $\forall x, -1 \leq \cos{x} \leq 1$, $f(x) = \cos{x} + 2 * 1 > 0$ 

    \item $\{\sin{x}, \cos{x}\}$ \\
    \ans It is not possible. $f(x) = c_1 \sin{x} + c_2 \cos{x}$, since $f(x) > 0$, 
    $f(0) = c_2 > 0, f(\pi) = -c_2 > 0$ which is not possible. 
  \end{enumerate}
  
\end{question}

\begin{question}{1.23}
  Let X, Y, and Z be as shown. Give four matrices (reader’s choice) that belong to their span. Give a matrix that does not belong to their span.

  \begin{pycode}
A = {}
A['X'] = np.array([[1, 2], [0, 3]])
A['Y'] = np.array([[2, -1], [0, 0]])
A['Z'] = np.array([[1, 1], [0, 1]])
  \end{pycode}
  
  $$\py{myprint.print_dict_of_arrays(A)}$$

  \ans \\
  Matrices in their span:
  $$\py{myprint.print_list_of_arrays([
    A['X'] + A['Y'] + A['Z'],
    A['X'] + 3 * A['Y'] + A['Z'],
    A['X'] + A['Y'] + 0 * A['Z'],
    3 * A['X'] + A['Y'] + 0 * A['Z']
  ])}$$
  Matrix not in their span:
  $$\py{myprint([[0, 0], [1, 0]])}$$
 
\end{question}

\begin{question}{1.30}
  Construct an example of your own choice of a 4 × 4 matrix with linearly dependent columns having all of its entries nonzero.

  \ans 
  $$\py{myprint([
    [1, 2, 3, 6],
    [1, 2, 3, 6],
    [1, 2, 3, 6],
    [1, 2, 3, 6]
  ])}$$
  
\end{question}

\section{Section 1.2}
\subsection{True/False}

\begin{question}{1.13}
  The solution set to a system of three equations in three unknowns cannot be a plane.

  \ans False
\end{question}

\begin{question}{1.14}
  A system of linear equations cannot have only two solutions.
  
  \ans True
\end{question}

\begin{question}{1.16}
  A system of four equations in four unknowns always has a solution.
  
  \ans False
\end{question}


\subsection{Exercises}

\begin{question}{1.49}
  One of these vectors is a solution to the system below and one is not. Which is which?
  \begin{pycode}
A = {
  'X': np.array([1, 1, 1, 1]),
  'Y': np.array([1, 2, -1, 1])
}
B = np.array([
  [4, -2, -1, -1],
  [1, 3, -2, -2]
])
  \end{pycode}

  \begin{gather*}
    \py{myprint.print_dict_of_arrays(A)} \\
    4x - 2y - z - w = 0 \\
    x + 3y - 2z - 2w = 0
  \end{gather*}

  \ans Lets rewrite our system as a matrix. 
  $$\py{myprint(B)}$$
  Now we can check if our vectors are solutions to this system:
  \begin{align*}
    \py{myprint(B)}X = \py{myprint(B)}\py{myprint(A['X'])}
     = \py{myprint(np.matmul(B, A['X']))} && X \text{ is a solution} \\
    \py{myprint(B)}Y = \py{myprint(B)}\py{myprint(A['Y'])}
     = \py{myprint(np.matmul(B, A['Y']))} && Y \text{ is not a solution} 
  \end{align*}
\end{question}

\begin{question}{1.55}
For each system: (i) Write the augmented matrix A. (ii) Find all solutions (if any exist). 
Express your answer in parametric form and give the translation vector and the spanning 
vectors. State whether the solution is a line or plane or neither. (iii) If one of the rows 
the solution of the augmented matrix becomes zero during process, explicitly exhibit one row 
of A as a linear combination of the other rows.

\begin{pycode}
s = sympy.var('s')
k = sympy.var('k')
\end{pycode}


\begin{enumerate}[label=\alph*]
  \item
\begin{pycode}
A = np.array([
  [1, -3, 2],
  [-2, 6, -4]
])
sol, ops = myalgo.rref(A)
  \end{pycode}
  \begin{align*}
    \py{ops}
  \end{align*}

  Solution: Line, $
  \py{myprint([['x'], ['y']])} 
  = \py{myprint([['2 + 3*s'], ['s']])} 
  = \py{myprint([[3], [1]])} * s + \py{myprint([[2], [0]])}
  $ \\
  Linear combination: $R_2 = -2R_1$

  \item \begin{pycode}
A = np.array([
  [1, 3, 1, 1],
  [2, 4, 7, 2],
  [3, 10, 5, 7]
])
sol, ops = myalgo.rref(A)
  \end{pycode}
  \begin{align*}
    \py{ops}
  \end{align*}

  Solution: Point, $\py{myprint(sol[:, -1])}$ 

  \item \begin{pycode}
A = np.array([
  [1, 3, 1, 1],
  [2, 4, 7, 2],
  [4, 10, 9, 4]
])
sol, ops = myalgo.rref(A)
  \end{pycode}
  \begin{align*}
    \py{ops}
  \end{align*}

  Solution: Line, $
  \py{myprint([['x'], ['y'], ['z']])} 
  = \py{myprint([['1-(17/2)*s'], ['(5/2)*s'], ['s']])} 
  = \py{myprint([['-(17/2)'], ['(5/2)'], ['1']])} * s + \py{myprint([[1], [0], [0]])}
  $ \\
  Linear combination: $R_3 = 2R_1 + R_2$

  \item \begin{pycode}
A = np.array([
  [1, 3, 1, 1],
  [2, 4, 7, 2],
  [4, 10, 9, 7]
])
sol, ops = myalgo.rref(A)
  \end{pycode}
  \begin{align*}
    \py{ops}
  \end{align*}

  Solution: Inconsistent

  \item \begin{pycode}
A = np.array([
  [1, 2, 1, 0, 1],
  [0, 1, 4, 3, 2],
  [0, 0, 2, 2, 4]
])
sol, ops = myalgo.rref(A)
  \end{pycode}
  \begin{align*}
    \py{ops}
  \end{align*}

  Solution: Line, $
  \py{myprint([['x_1'], ['x_2'], ['x_3'], ['x_4']])} 
  = \py{myprint([['11 - s'], ['-6 + s'], ['2 - s'], ['s']])} 
  = \py{myprint([['-1'], ['1'], ['-1'], [1]])} * s + \py{myprint([[11], [-6], [2], [0]])}
  $ 

  \item \begin{pycode}
A = np.array([
  [1, -1, 2, -2, 1],
  [2, 1, 0, 3, 4],
  [2, 3, 2, 0, 6]
])
sol, ops = myalgo.rref(A)
  \end{pycode}
  \begin{align*}
    \py{ops}
  \end{align*}

  Solution: Line, $
  \py{myprint(['x_1', 'x_2', 'x_3', 'x_4'])} 
  = \py{myprint([
      sol[0, 4]-sol[0, 3] * s,
      sol[1, 4]-sol[1, 3] * s,
      sol[2, 4]-sol[2, 3] * s,
      s
    ])} 
  = \py{myprint([
      -sol[0, 3],
      -sol[1, 3],
      -sol[2, 3],
      1
    ])} * s + \py{myprint([
      sol[0, 4],
      sol[1, 4],
      sol[2, 4],
      0
    ])}
  $

  \item \begin{pycode}
A = np.array([
  [3, 7, 2, 1],
  [1, -1, 1, 2],
  [5, 5, 4, 5]
])
sol, ops = myalgo.rref(A)
  \end{pycode}
  \begin{align*}
    \py{ops}
  \end{align*}

  Solution: Line, $
  \py{myprint(['x', 'y', 'z'])} 
  = \py{myprint([
      sol[0, 3]-sol[0, 2] * s,
      sol[1, 3]-sol[1, 2] * s,
      s
    ])} 
  = \py{myprint([
      -sol[0, 2],
      -sol[1, 2],
      1
    ])} * s + \py{myprint([
      sol[0, 3],
      sol[1, 3],
      0
    ])}
  $ \\
  Linear combination: $R_3 = R_1 + 2R_2$

  \item \begin{pycode}
A = np.array([
  [2, -3, 2, 1],
  [1, -6, 1, 2],
  [-1, -3, -1, 1]
])
sol, ops = myalgo.rref(A)
  \end{pycode}
  \begin{align*}
    \py{ops}
  \end{align*}

  Solution: Plane, $
  \py{myprint(['x', 'y', 'z'])} 
  = \py{myprint([
      's',
      sol[1, 3],
      'k'
    ])} 
  = \py{myprint([1, 0, 0])} * s
  + \py{myprint([0, 0, 1])} * k
  + \py{myprint([0, sol[1, 3], 0])}
  $ \\
  Linear combination: $R_3 = R_2 - R_1$

  \item \begin{pycode}
A = np.array([
  [2, 3, -1, -2],
  [1, -1, 1, 2],
  [2, 3, 4, 5]
])
sol, ops = myalgo.rref(A)
  \end{pycode}
  \begin{align*}
    \py{ops}
  \end{align*}

  Solution: Point, $
  \py{myprint(['x', 'y', 'z'])} 
  = \py{myprint(sol[:, 3])} 
  $ 

  \item \begin{pycode}
A = np.array([
  [1, 1, 1, 1, 1],
  [2, -2, 1, 2, 3],
  [2, 6, 3, 2, 1],
  [5, -3, 3, 5, 7]
])
sol, ops = myalgo.rref(A)
  \end{pycode}
  \begin{align*}
    \py{ops}
  \end{align*}

  Solution: Plane, $
  \py{myprint(['x', 'y', 'z', 'w'])} 
  = \py{myprint([
      sol[0, 4]-sol[0, 2] * k - s,
      sol[1, 4]-sol[1, 2] * k,
      k,
      s
    ])} 
  = \py{myprint([
      -1,
      0,
      0,
      1
    ])} * s + \py{myprint([
      -sol[0, 2],
      -sol[1, 2],
      1,
      0
    ])} * k + \py{myprint([
      sol[0, 4],
      sol[1, 4],
      0,
      0
    ])}
  $ \\ 
  Linear combination: $R_3 = 4R_1 - R_2$, $R_4 = R_1 + 2R_2$

  \item \begin{pycode}
A = np.array([
  [1, 1, 1, 1, 1],
  [2, -2, 1, 2, 3],
  [2, 6, 3, 2, 1],
  [5, -3, 3, 5, 8]
])
sol, ops = myalgo.rref(A)
  \end{pycode}
  \begin{align*}
    \py{ops}
  \end{align*}

  Solution: Inconsistent \\
  Linear combination: $R_3 = 4R_1 - R_2$

  \item \begin{pycode}
A = np.array([
  [1, 2, -1, -1, 1],
  [-3, -3, 1, 10, -6],
  [-5, -4, 1, 18, -11],
  [-2, 5, -4, 16, -11]
])
sol, ops = myalgo.rref(A)
  \end{pycode}
  \begin{align*}
    \py{ops}
  \end{align*}

  Solution: Line, $
  \py{myprint(['x', 'y', 'z', 'w'])} 
  = \py{myprint([
      sol[0, 4]-sol[0, 2] * s,
      sol[1, 4]-sol[1, 2] * s,
      s,
      0
    ])} 
  = \py{myprint([
      -sol[0, 2],
      -sol[1, 2],
      1,
      0
    ])} * s + \py{myprint([
      sol[0, 4],
      sol[1, 4],
      0,
      0
    ])}
  $

\end{enumerate}
  
\end{question}

\begin{question}{1.57}
  Create an example of a system of five equations in five unknowns that has rank 2. How about one with rank 3? Rank 1?

  \ans 
  
  \begin{pycode}
A = np.random.randint(1, 10000, (1, 6))
B = np.random.randint(1, 10000, (1, 6))
C = np.random.randint(1, 10000, (1, 6))
  \end{pycode}

  \begin{align*}
    \py{myprint.print_augmented_matrix_num(
      np.r_[A, A * 10, A * 7, A * 2, A * 4]
    )} & & \text{Rank 1} \\
    \py{myprint.print_augmented_matrix_num(
      np.r_[A, B, A - B, A * 2, B * 4]
    )} & & \text{Rank 2} \\
    \py{myprint.print_augmented_matrix_num(
      np.r_[A, B, C, A * 2, B * 4]
    )} & & \text{Rank 3}
  \end{align*}
  
\end{question}

% \begin{pycode}
% A = np.random.randint(1, 100, (3, 3))
% I = np.eye(3).astype(np.int)
% C = np.c_[A, I]
% sol, ops = myalgo.rref(C)
% \end{pycode}

% \begin{align*}
%   \py{ops} 
% \end{align*}

\end{document}