\documentclass[letterpaper]{article}

%% Sets page size and margins
\usepackage[letterpaper,top=3cm,bottom=2cm,left=3cm,right=3cm,marginparwidth=1.75cm]{geometry}
 
%% Useful packages
\usepackage{amsmath}
\usepackage{amsfonts}
\usepackage{mathtools}
\usepackage{graphicx}
\usepackage{enumitem}
\usepackage[colorlinks=true, allcolors=blue]{hyperref}
\usepackage{pythontex}
\usepackage{lastpage}
\usepackage{fancyhdr}

\pagestyle{fancy}
\graphicspath{ {images/} }
\newcommand{\true}{$T$}
\newcommand{\false}{$F$}
\newcommand{\ans}{\textit{Answer: }}
\newcommand{\prf}{\textbf{Proof:}}
\newenvironment{question}[2][Question]{\begin{trivlist}
\item[\hskip \labelsep {\bfseries #1}\hskip \labelsep {\bfseries #2.}]}{\end{trivlist}}

% Fancy matrix
\makeatletter
\renewcommand*\env@matrix[1][*\c@MaxMatrixCols c]{%
  \hskip -\arraycolsep
  \let\@ifnextchar\new@ifnextchar
  \array{#1}}
\makeatother

% My scripts setup
\begin{pycode}
import sympy
import numpy as np

import sys
sys.path.append('..')

from utils import Utils
utils = Utils(pytex)
sympify = np.vectorize(sympy.sympify)
\end{pycode}

\newcommand{\printobj}[1]{\py{utils.print.print_object(#1)}}

\title{MA 351, HW 3} 
\lhead{MA 351, HW 3}

\author{Elnard Utiushev}
\rhead{Elnard Utiushev}
\cfoot{\thepage\ of \pageref{LastPage}}

\begin{document}

\maketitle

Section 2.1: True/False: 2.2, 2.3, 2.4; Exercises: 2.1, 2.7, 2.8 \\
Section 2.2: True/False: 2.10, 2.11, 2.12, 2.15, 2.16, 2.17, 2.19; Exercises: 2.32, 2.35, 2.36

\section{Section 2.1}
\subsection{True/False}

\begin{question}{2.2}
    I start with a certain 4×6 matrix A and reduce, obtaining
    $$M = \left[ \begin{array} { c c c c c c } { 1 } & { 0 } & { 1 } & { 1 } & { 0 } & { 0 } \\ { 0 } & { 1 } & { 1 } & { 3 } & { - 2 } & { 3 } \\ { 0 } & { 0 } & { 0 } & { 0 } & { 1 } & { 1 } \\ { 0 } & { 0 } & { 0 } & { 0 } & { 0 } & { 0 } \end{array} \right]$$

    Which of the following statements about $A$ are guaranteed to be true, where $A_i$ denotes the ith column of $A$.

    \begin{enumerate}[label=\textbf{(\alph*)}]
        \item The columns of A are linearly independent \\ 
        \ans False, since we got a couple of free variables in $M$

        \item $A_4$ is a linear combination of $A_1$, $A_2$, and $A_5$ \\
        \ans True, $A_4 = A_1 + 3A_2 + 0A_5$

        \item $A_3$ is a linear combination of $A_1$, $A_2$, and $A_5$ \\
        \ans True, $A_3 = A_1 + A_2 + 0A_5$

        \item $A_3$ is a linear combination of $A_1$ and $A_2$ \\
        \ans True, $A_3 = A_1 + A_2$

        \item $A_4 = A_1 - 2A_2$ \\ 
        \ans False, $A_4 \neq A_1 - 2A_2$
    \end{enumerate}

\end{question}

\begin{question}{2.3}
    Suppose that A and B are n × n matrices that both have linearly independent
    columns. Then A and B have the same reduced echelon form.

    \ans False, 

    $$
    rref \printobj{[
        [1, 2, 0],
        [0, 1, 0],
        [0, 0, 1]
    ]}
    \neq
    rref \printobj{[
        [1, 5, 0],
        [0, 1, 0],
        [0, 0, 1]
    ]}
    $$

\end{question}

\begin{question}{2.4}
    Suppose that A is an n×n matrix and B is an n×1 column vector such that 
    the equation $AX = B$ has an infinite number of solutions. Then the columns
    of A are linearly dependent.

    \ans True, in order of the equation to have an infinite number of solutions
    it should have at least one free variable in its row echelon form, which means 
    that colums of A are linearly dependent
\end{question}

\subsection{Exercises}

\begin{question}{2.1}
    Test the given matrices for linear dependence using the test for linear 
    independence. Then find a basis for their span and express the other vectors 
    (if there are any) as linear combinations of the basis elements.

\begin{pycode}

problems = [
    [
        [[1], [2], [1], [2]],
        [[2], [3], [-1], [0]],
        [[1], [0], [1], [0]]
    ],
    [
        [[1], [0], [0]],
        [[0], [1], [0]],
        [[0], [0], [1]]
    ],
    [
        [[1, 2, 1]],
        [[3, -1, 2]],
        [[7, -7, 4]]
    ],
    [
        [[2, 3], [0, 1]],
        [[1, 3], [0, 0]],
        [[17, 0], [9, 1]],
        [[0, 5], [0, 6]]
    ],
    [
        [[2, -2], [3, -1], [0, 2]],
        [[4, -1], [2, 3], [1, 0]],
        [[8, -2], [4, 6], [2, 0]]
    ],
    [
        [[1, 1], [4, 3]],
        [[2, 1], [3, 0]],
        [[2, 1], [2, -1]],
        [[1, 1], [6, 5]]
    ],
    [
        [[1, 2], [3, 2]],
        [[2, 1], [3, 4]],
        [[5, -2], [3, 10]]
    ],
    [
        [[1, 0], [0, 0]],
        [[0, 1], [0, 0]],
        [[0, 0], [1, 0]],
        [[0, 0], [0, 1]]
    ],
    [
        [[1], [2], [3], [1]],
        [[2], [0], [1], [0]],
        [[1], [1], [1], [1]]
    ],
    [
        [[4, 2, -1]],
        [[3, 3, 2]],
        [[1, 0, 1]]
    ],
    [
        [[3], [2], [4], [5]],
        [[-4], [-3], [-5], [-6]],
        [[2], [-3], [7], [12]]
    ]
]

\end{pycode}

\begin{enumerate}[label=\textbf{(\alph*)}]
    \begin{pycode}
problem = problems[0]
dep_system = np.column_stack(tuple(map(np.ravel, problem)))
rref, _ = utils.algorithms.rref(dep_system)
pivots = utils.algorithms.get_pivots(rref)
basis = ', '.join(map(lambda pivot: '$A_{}$'.format(pivot[0] + 1), pivots))
    \end{pycode}
    \item
    $$\py{utils.print.print_list_of_arrays(problem)}$$
    \ans 
    $$\printobj{dep_system} \rightarrow \printobj{rref}$$
    Basis: \py{basis} \\
    \py{'Independent' if len(pivots) == rref.shape[1] else 'Dependent'}
% -----------------------
    \begin{pycode}
problem = problems[1]
dep_system = np.column_stack(tuple(map(np.ravel, problem)))
rref, _ = utils.algorithms.rref(dep_system)
pivots = utils.algorithms.get_pivots(rref)
basis = ', '.join(map(lambda pivot: '$A_{}$'.format(pivot[0] + 1), pivots))
    \end{pycode}
    \item
    $$\py{utils.print.print_list_of_arrays(problem)}$$
    \ans 
    $$\printobj{dep_system} \rightarrow \printobj{rref}$$
    Basis: \py{basis} \\ 
    \py{'Independent' if len(pivots) == rref.shape[1] else 'Dependent'}
% -----------------------
    \begin{pycode}
problem = problems[2]
dep_system = np.column_stack(tuple(map(np.ravel, problem)))
rref, _ = utils.algorithms.rref(dep_system)
pivots = utils.algorithms.get_pivots(rref)
basis = ', '.join(map(lambda pivot: '$A_{}$'.format(pivot[0] + 1), pivots))
    \end{pycode}
    \item
    $$\py{utils.print.print_list_of_arrays(problem)}$$
    \ans 
    $$\printobj{dep_system} \rightarrow \printobj{rref}$$
    Basis: \py{basis} \\
    Linear combination: $A_3 = -2A_1 + 3A_2$ \\
    \py{'Independent' if len(pivots) == rref.shape[1] else 'Dependent'}
% -----------------------
    \begin{pycode}
problem = problems[3]
dep_system = np.column_stack(tuple(map(np.ravel, problem)))
rref, _ = utils.algorithms.rref(dep_system)
pivots = utils.algorithms.get_pivots(rref)
basis = ', '.join(map(lambda pivot: '$A_{}$'.format(pivot[0] + 1), pivots))
    \end{pycode}
    \item
    $$\py{utils.print.print_list_of_arrays(problem)}$$
    \ans 
    $$\printobj{dep_system} \rightarrow \printobj{rref}$$
    Basis: \py{basis} \\
    \py{'Independent' if len(pivots) == rref.shape[1] else 'Dependent'}
% -----------------------
    \begin{pycode}
problem = problems[4]
dep_system = np.column_stack(tuple(map(np.ravel, problem)))
rref, _ = utils.algorithms.rref(dep_system)
pivots = utils.algorithms.get_pivots(rref)
basis = ', '.join(map(lambda pivot: '$A_{}$'.format(pivot[0] + 1), pivots))
    \end{pycode}
    \item
    $$\py{utils.print.print_list_of_arrays(problem)}$$
    \ans 
    $$\printobj{dep_system} \rightarrow \printobj{rref}$$
    Basis: \py{basis} \\
    Linear combination: $A_3 = 0A_1 + 2A_2$ \\
    \py{'Independent' if len(pivots) == rref.shape[1] else 'Dependent'}
% -----------------------
    \begin{pycode}
problem = problems[5]
dep_system = np.column_stack(tuple(map(np.ravel, problem)))
rref, _ = utils.algorithms.rref(dep_system)
pivots = utils.algorithms.get_pivots(rref)
basis = ', '.join(map(lambda pivot: '$A_{}$'.format(pivot[0] + 1), pivots))
    \end{pycode}
    \item
    $$\py{utils.print.print_list_of_arrays(problem)}$$
    \ans 
    $$\printobj{dep_system} \rightarrow \printobj{rref}$$
    Basis: \py{basis} \\
    Linear combination: $A_4 = A_1 + 2A_2 - 2A_3$ \\ 
    \py{'Independent' if len(pivots) == rref.shape[1] else 'Dependent'}
% -----------------------
    \begin{pycode}
problem = problems[6]
dep_system = np.column_stack(tuple(map(np.ravel, problem)))
rref, _ = utils.algorithms.rref(dep_system)
pivots = utils.algorithms.get_pivots(rref)
basis = ', '.join(map(lambda pivot: '$A_{}$'.format(pivot[0] + 1), pivots))
    \end{pycode}
    \item
    $$\py{utils.print.print_list_of_arrays(problem)}$$
    \ans 
    $$\printobj{dep_system} \rightarrow \printobj{rref}$$
    Basis: \py{basis} \\
    Linear combination: $A_3 = -3A_1 + 4A_2$ \\
    \py{'Independent' if len(pivots) == rref.shape[1] else 'Dependent'}
% -----------------------
    \begin{pycode}
problem = problems[7]
dep_system = np.column_stack(tuple(map(np.ravel, problem)))
rref, _ = utils.algorithms.rref(dep_system)
pivots = utils.algorithms.get_pivots(rref)
basis = ', '.join(map(lambda pivot: '$A_{}$'.format(pivot[0] + 1), pivots))
    \end{pycode}
    \item
    $$\py{utils.print.print_list_of_arrays(problem)}$$
    \ans 
    $$\printobj{dep_system} \rightarrow \printobj{rref}$$
    Basis: \py{basis} \\
    \py{'Independent' if len(pivots) == rref.shape[1] else 'Dependent'}
% -----------------------
    \begin{pycode}
problem = problems[8]
dep_system = np.column_stack(tuple(map(np.ravel, problem)))
rref, _ = utils.algorithms.rref(dep_system)
pivots = utils.algorithms.get_pivots(rref)
basis = ', '.join(map(lambda pivot: '$A_{}$'.format(pivot[0] + 1), pivots))
    \end{pycode}
    \item
    $$\py{utils.print.print_list_of_arrays(problem)}$$
    \ans 
    $$\printobj{dep_system} \rightarrow \printobj{rref}$$
    Basis: \py{basis} \\
    \py{'Independent' if len(pivots) == rref.shape[1] else 'Dependent'}
% -----------------------
    \begin{pycode}
problem = problems[9]
dep_system = np.column_stack(tuple(map(np.ravel, problem)))
rref, _ = utils.algorithms.rref(dep_system)
pivots = utils.algorithms.get_pivots(rref)
basis = ', '.join(map(lambda pivot: '$A_{}$'.format(pivot[0] + 1), pivots))
    \end{pycode}
    \item
    $$\py{utils.print.print_list_of_arrays(problem)}$$
    \ans 
    $$\printobj{dep_system} \rightarrow \printobj{rref}$$
    Basis: \py{basis} \\
    \py{'Independent' if len(pivots) == rref.shape[1] else 'Dependent'}
% -----------------------
    \begin{pycode}
problem = problems[10]
dep_system = np.column_stack(tuple(map(np.ravel, problem)))
rref, _ = utils.algorithms.rref(dep_system)
pivots = utils.algorithms.get_pivots(rref)
basis = ', '.join(map(lambda pivot: '$A_{}$'.format(pivot[0] + 1), pivots))
    \end{pycode}
    \item
    $$\py{utils.print.print_list_of_arrays(problem)}$$
    \ans 
    $$\printobj{dep_system} \rightarrow \printobj{rref}$$
    Basis: \py{basis} \\
    Linear combination: $A_3 = 18A_1 + 13A_2$ \\
    \py{'Independent' if len(pivots) == rref.shape[1] else 'Dependent'}
\end{enumerate}
\end{question}

\begin{question}{2.7}
    Use the test for linear independence to prove that the rows of the following
    3 × 6 matrix are linearly independent:

    \begin{pycode}
A = sympify(np.array([[1, 'a', 'b', 'c', 'd', 'e'], [0, 0, 1, 'f', 'g', 'h'], [0, 0, 0, 0, 1, 'k']]))
rref, _ = utils.algorithms.rref(A.T)
    \end{pycode}

    $$A = \printobj{A}$$

    \ans 
    $$\printobj{A.T} \xrightarrow{rref} \printobj{rref}$$
\end{question}

\begin{question}{2.8}
    Let $A = \left[ A _ { 1 } , A _ { 2 } , A _ { 3 } \right]$ be a $3 \times 3$
    matrix with linearly independent columns $A _ { i }$ .

    \begin{enumerate}[label=\textbf{(\alph*)}]
        \item Explain why the row reduced form of $A$ is the following matrix $R$
        $$
        R = \left[ \begin{array} { l l l } { 1 } & { 0 } & { 0 } \\ { 0 } & { 1 } & { 0 } \\ { 0 } & { 0 } & { 1 } \end{array} \right]
        $$

        \ans Since columns are linearly independent, by presenting the matrix 
        in rref, we solve dependency equation. Matrix $R$ shows that solution 
        to that equation is all zeroes which means that columns are linearly 
        independent

        \item Let $A = \left[ A _ { 1 } , A _ { 2 } , A _ { 3 } , A _ { 4 } , A _ { 5 } \right]$
        be a $3 \times 5$ matrix such that the first three
        columns are linearly independent. Explain why the pivot columns must
        be the first three. 

        \ans Since the first three columns are linearly independent,
        rref of $\left[ A _ { 1 } , A _ { 2 } , A _ { 3 }\right]$ will result
        in no free variables, which means that the pivot columns will
        be the first three
    \end{enumerate}

\end{question}
 

\section{Section 2.2}
\subsection{True/False}

\begin{question}{2.10}
    $\left\{ [ 17,6 , - 4 ] ^ { t } , [ 2,3,3 ] ^ { t } , [ 19,9 , - 1 ] ^ { t } \right\}$ 
    does not span $\mathbb { R } ^ { 3 }$

    \begin{pycode}
A = np.array([17, 6, -4])
B = np.array([2, 3, 3])
C = np.array([19, 9, -1])
dep_system = np.c_[A, B, C]
rref, _ = utils.algorithms.rref(dep_system)
pivots = utils.algorithms.get_pivots(rref)
basis = ', '.join(map(lambda pivot: '$A_{}$'.format(pivot[0] + 1), pivots))
    \end{pycode}

    \ans True, because matrices are not linearly independent
    $$\printobj{dep_system} \rightarrow \printobj{rref}$$
    Basis: \py{basis}
\end{question}

\begin{question}{2.11}
    $\left\{ [ 1,1 ] ^ { t } , [ 1,2 ] ^ { t } , [ 4,7 ] ^ { t } \right\}$ spans $\mathbb { R } ^ { 2 }$

    \begin{pycode}
A = np.array([1, 1])
B = np.array([1, 2])
C = np.array([4, 7])
dep_system = np.c_[A, B, C]
rref, _ = utils.algorithms.rref(dep_system)
pivots = utils.algorithms.get_pivots(rref)
basis = ', '.join(map(lambda pivot: '$A_{}$'.format(pivot[0] + 1), pivots))
    \end{pycode}

    \ans True, because first two matrices are linearly independent
    $$\printobj{dep_system} \rightarrow \printobj{rref}$$
    Basis: \py{basis}
\end{question}

\begin{question}{2.12}
    Let $\mathcal { W }$ be a two-dimensional subspace of $\mathbb { R } ^ { 3 } .$ 
    Then two of the following three
    vectors span $\mathcal { W } : X = [ 1,0,0 ] ^ { t } , Y = [ 0,1,0 ] ^ { t } , Z = [ 0,0,1 ] ^ { t }$

    \ans False, for example, let $\mathcal { W }$ be a two-dimensional subspace 
    spanned by vectors $\printobj{[1, 0, 3]}, \printobj{[0, 1, 3]}$
\end{question}

\begin{question}{2.15}
    Suppose that $\mathcal { W }$ is a four-dimensional subspace of
    $\mathbb { R } ^ { 7 }$ and $X _ { 1 } , X _ { 2 } , X _ { 3 } ,$ and $X _ { 4 }$
    are vectors that belong to $\mathcal { W } .$ Then $\left\{ X _ { 1 } , X _ { 2 } ,
    X _ { 3 } , X _ { 4 } \right\}$ spans $\mathcal { W }$.

    \ans False, since vectors $X _ { 1 } , X _ { 2 } , X _ { 3 } ,$ and $X _ { 4 }$
    can be linearly dependent and still belong to $\mathcal { W }$
\end{question}

\begin{question}{2.16}
    Suppose that $\left\{ X _ { 1 } , X _ { 2 } , X _ { 3 } , X _ { 4 } ,
     X _ { 5 } \right\}$ spans a four-dimensional vector space $\mathcal { W }$ of
    $\mathbb { R } ^ { 7 } .$ Then $\left\{ X _ { 1 } , X _ { 2 } , X _ { 3 } , 
    X _ { 4 } \right\}$ also spans $\mathcal { W }$

    \ans False, since vector space $\mathcal { W }$ is four-dimensional, one of 
    the vectors $\left\{ X _ { 1 } , X _ { 2 } , X _ { 3 } , X _ { 4 } ,
    X _ { 5 } \right\}$ is a linear combination of the others, but it is not guaranteed
    that it is $X _ { 5 }$. 
\end{question}

\begin{question}{2.17}
    Suppose that $S = \left\{ X _ { 1 } , X _ { 2 } , X _ { 3 } , X _ { 4 } , 
    X _ { 5 } \right\}$ spans a four-dimensional subspace $\mathcal { W }$ of
    $\mathbb { R } ^ { 7 } .$ Then $S$ contains a basis for $\mathcal { W } .$

    \ans True
\end{question}

\begin{question}{2.19}
    Suppose that $\mathcal { W }$ is a four-dimensional subspace of $\mathbb { R } ^ { 7 }$ that is spanned by
    $\left\{ X _ { 1 } , X _ { 2 } , X _ { 3 } , X _ { 4 } \right\} .$ 
    Then one of the $X _ { i }$ must be a linear combination of the others.

    \ans False, since $\mathcal { W }$ is a four-dimensional subspace, so its basis
    has exactly 4 elements. 
\end{question}

\subsection{Exercises}

\begin{question}{2.32}
    Prove that the given sets $\mathcal { W }$ are subspaces of $\mathbb { R } ^ { n }$ 
    for the appropriate $n .$ Find
    spanning sets for these spaces and find at least two different bases for each
    space. Give the dimension of each space. 

    \begin{enumerate}[label=\textbf{(\alph*)}]
        \begin{pycode}
problem = [
    np.array([1, 2, 1, 1]),
    np.array([1, 1, 1, 2]),
    np.array([2, 3, 2, 3])
]
dep_system = np.column_stack(tuple(map(np.ravel, problem)))
rref, _ = utils.algorithms.rref(dep_system)
pivots = utils.algorithms.get_pivots(rref)
basis = ', '.join(map(lambda pivot: '$A_{}$'.format(pivot[0] + 1), pivots))
        \end{pycode}
        \item $\mathcal { W } = \left\{ [ a + b + 2 c , 2 a + b + 3 c , a + b + 2 c , a + 2 b + 3 c ] ^ { t } | a , b , c \in \mathbb { R } \right\}$
        
        \ans 
        $$\py{utils.print.print_list_of_arrays(problem)}$$
        $$\printobj{dep_system} \rightarrow \printobj{rref}$$
        Dimension: \printobj{len(pivots)} \\ 
        Basis 1:  $\printobj{problem[0]}$, $\printobj{problem[1]}$ \\ 
        Basis 2:  $\printobj{problem[0] * 2}$, $\printobj{problem[1] * 2}$
% ----------------------
        \begin{pycode}
problem = [
    np.array([1, 2, 1]),
    np.array([0, 1, 1]),
    np.array([2, 3, 1])
]
dep_system = np.column_stack(tuple(map(np.ravel, problem)))
rref, _ = utils.algorithms.rref(dep_system)
pivots = utils.algorithms.get_pivots(rref)
basis = ', '.join(map(lambda pivot: '$A_{}$'.format(pivot[0] + 1), pivots))
        \end{pycode}
        \item $\mathcal { W } = \left\{ [ a + 2 c , 2 a + b + 3 c , a + b + c ] ^ { t } | a , b , c \in \mathbb { R } \right\}$
        
        \ans 
        $$\py{utils.print.print_list_of_arrays(problem)}$$
        $$\printobj{dep_system} \rightarrow \printobj{rref}$$
        Dimension: \printobj{len(pivots)} \\ 
        Basis 1:  $\printobj{problem[0]}$, $\printobj{problem[1]}$ \\ 
        Basis 2:  $\printobj{problem[0] * 2}$, $\printobj{problem[1] * 2}$
% ----------------------
        \begin{pycode}
problem = [
    np.array([1, 2, 1]),
    np.array([1, 1, 1]),
    np.array([2, 3, 1])
]
dep_system = np.column_stack(tuple(map(np.ravel, problem)))
rref, _ = utils.algorithms.rref(dep_system)
pivots = utils.algorithms.get_pivots(rref)
basis = ', '.join(map(lambda pivot: '$A_{}$'.format(pivot[0] + 1), pivots))
        \end{pycode}
        \item $\left\{ [ a + b + 2 c , 2 a + b + 3 c , a + b + c ] ^ { t } | a , b , c \in \mathbb { R } \right\}$
        
        \ans 
        $$\py{utils.print.print_list_of_arrays(problem)}$$
        $$\printobj{dep_system} \rightarrow \printobj{rref}$$
        Dimension: \printobj{len(pivots)} \\ 
        Basis 1:  $\printobj{problem[0]}$, $\printobj{problem[1]}$, $\printobj{problem[2]}$ \\ 
        Basis 2:  $\printobj{problem[0] * 2}$, $\printobj{problem[1] * 2}$, $\printobj{problem[2] * 2}$
% ----------------------
        \begin{pycode}
problem = [
    np.array([1, -2, 6, 3]),
    np.array([2, -2, 4, 1]),
    np.array([-4, 2, 0, 3]),
    np.array([5, -6, 14, 5])
]
dep_system = np.column_stack(tuple(map(np.ravel, problem)))
rref, _ = utils.algorithms.rref(dep_system)
pivots = utils.algorithms.get_pivots(rref)
basis = ', '.join(map(lambda pivot: '$A_{}$'.format(pivot[0] + 1), pivots))
        \end{pycode}
        \item $\mathcal { W } = \{ [ a + 2 b - 4 c + 5 d , - 2 a - 2 b + 2 c - 6 d , 6 a + 4 b + 14 d, 3 a + b + 3 c + 5 d ] | a , b , c , d \in \mathbb { R } \}$
        
        \ans 
        $$\py{utils.print.print_list_of_arrays(problem)}$$
        $$\printobj{dep_system} \rightarrow \printobj{rref}$$
        Dimension: \printobj{len(pivots)} \\ 
        Basis 1:  $\printobj{problem[0]}$, $\printobj{problem[1]}$ \\ 
        Basis 2:  $\printobj{problem[0] * 2}$, $\printobj{problem[1] * 2}$
    \end{enumerate}
\end{question}

\begin{question}{2.35}
    Find a basis and give the dimension for the following spaces of matrices $A$
    
    \begin{enumerate}[label=\textbf{(\alph*)}]
        \item $2 \times 2 , A = A ^ { t }$
        
        \ans 
        $$
        \printobj{[
            ['a', 'b'],
            ['b', 'a']
        ]}
        $$

        Dimension: 2 \\
        Basis: $\printobj{[[1, 0], [0, 1]]}, \printobj{[[0, 1], [1, 0]]}$
% ------------------------
        \item $3 \times 3 , A = A ^ { t }$
        
        \ans 
        $$
        \printobj{[
            ['a', 'b', 'c'],
            ['b', 'd', 'e'],
            ['c', 'e', 'f']
        ]}
        $$

        Dimension: 6 \\
        Basis: $
        \printobj{[[1, 0, 0], [0, 0, 0], [0, 0, 0]]},
        \printobj{[[0, 1, 0], [1, 0, 0], [0, 0, 0]]},
        \printobj{[[0, 0, 1], [0, 0, 0], [1, 0, 0]]},
        \printobj{[[0, 0, 0], [0, 1, 0], [0, 0, 0]]},
        \printobj{[[0, 0, 0], [0, 0, 1], [0, 1, 0]]},
        \printobj{[[0, 0, 0], [0, 0, 0], [0, 0, 1]]}
        $
% ------------------------
        \item $2 \times 2 , A = - A ^ { t }$
        
        \ans 
        $$
        \printobj{[
            [0, '-b'],
            ['b', 0],
        ]}
        $$

        Dimension: 1 \\
        Basis: $
        \printobj{[[0, -1], [1, 0]]}
        $
% ------------------------
        \item $3 \times 3$ upper triangular.
        
        \ans 
        $$
        \printobj{[
            ['a', 'b', 'c'],
            [0, 'd', 'e'],
            [0, 0, 'f']
        ]}
        $$

        Dimension: 6 \\
        Basis: $
        \printobj{[[1, 0, 0], [0, 0, 0], [0, 0, 0]]},
        \printobj{[[0, 1, 0], [0, 0, 0], [0, 0, 0]]},
        \printobj{[[0, 0, 1], [0, 0, 0], [0, 0, 0]]},
        \printobj{[[0, 0, 0], [0, 1, 0], [0, 0, 0]]},
        \printobj{[[0, 0, 0], [0, 0, 1], [0, 0, 0]]},
        \printobj{[[0, 0, 0], [0, 0, 0], [0, 0, 1]]}
        $
% ------------------------
        \item $3 \times 3$ lower triangular.
        
        \ans 
        $$
        \printobj{sympify([
            ['a', 'b', 'c'],
            [0, 'd', 'e'],
            [0, 0, 'f']
        ]).T}
        $$

        Dimension: 6 \\
        Basis: $
        \printobj{sympify([[1, 0, 0], [0, 0, 0], [0, 0, 0]]).T},
        \printobj{sympify([[0, 1, 0], [0, 0, 0], [0, 0, 0]]).T},
        \printobj{sympify([[0, 0, 1], [0, 0, 0], [0, 0, 0]]).T},
        \printobj{sympify([[0, 0, 0], [0, 1, 0], [0, 0, 0]]).T},
        \printobj{sympify([[0, 0, 0], [0, 0, 1], [0, 0, 0]]).T},
        \printobj{sympify([[0, 0, 0], [0, 0, 0], [0, 0, 1]]).T}
        $
    \end{enumerate}
\end{question}

\begin{question}{2.36}
    Find a basis for the subspace of $M ( 2,2 )$ spanned by the following matrices.
    What is the dimension of this subspace?

    \begin{pycode}
problem = [
    np.array([[1, 2], [1, 3]]),
    np.array([[1, 2], [2, 4]]),
    np.array([[-1, -2], [-3, -5]]),
    np.array([[-1, -2], [0, -2]])
]
dep_system = np.column_stack(tuple(map(np.ravel, problem)))
rref, _ = utils.algorithms.rref(dep_system)
pivots = utils.algorithms.get_pivots(rref)
basis = ', '.join(map(lambda pivot: '$A_{}$'.format(pivot[0] + 1), pivots))
    \end{pycode}

    $$\py{utils.print.print_list_of_arrays(problem)}$$
    
    \ans 
    $$\printobj{dep_system} \rightarrow \printobj{rref}$$
    Basis: \py{basis} \\
    Dimension: \py{len(pivots)}
    
\end{question}


\end{document}
