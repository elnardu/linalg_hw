\documentclass[letterpaper]{article}

%% Sets page size and margins
\usepackage[letterpaper,top=3cm,bottom=2cm,left=3cm,right=3cm,marginparwidth=1.75cm]{geometry}
 
%% Useful packages
\usepackage{amsmath}
\usepackage{amsfonts}
\usepackage{mathtools}
\usepackage{graphicx}
\usepackage{enumitem}
\usepackage[colorlinks=true, allcolors=blue]{hyperref}
\usepackage{pythontex}
\usepackage{lastpage}
\usepackage{fancyhdr}

\pagestyle{fancy}
\graphicspath{ {images/} }
\newcommand{\true}{$T$}
\newcommand{\false}{$F$}
\newcommand{\ans}{\textit{Answer: }}
\newcommand{\prf}{\textbf{Proof:}}
\newenvironment{question}[2][Question]{\begin{trivlist}
\item[\hskip \labelsep {\bfseries #1}\hskip \labelsep {\bfseries #2.}]}{\end{trivlist}}

% Fancy matrix
\makeatletter
\renewcommand*\env@matrix[1][*\c@MaxMatrixCols c]{%
  \hskip -\arraycolsep
  \let\@ifnextchar\new@ifnextchar
  \array{#1}}
\makeatother

% My scripts setup
\begin{pycode}
import sympy
import numpy as np

import sys
sys.path.append('..')

from utils import Utils
utils = Utils(pytex)
sympify = np.vectorize(sympy.sympify)
\end{pycode}

\newcommand{\printobj}[1]{\py{utils.print.print_object(#1)}}

\title{MA 351, HW 4} 
\lhead{MA 351, HW 4}

\author{Elnard Utiushev}
\rhead{Elnard Utiushev}
\cfoot{\thepage\ of \pageref{LastPage}}

\begin{document}

\maketitle

Section 2.3: True/False: 2.20, 2.25; Exercises: 2.66, 2.70, 2.80 \\ 
Section 3.1: True/False: 3.1, 3.4, 3.5; Exercises: 3.9, 3.10, 3.17, 3.21, 3.24 \\
Section 3.2: True/False: 3.14, 3.15, 3.18; Exercises: 3.34, 3.49

\section{Section 2.3}
\subsection{True/False}

\begin{question}{2.20}
    Suppose that $A$ is a $3 \times 5$ matrix such that the vectors 
    $ X = [ 1,1,1,1,1 ] ^ { t } , Y = [ 0,1,1,1,1 ] ^ { t }$,
    and $Z = [ 0,0,1,1,1 ] ^ { t }$ belong to the nullspace of A. 
    Classify the following statements as true or false.

\begin{pycode}
X = np.array([1, 1, 1, 1, 1]).reshape(1, -1)
Y = np.array([0, 1, 1, 1, 1]).reshape(1, -1)
Z = np.array([0, 0, 1, 1, 1]).reshape(1, -1)

A = np.r_[X, Y, Z]
rref, _ = utils.algorithms.rref(A)
\end{pycode}

$$\printobj{A} \rightarrow \printobj{rref}$$

    \begin{enumerate}[label=(\alph*)]
        \item The rows of $A$ are dependent. \\
        \ans False, since we need at least 3 independent row vectors, for 
        $X, Y, Z$ to belong to the nullspace.

        \item $A X = B$ has a solution for all $B \in \mathbb { R } ^ { 3 }$ . \\
        \ans True, because $A$ is a $3 \times 5$ matrix

        \item The solution to $A X = B ,$ when it exists, is unique. \\
        \ans False
    \end{enumerate}

\end{question}

\begin{question}{2.25}
    Suppose that $A$ is a $3 \times 7$ matrix such that the equation $A X = B$ is solvable for
    all $B$ in $\mathbb { R } ^ { 3 } .$ Then $A$ has rank 3.

    \ans True, according to the theorem 2.16
\end{question}

\subsection{Exercises}

\begin{question}{2.66}
    For each matrix $( a ) - ( d ) ,$ find its rank and bases for its column and
    row spaces.

\begin{pycode}
problems = [
    [[1, 2, 0, -2], [2, 3, 2, 3], [2, 10, 4, 10]],
    [[-1, 4, -2], [4, 4, 2], [3, 0, -3]],
    [[2, 1, 1], [1, 3, 2], [5, 0, 1]],
    [[1, 2, 2], [2, 4, 4], [3, 6, 6], [-2, -4, -4]]
]

sol = "\\begin{enumerate}[label=(\\alph*)]\n"
for problem in problems:
    sol += "\\item $$ " + utils.print(problem) + " $$ \n"
    sol += "\\textit{Answer: } \n"

    rref, _ = utils.algorithms.rref(problem)

    sol += "$$ "
    sol += utils.print(problem)
    sol += "\\rightarrow"
    sol += utils.print(rref)
    sol += " $$ \n"

    pivots = utils.algorithms.get_pivots(rref)
    sol += "Rank: {} \\\\".format(len(pivots))
    sol += "Row space basis: "
    basis = []
    for i, j in pivots:
        basis.append(rref[i])
    sol += "$ " + utils.print.print_list_of_arrays(basis) + " $\n"
    sol += "\\\\"

    sol += "Column space basis: "
    basis = []
    for i, j in pivots:
        basis.append(np.array(problem)[:, j])
    sol += "$ " + utils.print.print_list_of_arrays(basis) + " $\n"

sol += "\\end{enumerate}"
\end{pycode}
    
    \py{sol}
\end{question}

\begin{question}{2.70}
    Let $\mathcal { W }$ be the span of the following vectors.

\begin{pycode}
vecs = [
    [2, 3, 1, 2],
    [5, 2, 1, 2],
    [1, -4, -1, -2],
    [11, 0, 1, 2]
]

A = np.array(vecs)
rref, _ = utils.algorithms.rref(A)
pivots = utils.algorithms.get_pivots(rref)
basis = []
for i, j in pivots:
    basis.append(rref[i])
\end{pycode}

$$\py{utils.print.print_list_of_arrays(vecs)}$$

    \begin{enumerate}[label=(\alph*)]
        \item Use Theorem 2.12 on page 134 to find a basis for $\mathcal { W }$ \\
        \ans $\py{utils.print.print_list_of_arrays(basis)}$

        \item Express each the given vectors as a linear combination of the 
        basis elements. \\
\begin{pycode}
sol = ""
for vec in vecs:
    _, lincomb_string = utils.algorithms.as_linear_combination(basis, vec)
    sol += "$"
    sol += lincomb_string
    sol += "$ \\\\"
\end{pycode}
        \ans \py{sol}

        \item Use Theorem 2.3 on page 104 to find a basis for $\mathcal { W }$. \\
\begin{pycode}
rref, _ = utils.algorithms.rref(A.T)
pivots = utils.algorithms.get_pivots(rref)
basis_cols = []
for i, j in pivots:
    basis_cols.append(A.T[:, j])
\end{pycode}
        \ans $\py{utils.print.print_list_of_arrays(basis_cols)}$
    \end{enumerate}
\end{question}

\begin{question}{2.80}
    An $m \times n$ matrix $A$ has a $d$ -dimensional nullspace. What is the 
    dimension of the nullspace of $A ^ { t } ?$

    \ans According to Rank-Nullity Theorem, $m - (n - d)$
\end{question}

\section{Section 3.1}
\subsection{True/False}

\begin{question}{3.1}
    A linear transformation of $\mathbb { R } ^ { 2 }$ into $\mathbb { R } ^ { 2 }$
    that transforms $[ 1,2 ] ^ { t }$ to $[ 7,3 ] ^ { t }$ and $[ 3,4 ] ^ { t }$
    to $[ - 1,1 ] ^ { t }$ will also transform $[ 5,8 ] ^ { t }$ to $[ 13,7 ] ^ { t }$

\begin{pycode}
A = [
    [1, 2, 0, 0, 7],
    [0, 0, 1, 2, 3],
    [3, 4, 0, 0, -1],
    [0, 0, 3, 4, 1]
]
rref, _ = utils.algorithms.rref(A)

trans = [
    [-15, 11],
    [-5, 4]
]
\end{pycode}

    \ans True. We can find the transformation by solving this augmented matrix
    $$\printobj{A} \rightarrow \printobj{rref}$$
    So the transformation is $\printobj{trans}$, we can verify that
    $$\printobj{trans} \printobj{[1, 2]} = \printobj{np.matmul(trans, [1, 2])}$$
    $$\printobj{trans} \printobj{[3, 4]} = \printobj{np.matmul(trans, [3, 4])}$$
    $$\printobj{trans} \printobj{[5, 8]} = \printobj{np.matmul(trans, [5, 8])}$$
\end{question}

\begin{question}{3.4}
    It is impossible for a linear transformation from $\mathbb { R } ^ { 2 }$ 
    into $\mathbb { R } ^ { 2 }$ to transform a parallelogram onto a line segment.

    \ans False
\end{question}

\begin{question}{3.5}
    All transformations of $\mathbb { R } ^ { 2 }$ into $\mathbb { R } ^ { 2 }$ 
    transform line segments onto line segments.

    \ans False, it is possible to transform (example $\printobj{np.zeros((2, 2)).astype(int)}$)
    line segment to a point.
\end{question}

\subsection{Exercises}

\begin{question}{3.9}
    Describe geometrically the effect of the transformation of 
    $\mathbb { R } ^ { 3 }$ into $\mathbb { R } ^ { 3 }$ defined
    by multiplication by the following matrices.

    \begin{enumerate}[label=(\alph*)]
        \item $$R _ { \psi } ^ { x } = \left[ \begin{array} { c c c } { 1 } & { 0 } & { 0 } \\ { 0 } & { \cos \psi } & { - \sin \psi } \\ { 0 } & { \sin \psi } & { \cos \psi } \end{array} \right]$$
        \ans Rotation counterclockwise around $x$ axis for $\psi$

        \item $$
        R _ { \psi } ^ { y } = \left[ \begin{array} { c c c } { \cos \psi } & { 0 } & { - \sin \psi } \\ { 0 } & { 1 } & { 0 } \\ { \sin \psi } & { 0 } & { \cos \psi } \end{array} \right]
        $$
        \ans Rotation counterclockwise around $y$ axis for $\psi$

        \item $$
        R _ { \psi } ^ { z } = \left[ \begin{array} { c c c } { \cos \psi } & { - \sin \psi } & { 0 } \\ { \sin \psi } & { \cos \psi } & { 0 } \\ { 0 } & { 0 } & { 1 } \end{array} \right]
        $$
        \ans Rotation counterclockwise around $z$ axis for $\psi$
    \end{enumerate}
\end{question}

\begin{question}{3.10}
    Consider the points
    $$
    \begin{array} { l l } { X _ { 1 } = [ 1,1 ] ^ { t } , } & { X _ { 2 } = [ 2,2 ] ^ { t } } \\ { Y _ { 1 } = [ 4,5 ] ^ { t } , } & { Y _ { 2 } = [ 5,6 ] ^ { t } } \end{array}
    $$

    Is it possible to find a $2 \times 2$ matrix $A$ for which multiplication by $A$ 
    transforms $X _ { 1 }$ into $Y _ { 1 }$ and $X _ { 2 }$ into $Y _ { 2 } ?$ 

    \ans No
    \begin{gather*}
        X_2 = 2X_1
        Y_2 = AX_2 = A(2X_1) = 2AX_1 = 2Y_1
    \end{gather*}
\end{question}

\begin{question}{3.17}
    What matrix describes rotation in $\mathbb { R } ^ { 2 }$ clockwise by $\theta$ radians?
    
    \ans 
    $$
    \begin{bmatrix}
        \cos \theta & \sin \theta \\
        -\sin \theta & \cos \theta
    \end{bmatrix}
    $$
\end{question}

\begin{question}{3.21}
    Let $\mathcal { V }$ and $\mathcal { W }$ be vector spaces. Let 
    $T : \mathcal { V } \rightarrow \mathcal { W }$ be a linear transformation.
    Prove that $T ( 0 ) = 0 .$ (Note: You should not assume that $T$ is a matrix 
    transformation. Instead, think about the property that in any vector space $0 , X = 0 . )$

    \ans $T(0 + 0) = T(0) + T(0)$
\end{question}

\begin{question}{3.24}
    Let $T : \mathcal { V } \rightarrow \mathcal { W }$ be a linear transformation between two vector spaces. We
define the image $T ( \mathcal { V } )$ by $T ( \mathcal { V } ) = \{ T ( X ) | X \in \mathcal { V } \} .$ The image is the set
of $Y \in \mathcal { W }$ such that the equation $T ( X ) = Y$ is solvable for $X \in \mathcal { V }$ . Show that
the image of $T$ is a subspace of $\mathcal { V }$ .

    \ans Let $T(X_1) = Y_1, T(X_2) = Y_2$ to be in $T ( \mathcal { V } )$. We can
    show that $T(aY_1 + bY_2) = aT(Y_1) + bT(Y_2)$, where $a$ and $b$ are scalars, meaning
    that image is closed under linear combinations.
\end{question}

\section{Section 3.2}
\subsection{True/False}

\begin{question}{3.14}
    Let $A = R _ { \pi / 2 }$ be the matrix that describes rotation by $\pi / 2$ radians [formula $( 3.1 )$
    on page $150 ] .$ Then $A ^ { 4 } = I ,$ where $I$ is the $2 \times 2$ identity matrix.

    \ans True, since application of the rotation by $\pi / 2$ radians will result
    in the full circle of rotation.
\end{question}

\begin{question}{3.15}
    Assume that $A$ and $B$ are matrices such that $A B$ is defined and $B$ has a column
    that has all its entries equal to zero. Then one of the columns of $A B$ also has all
    its entries equal to zero.

    \ans True
\end{question}

\begin{question}{3.18}
    Assume that $A$ and $B$ are matrices such that $A B$ is defined and the columns of $B$
    are linearly dependent. Then the columns of $A B$ are also linearly dependent.

    \ans True
\end{question}

\subsection{Exercises}

\begin{question}{3.34}
    Define a transformation $T : \mathbb { R } ^ { 2 } \rightarrow \mathbb { R } ^ { 2 }$ by 
    the following rule: $T ( X )$ is the result
    of first rotating $X$ counterclockwise by $\pi / 6$ radians and then multiplying by
    $$
    A = \left[ \begin{array} { l l } { 2 } & { 0 } \\ { 0 } & { 3 } \end{array} \right]
    $$

\begin{pycode}
A = [
    [2, 0],
    [0, 3]
]
R = [
    [sympy.cos(sympy.pi / 6), -sympy.sin(sympy.pi / 6)],
    [sympy.sin(sympy.pi / 6), sympy.cos(sympy.pi / 6)]
]

A = sympy.Matrix(A)
R = sympy.Matrix(R)
B_sympy = sympy.MatMul(A, R).doit()
B = np.array(B_sympy)
\end{pycode}


    $$
    B = AR_{\pi/6} = \printobj{A} \printobj{R} = \printobj{B}
    $$

    \begin{enumerate}[label=(\alph*)]
        \item What is the image of the circle $x ^ { 2 } + y ^ { 2 } = 1$ under $T ?$ \\
        \ans 
\begin{pycode}
C = np.c_[B, ['u', 'v']]
rref, _ = utils.algorithms.rref(C)
img = rref[0, 2] ** 2 + rref[1, 2] ** 2
\end{pycode}

        $$\printobj{C} \rightarrow \printobj{rref}$$

        Our image of the circle:
        $$\printobj{img} = 1$$

        \item What is the image of the unit square under $T ?$ \\
        \ans 
\begin{pycode}
problems = [
    [0, 0],
    [1, 0],
    [0, 1],
    [1, 1]
]

solutions = []
for problem in problems:
    solutions.append(np.array(sympy.MatMul(B_sympy, sympy.Matrix(problem)).doit()))

\end{pycode}

        $$
        \printobj{utils.print.print_list_of_arrays(problems)}
        \rightarrow
        \printobj{utils.print.print_list_of_arrays(solutions)}
        $$

        \item Find a matrix $B$ such that $T ( X ) = B X$ for all $X \in \mathbb { R } ^ { 2 }$ \\
        \ans 

        $$B = \printobj{B}$$
    \end{enumerate}
\end{question}

\begin{question}{3.49}
    Find a $3 \times 3$ matrix $A$ such that $A ^ { 3 } = 0$ but $A ^ { 2 } \neq 0 .$

\begin{pycode}
A = np.array([
    [0, 0, 0],
    [1, 0, 0],
    [0, 1, 0]
])
\end{pycode}

    \ans 
    \begin{gather*}
        A = \printobj{A}, 
        A^2 = \printobj{np.linalg.matrix_power(A, 2)},
        A^3 = \printobj{np.linalg.matrix_power(A, 3)}
    \end{gather*}

\end{question}

\end{document}