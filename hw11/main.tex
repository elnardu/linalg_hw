\documentclass[letterpaper]{article}

%% Sets page size and margins
\usepackage[letterpaper,top=3cm,bottom=2cm,left=3cm,right=3cm,marginparwidth=1.75cm]{geometry}
 
%% Useful packages
\usepackage{amsmath}
\usepackage{amsfonts}
\usepackage{mathtools}
\usepackage{graphicx}
\usepackage{enumitem}
\usepackage[colorlinks=true, allcolors=blue]{hyperref}
\usepackage{pythontex}
\usepackage{lastpage}
\usepackage{fancyhdr}

\pagestyle{fancy}
\graphicspath{ {images/} }
\newcommand{\true}{$T$}
\newcommand{\false}{$F$}
\newcommand{\ans}{\textit{Answer: }}
\newcommand{\prf}{\textbf{Proof:}}
\newenvironment{question}[2][Question]{\begin{trivlist}
\item[\hskip \labelsep {\bfseries #1}\hskip \labelsep {\bfseries #2.}]}{\end{trivlist}}

\allowdisplaybreaks

% Fancy matrix
\makeatletter
\renewcommand*\env@matrix[1][*\c@MaxMatrixCols c]{%
  \hskip -\arraycolsep
  \let\@ifnextchar\new@ifnextchar
  \array{#1}}
\makeatother

% My scripts setup
\begin{pycode}
import sympy
import numpy as np

import sys
sys.path.append('..')

from utils import Utils
utils = Utils(pytex)
sympify = np.vectorize(sympy.sympify)

# display det
ddet = utils.print.print_matrix_det


def proj(basis, vector):
  solution = []
  solution_vec = 0
  for b in basis:
    solution.append(f"\\frac{{{utils.print(vector)}\\cdot {utils.print(b)}}}{{{utils.print(b)} \cdot {utils.print(b)}}}{utils.print(b)}")
    solution_vec += (np.dot(vector, b) / np.dot(b, b)) * b

  solution = "+".join(solution)
  solution += f"= {utils.print(solution_vec)}"
  return solution, solution_vec


def gram_schi(basis):
  solution = "\\begin{itemize}\n"
  p1 = basis[0]
  new_basis = [p1]

  solution += f"\\item $P_1 = X_1 = {utils.print(p1)}$\n"
  for i, b in enumerate(basis[1:]):
    new_p = b - proj(new_basis, b)[1]

    solution += f"\\item $Y_{i+2} = {proj(new_basis, b)[0]}$, "
    solution += f"$P_{i+2} = X_{i+2} - Y_{i+2} = {utils.print(new_p)}$"
    solution += "\n"

    if (new_p == 0).all():
      continue

    new_basis.append(new_p)

  solution += "\\end{itemize}\n\n"
  return solution, new_basis

\end{pycode}

\newcommand{\printobj}[1]{\py{utils.print.print_object(#1)}}

\title{MA 351, HW 11} 
\lhead{MA 351, HW 11}

\author{Elnard Utiushev}
\rhead{Elnard Utiushev}
\cfoot{\thepage\ of \pageref{LastPage}}

\begin{document}

\maketitle

Section 6.4: True/False: 6.15, 6.16, 6.17, 6.18, 6.19, 6.20, 6.22 \\
Exercises: 6.70, 6.71, 6.74, 6.75, 6.76, 6.81, 6.82, 6.83, 6.84

\section{Section 6.4}
\subsection{True/False}

\begin{question}{6.15}
  There exist $4 \times 4$ orthogonal matrices with rank 3.

  \ans It is not possible since $\det(AA^t) = \det(I) = 1$, so $A$ cannot 
  have a rank less than 4.
\end{question}

\begin{question}{6.16}
  If $A$ is an $n \times n$ orthogonal matrix and $B$ is an $n \times 1$ matrix, then the equation
  $A X=B$ has a unique solution.

  \ans Since $A$ is an orthogonal matrix, $\det(A) \neq 0$, therefore equation
  $A X=B$ has a unique solution.
\end{question}

\begin{question}{6.17}
  Multiplication by an orthogonal matrix transforms congruent triangles into
  congruent triangles.

  \ans True, since an orthogonal matrix transform does not change the length 
  of triangles' sides
\end{question}

\begin{question}{6.18}
  If multiplication by an orthogonal matrix transforms a given parallelogram into
  a square, then the parallelogram was a square to begin with.

  \ans True, since orthogonal matrix transformation preserves both angles and lengths
\end{question}

\begin{question}{6.19}
  The following matrix is orthogonal:

  $$
  A = \left[ \begin{array}{rrr}{1} & {0} & {0} \\ {0} & {1} & {-1} \\ {0} & {1} & {1}\end{array}\right]
  $$

  \begin{pycode}
problem = [[1, 0, 0], [0, 1, -1], [0, 1, 1]]
problem = np.array(problem)
  \end{pycode}

  \ans False

  $$A A^t = \printobj{problem @ problem.T} \neq I$$

\end{question}

\begin{question}{6.20}
  Suppose that $A=\left[A_{1}, A_{2}, A_{3}\right]$ is an orthogonal matrix where $A_{i}$ are the columns
  of $A .$ Then $\left|A_{1}+A_{2}+A_3 \right|=\sqrt{3} .$

  \ans True, since we can form an orthonormal basis from the columns of an orthogonal matrix
\end{question}

\begin{question}{6.22}
  The following matrix is orthogonal:

  $$
  A=\left[ \begin{array}{lll}{1} & {0} & {0} \\ {0} & {1} & {0} \\ {0} & {0} & {1}\end{array}\right]-\frac{2}{3} \left[ \begin{array}{l}{1} \\ {1} \\ {1}\end{array}\right][1,1,1]
  $$

  \begin{pycode}
problem = sympify(np.eye(3).astype(int)) 
problem = problem - 2 * sympify(np.ones((3, 3)).astype(int)) / 3
  \end{pycode}

  \ans True

  $$A = \printobj{problem}$$

  $$AA^t = \printobj{problem} \printobj{problem.T} 
  = I$$
\end{question}

\subsection{Exercises}

\begin{question}{6.70}
  Change just one column of each of the following matrices to make them
  orthogonal:

  \begin{enumerate}[label=(\alph*)]
    \item 
    $$
    \frac{1}{5} \left[ \begin{array}{rrr}{3} & {4} & {3} \\ {-4} & {3} & {0} \\ {0} & {0} & {4}\end{array}\right]
    $$

    \ans
    $$
    \frac{1}{5} \left[ \begin{array}{rrr}{3} & {4} & {0} \\ {-4} & {3} & {0} \\ {0} & {0} & {5}\end{array}\right]
    $$
    
%     \begin{pycode}
% basis = [
%   [3, -4, 0],
%   [4, 3, 0],
%   [3, 0, 4]
% ]

% basis = list(map(sympify, basis))

% sol, new_basis = gram_schi(basis)

% new_matrix = np.column_stack(new_basis)
% new_matrix = [
%   [3, 4, 0],
%   [-4, 3, 0],
%   [0, 0, 1]
% ]
% new_matrix = sympify(new_matrix) / 5
%     \end{pycode}
    
    % \py{sol}

    % $$\printobj{new_matrix}$$

    % $$\printobj{sympy.MatMul(sympy.Matrix(new_matrix), sympy.Matrix(new_matrix)).doit()}$$

    \item 
    $$
    \frac{1}{25} \left[ \begin{array}{rrr}{16} & {-12} & {-15} \\ {-12} & {9} & {-20} \\ {-15} & {-20} & {1}\end{array}\right]
    $$

    \ans 
    $$
    \frac{1}{25} \left[ \begin{array}{rrr}{16} & {-12} & {-15} \\ {-12} & {9} & {-20} \\ {-15} & {-20} & {0}\end{array}\right]
    $$


    \item 
    $$
    \frac{1}{9} \left[ \begin{array}{rrr}{8} & {1} & {-4} \\ {-4} & {1} & {-7} \\ {1} & {1} & {4}\end{array}\right]
    $$

    \ans 
    $$
    \frac{1}{9} \left[ \begin{array}{rrr}{8} & {1} & {-4} \\ {-4} & {4} & {-7} \\ {1} & {8} & {4}\end{array}\right]
    $$

  \end{enumerate}

\end{question}

\begin{question}{6.71}
  In each part, find numbers $C$ or $a, b, c,$ and $d$ such that the given matrix $A$ is
  orthogonal:

  \begin{enumerate}
    \item 
    $$
    A=C \left[ \begin{array}{ccc}{9} & {-12} & {-8} \\ {-12} & {-1} & {-12} \\ {-8} & {-12} & {9}\end{array}\right]
    $$

    \ans $C = \pm \frac{1}{17}$

    \item
    $$
    A=\frac{1}{2} \left[ \begin{array}{rrrr}{1} & {1} & {\sqrt{2}} & {a} \\ {1} & {-1} & {0} & {b} \\ {1} & {1} & {-\sqrt{2}} & {c} \\ {1} & {-1} & {0} & {d}\end{array}\right]
    $$

    \ans $a = 0, b = \pm \sqrt{2}, c = 0, d = \mp \sqrt{2}$ 
  \end{enumerate}  

\end{question}

\begin{question}{6.74}
  Let $A$ and $B$ be $n \times n$ orthogonal matrices. Prove that $A B$ is orthogonal by
  showing that for all $X \in \mathbb{R}^{n},|(A B) X|=|X|$

  \ans $|(A B) X| = |A (B X)| = |B X| = |X|$
\end{question}

\begin{question}{6.75}
  Redo Exercise 6.74 by showing that $(A B)^{t}(A B)=I$

  \ans $(A B)^{t}(A B) = B^tA^tAB = B^tIB = B^tB = I$
\end{question}

\begin{question}{6.76}
  Prove that the inverse of an orthogonal matrix is orthogonal.

  \ans Let $A$ to be an orthogonal matrix. $A^t$ is an orthogonal matrix according 
  to $A^tA = I$. Since $A^{-1} = A^t$, $A^{-1}$ is an orthogonal matrix
\end{question}

\begin{question}{6.81}
  Is it possible to find a $3 \times 2$ matrix with orthonormal rows? Explain.

  \ans It is not possible, since we are going to have 3 vectors, with size $2 \times 1$
  spanning $\mathbb{R}^{2}$, so one of them will be a linear combination of others.
\end{question}

\begin{question}{6.82}
  Give an example of a $3 \times 2$ matrix $A$ with all entries nonzero that has
  orthonormal columns. Compute $A A^{t}$ and $A^{t} A .$ Which is the identity? Prove that the
  similar product equals $I$ for any $A$ that has orthonormal columns.

  \ans 

  \begin{pycode}
A = [
  [2, -1],
  [-1, 2],
  [2, 2]
]

A = sympify(A)
A = A / 3
A = sympy.Matrix(A)
  \end{pycode}

  $$A = \frac{1}{3}\printobj{[
    [2, -1],
    [-1, 2],
    [2, 2]
  ]}$$

  $$AA^t = \printobj{sympy.MatMul(A, A.T).doit()}$$
  $$A^tA = \printobj{sympy.MatMul(A.T, A).doit()}$$

  $A^tA$ results in an identity matrix.

  \begin{pycode}
A = [
  ['a', 'b'],
  ['c', 'd'],
  ['e', 'f']
]

A = sympify(A)
A = sympy.Matrix(A)
  \end{pycode}

  $$A = \printobj{A}$$
  $$A^tA = \printobj{sympy.MatMul(A.T, A).doit()}$$

  Since the columns are orthogonal, $ab+cd+ef = 0$. Since the columns are 
  orthonormal, $a^2 + c^2 + e^2 = b^2 + d^2 + f^2 = 1$
\end{question}

\begin{question}{6.83}
  In parts $(\mathrm{a})-(\mathrm{c}),$ find the matrix $M$ such that multiplication by $M$ describes
  reflection about the given line, plane, or hyperplane. (Use formula $(6.62)$ on
  page $362 .$ .

  \begin{pycode}
def householder_matrix(A):
  solution = "I - \\frac{2}{|P|^2}PP^t ="
  solution += utils.print(np.eye(A.shape[0]).astype(int)) + " "
  solution += f"- \\frac{{2}}{{{utils.print(A.dot(A))}}} {utils.print(sympy.MatMul(A, A.T).doit())} ="
  matrix_solution = sympify(np.eye(A.shape[0]).astype(int)) - (2 / A.dot(A)) * sympy.MatMul(A, A.T)
  matrix_solution = matrix_solution.doit()
  solution += utils.print(matrix_solution)
  return solution 
  \end{pycode}

  \begin{enumerate}
    \item $2 x-5 y=0$
    \begin{pycode}
A = sympy.Matrix([2, -5])
    \end{pycode}

    \ans 
    $$P = \printobj{A}$$
    $$M_P = \printobj{householder_matrix(A)}$$

    \item $x+y-3 z=0$
    \begin{pycode}
A = sympy.Matrix([1, 1, -3])
    \end{pycode}

    \ans 
    $$P = \printobj{A}$$
    $$M_P = \printobj{householder_matrix(A)}$$

    \item $2 x+y-z-3 w=0$
    \begin{pycode}
A = sympy.Matrix([2, 1, 1, -3])
    \end{pycode}

    \ans 
    $$P = \printobj{A}$$
    $$M_P = \printobj{householder_matrix(A)}$$
  \end{enumerate}

\end{question}

\begin{question}{6.84}
  For each given vector $X \in \mathbb{R}^{n},$ find a scalar $k,$ a vector $P \in \mathbb{R}^{n},$ and a 
  Householder matrix $M_{P}[$ formula $(6.62)$ on page 362$]$ such that $M_{P} X=k I_{1}$ where $I_{1}$
  is the first standard basis element of $\mathbb{R}^{n}[$ formula $(6.62)$ on page 362$]$ .

  \begin{enumerate}
    \item $X=[3,4]^{t}$
    \begin{pycode}
X = sympy.Matrix([3, 4])
    \end{pycode}

    \ans 
    $$k = |X| = 5$$
    $$P = X - |X|I_1 = \printobj{X - 5 * sympy.Matrix([1, 0])}$$
    $$M_P = \printobj{householder_matrix(X - 5 * sympy.Matrix([1, 0]))}$$

    \item $X=[1,1]^{t}$
    \begin{pycode}
X = sympy.Matrix([1, 1])
    \end{pycode}

    \ans 
    $$k = |X| = 1$$
    $$P = X - |X|I_1 = \printobj{X - 1 * sympy.Matrix([1, 0])}$$
    $$M_P = \printobj{householder_matrix(X - 1 * sympy.Matrix([1, 0]))}$$

    \item $X=[1,2,2]^{t}$
    \begin{pycode}
X = sympy.Matrix([1, 2, 2])
    \end{pycode}

    \ans 
    $$k = |X| = 3$$
    $$P = X - |X|I_1 = \printobj{X - 3 * sympy.Matrix([1, 0, 0])}$$
    $$M_P = \printobj{householder_matrix(X - 3 * sympy.Matrix([1, 0, 0]))}$$
  \end{enumerate}
\end{question}

\end{document}