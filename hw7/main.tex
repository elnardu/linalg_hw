\documentclass[letterpaper]{article}

%% Sets page size and margins
\usepackage[letterpaper,top=3cm,bottom=2cm,left=3cm,right=3cm,marginparwidth=1.75cm]{geometry}
 
%% Useful packages
\usepackage{amsmath}
\usepackage{amsfonts}
\usepackage{mathtools}
\usepackage{graphicx}
\usepackage{enumitem}
\usepackage[colorlinks=true, allcolors=blue]{hyperref}
\usepackage{pythontex}
\usepackage{lastpage}
\usepackage{fancyhdr}

\pagestyle{fancy}
\graphicspath{ {images/} }
\newcommand{\true}{$T$}
\newcommand{\false}{$F$}
\newcommand{\ans}{\textit{Answer: }}
\newcommand{\prf}{\textbf{Proof:}}
\newenvironment{question}[2][Question]{\begin{trivlist}
\item[\hskip \labelsep {\bfseries #1}\hskip \labelsep {\bfseries #2.}]}{\end{trivlist}}

\allowdisplaybreaks

% Fancy matrix
\makeatletter
\renewcommand*\env@matrix[1][*\c@MaxMatrixCols c]{%
  \hskip -\arraycolsep
  \let\@ifnextchar\new@ifnextchar
  \array{#1}}
\makeatother

% My scripts setup
\begin{pycode}
import sympy
import numpy as np

import sys
sys.path.append('..')

from utils import Utils
utils = Utils(pytex)
sympify = np.vectorize(sympy.sympify)

# display det
ddet = utils.print.print_matrix_det
\end{pycode}

\newcommand{\printobj}[1]{\py{utils.print.print_object(#1)}}

\title{MA 351, HW 7} 
\lhead{MA 351, HW 7}

\author{Elnard Utiushev}
\rhead{Elnard Utiushev}
\cfoot{\thepage\ of \pageref{LastPage}}

\begin{document}

\maketitle

Section 4.2.1: Exercises: 4.31, 4.33 \\ 
Section 4.3: Exercises: 4.35, 4.36, 4.38, 4.43 \\
Section 5.1: True/False 5.1, 5.4, 5.5, 5.6, 5.11; Exercises: 5.5, 5.11, 5.12, 5.13

\section{Section 4.2.1}
\subsection{Exercises}

\begin{question}{4.31}
  Use volume to explain why it is expected that the determinant of a \(3 \times 3\) matrix
  with linearly dependent columns has determinant zero.

  \ans If columns are linearly independent, the vectors that \(3 \times 3\) matrix 
  describes will be in the same plane, which makes volume equal to zero.
  
\end{question}

\begin{question}{4.33}
  Use volume to explain why it is expected that for \(3 \times 3\) matrices \(B\) and \(C \)
\(|\operatorname{det}(B C)|=|\operatorname{det} B||\operatorname{det} C| .[\)Hint\(:\) Use Theorem \(4.13 .] \)
  
  \ans Lets take an object with the volume of 1. By applying $B$ on that object, we 
  will get that the volume of that object has changed to $det(B)$. Similarly, with $C$.
  Applying both of those, the resulting volume will be $det(B) * det(C)$.
\end{question}

\section{Section 4.3}
\subsection{Exercises}

\begin{question}{4.35}
  Use Cramer's rule to express the value of \(y\) in system \((a)\) and the value of \(z\) in
  system \((b)\) as a ratio of two determinants. Do not evaluate the determinants.

  \begin{enumerate}[label=\textbf{(\alph*)}]
    \item 
    $$
    \begin{aligned} 2 x+y+3 z+w &=4 \\ x+4 y+2 z-3 w &=-1 \\-x+y+z+w &=0 \\ 4 x-y+z+2 w &=0 \end{aligned}
    $$

    \ans 

    $$
    y = \frac{
      \py{ddet([
      [2, 4, 3, 1],
      [1, -1, 2, -3],
      [-1, 0, 1, 1],
      [4, 0, 1, 2]
      ])}
    }{
      \py{ddet([
      [2, 1, 3, 1],
      [1, 4, 2, -3],
      [-1, 1, 1, 1],
      [4, -1, 1, 2]
      ])}
    }
    $$

    \item
    $$
    \begin{array}{c}{x+3 y+z=1} \\ {3 x+4 y+5 z=7} \\ {2 x+5 y+7 z=2}\end{array}
    $$

    \ans 

    $$
    z = \frac{
      \py{ddet([
        [1, 3, 1],
        [3, 4, 7],
        [2, 5, 2]
      ])}
    }{
      \py{ddet([
        [1, 3, 1],
        [3, 4, 5],
        [2, 5, 7]
      ])}
    }
    $$
  \end{enumerate}
  
\end{question}

\begin{question}{4.36}
  Use Cramer's rule to solve the following system for z in terms of $p_{1}, p_{2},$ and
  $p_{3} .($ Note that we have not asked for $x$ or $y .) $

  $$
  \begin{array}{c}{x+2 y-3 z=p_{1}} \\ {3 x+y-z=p_{2}} \\ {2 x+3 y+5 z=p_{3}}\end{array}
  $$

  \begin{pycode}
problem0 = sympify([
  [1, 2, -3],
  [3, 1, -1],
  [2, 3, 5]
])

problem1 = sympify([
  [1, 2, 'p_1'],
  [3, 1, 'p_2'],
  [2, 3, 'p_3']
])

det_0, sol_0 = utils.algorithms.det(problem0)
det_1, sol_1 = utils.algorithms.det(problem1)
  \end{pycode}

  \ans 

  $$\py{utils.print.print_matrix_det(problem1)} = \py{sol_1} = \py{det_1}$$

  $$\py{utils.print.print_matrix_det(problem0)} = \py{sol_0} = \py{det_0}$$

  $$\frac{
    \py{utils.print.print_matrix_det(problem1)}
    }{
    \py{utils.print.print_matrix_det(problem0)}
    } = \frac{\py{det_1}}{\py{det_0}}$$
\end{question}

\begin{question}{4.38}
  Use Theorem 4.15 to find the $(1,2)$ entry for the inverse of each invertible
  matrix in Exercise 4.12 on page 258

  \begin{pycode}
problems = [
  [[1, 0, 1], [2, 1, 1], [3, 2, 1]],
  [[-3, 2, 2], [1, 4, 1], [7, 6, -2]],
  [[2, 0, 2, 0], [1, 1, 1, 1], [0, 0, 3, 2], [1, 0, 0, 5]],
  [[3, 1, 3, 0], [3, 1, 3, 1], [0, 0, 2, 1], [6, 3, 4, 5]],
  [[1, 2, 3, 4], [5, 6, 7, 8], [9, 10, 11, 12], [13, 14, 15, 16]],
  [[2, 3, 2, 0], [9, 0, 1, 1], [1, 0, 1, 4], [13, 10, 0, 9]]
]

problems = map(np.array, problems)

solution = "\\begin{enumerate}[label=\\textbf{(\\alph*)}]\n"
for problem in problems:
  det, _ = utils.algorithms.det(problem)
  solution += "\\item\n$$"

  if det == 0:
    solution += ddet(problem)
    solution += "= 0"
    solution += ",\\quad\\text{Matrix is not invertible}"
  else:
    solution += utils.print(problem)
    solution += ",\\quad"
    solution += "(A^{-1})_{12} = "

    inter = problem[:, 1:]
    inter = np.delete(inter, [1], 0)

    solution += "\\frac{-" + ddet(inter) + "}{" + ddet(problem) + "} = "
    
    inter_det, _ = utils.algorithms.det(inter)

    solution += "\\frac{" + str(-1 * inter_det) + "}{" + str(det) + "}"

  solution += "$$\n"

solution += "\\end{enumerate}\n"
  \end{pycode}

  \ans 

  \py{solution}
  
\end{question}

\begin{question}{4.43}
  Let $A$ be an $n \times n$ matrix that has only integers as entries. State a necessary and
  sufficient condition on the determinant of such a matrix that guarantees that
  the inverse has only integers as entries. Prove your condition. [Hint: Consider
  the property $A A^{-1}=I . ] $

  \ans 
  \begin{gather*}
    \det(AA^{-1}) = \det(I) = 1 \\
    \det(AA^{-1}) = \det(A)\det(A^{-1}) \\
    \det(A)\det(A^{-1}) = 1
  \end{gather*}

  Therefore, the inverse has only integers as entries if and only if $\det(A) = \pm 1$ 

\end{question}

\section{Section 5.1}
\subsection{True/False}

\begin{question}{5.1}
  If $A$ is an $n \times n$ matrix that has zero for an eigenvalue, then $A$ cannot be invertible.

  \ans True, 
  \begin{gather*}
    (A - 0I)X = 0 \\
    AX = 0 \\
    A^{-1}AX = A^{-1} 0 \\
    X = 0
  \end{gather*}
  
\end{question}

\begin{question}{5.4}
  The sum of two eigenvectors is an eigenvector.

  \ans False, for example, $\py{utils.print([[4, 0], [0, 5]])}$. Eigenvectors for that matrix 
  are $\py{utils.print([1, 0])}, \py{utils.print([0, 1])}$. Their sum is $\py{utils.print([1, 1])}$ is 
  not an eigenvector.
  
\end{question}

\begin{question}{5.5}
  If $X$ is an eigenvector for $A$ with eigenvalue $3,$ then 2$X$ is an eigenvector for $A $
  with eigenvalue $6 . $

  \ans False, $2X$ will still have the eigenvalue of 3. 
  
\end{question}

\begin{question}{5.6}
  If $X$ is an eigenvector for an $n \times n$ matrix $A,$ then $X$ is also an eigenvector for
  2$A . $

  \ans True, 
  \begin{gather*}
    AX = \lambda X \\
    2AX = 2\lambda X \\
    (2A)X = (2\lambda) X
  \end{gather*}
  
\end{question}

\begin{question}{5.11}
  There is a $3 \times 3$ matrix with eigenvalues $1,2,3,$ and 4

  \ans False, since the cubic polynomial cannot have 4 roots.
  
\end{question}

\subsection{Exercises}

\begin{question}{5.5}
  For the following matrices, find all eigenvalues and a basis for each eigenspace.
  State whether or not the given matrix is diagonalizable over $\mathbb{R}$ .

  \begin{enumerate}[label=\textbf{(\alph*)}]
    \begin{pycode}
problem = sympify([[-18, 30], [-10, 17]])
lambda_ = sympy.Symbol("\\lambda")
inter = problem - lambda_ * sympify(np.eye(problem.shape[0]).astype(int))
det, _ = utils.algorithms.det(inter)
det = sympy.factor(det)

roots = sympy.solvers.solve(det, lambda_)

solution = "\\begin{enumerate}[label=\\arabic*.]\n"
for root in roots:
  solution += "\\item$\\lambda = {}$\n".format(root)
  eigen_prep = problem - root * np.eye(problem.shape[0]).astype(int)
  rref, _ = utils.algorithms.rref(np.c_[eigen_prep, np.zeros((problem.shape[0], 1)).astype(int)])
  matrix_sol, _ = utils.algorithms.show_solutions(rref)

  solution += "$$A - ({}) I = ".format(root)
  solution += utils.print(eigen_prep)
  solution += "$$\n"
  solution += "$$" + utils.print(np.c_[eigen_prep, np.zeros((problem.shape[0], 1)).astype(int)])
  solution += "\\xrightarrow{rref}" + utils.print(rref)
  solution += "$$\n"

  solution += "Eigenvectors: $"

  sep = ""
  for vec in matrix_sol['param_vecs']:
    if not (vec == 0).all():
      solution += sep + utils.print(vec)
      sep = ",\\quad"

  solution += "$\n"

solution += "\\end{enumerate}"

    \end{pycode}

    \item $$\py{utils.print(problem)}$$
    \ans

    $$\det(A - \lambda I) = \py{utils.print.print_matrix_det(inter)} = \printobj{det} = 0$$
    $\lambda = \py{", ".join(map(str, roots))}$

    \py{solution}

    Matrix is diagonalizable
% ----------------------
    \begin{pycode}
problem = sympify([[10, -17], [6, -10]])
lambda_ = sympy.Symbol("\\lambda")
inter = problem - lambda_ * sympify(np.eye(problem.shape[0]).astype(int))
det, _ = utils.algorithms.det(inter)
det = sympy.factor(det)

roots = sympy.solvers.solve(det, lambda_)

solution = "\\begin{enumerate}[label=\\arabic*.]\n"
for root in roots:
  solution += "\\item$\\lambda = {}$\n".format(root)
  eigen_prep = problem - root * np.eye(problem.shape[0]).astype(int)
  rref, _ = utils.algorithms.rref(np.c_[eigen_prep, np.zeros((problem.shape[0], 1)).astype(int)])
  matrix_sol, _ = utils.algorithms.show_solutions(rref)

  solution += "$$A - ({}) I = ".format(root)
  solution += utils.print(eigen_prep)
  solution += "$$\n"
  solution += "$$" + utils.print(np.c_[eigen_prep, np.zeros((problem.shape[0], 1)).astype(int)])
  solution += "\\xrightarrow{rref}" + utils.print(rref)
  solution += "$$\n"

  solution += "Eigenvectors: $"

  sep = ""
  for vec in matrix_sol['param_vecs']:
    if not (vec == 0).all():
      solution += sep + utils.print(vec)
      sep = ",\\quad"

  solution += "$\n"

solution += "\\end{enumerate}"

    \end{pycode}

    \item $$\py{utils.print(problem)}$$
    \ans

    $$\det(A - \lambda I) = \py{utils.print.print_matrix_det(inter)} = \printobj{det} = 0$$
    $\lambda = \py{", ".join(map(utils.print, roots))}$

    % $\py{solution}$
    Matrix is not diagonalizable over real numbers
% ----------------------
    \begin{pycode}
problem = sympify([[-12, 21], [-6, 11]])
lambda_ = sympy.Symbol("\\lambda")
inter = problem - lambda_ * sympify(np.eye(problem.shape[0]).astype(int))
det, _ = utils.algorithms.det(inter)
det = sympy.factor(det)

roots = sympy.solvers.solve(det, lambda_)

solution = "\\begin{enumerate}[label=\\arabic*.]\n"
for root in roots:
  solution += "\\item$\\lambda = {}$\n".format(root)
  eigen_prep = problem - root * np.eye(problem.shape[0]).astype(int)
  rref, _ = utils.algorithms.rref(np.c_[eigen_prep, np.zeros((problem.shape[0], 1)).astype(int)])
  matrix_sol, _ = utils.algorithms.show_solutions(rref)

  solution += "$$A - ({}) I = ".format(root)
  solution += utils.print(eigen_prep)
  solution += "$$\n"
  solution += "$$" + utils.print(np.c_[eigen_prep, np.zeros((problem.shape[0], 1)).astype(int)])
  solution += "\\xrightarrow{rref}" + utils.print(rref)
  solution += "$$\n"

  solution += "Eigenvectors: $"

  sep = ""
  for vec in matrix_sol['param_vecs']:
    if not (vec == 0).all():
      solution += sep + utils.print(vec)
      sep = ",\\quad"

  solution += "$\n"

solution += "\\end{enumerate}"

    \end{pycode}

    \item $$\py{utils.print(problem)}$$
    \ans

    $$\det(A - \lambda I) = \py{utils.print.print_matrix_det(inter)} = \printobj{det} = 0$$
    $\lambda = \py{", ".join(map(str, roots))}$

    \py{solution}

    Matrix is diagonalizable
% ----------------------
    \begin{pycode}
problem = sympify([[2, 12, -8], [0, -8, 6], [0, -9, 7]])
lambda_ = sympy.Symbol("\\lambda")
inter = problem - lambda_ * sympify(np.eye(problem.shape[0]).astype(int))
det, _ = utils.algorithms.det(inter)
det = sympy.factor(det)

roots = sympy.solvers.solve(det, lambda_)

solution = "\\begin{enumerate}[label=\\arabic*.]\n"
for root in roots:
  solution += "\\item$\\lambda = {}$\n".format(root)
  eigen_prep = problem - root * np.eye(problem.shape[0]).astype(int)
  rref, _ = utils.algorithms.rref(np.c_[eigen_prep, np.zeros((problem.shape[0], 1)).astype(int)])
  matrix_sol, _ = utils.algorithms.show_solutions(rref)

  solution += "$$A - ({}) I = ".format(root)
  solution += utils.print(eigen_prep)
  solution += "$$\n"
  solution += "$$" + utils.print(np.c_[eigen_prep, np.zeros((problem.shape[0], 1)).astype(int)])
  solution += "\\xrightarrow{rref}" + utils.print(rref)
  solution += "$$\n"

  solution += "Eigenvectors: $"

  sep = ""
  for vec in matrix_sol['param_vecs']:
    if not (vec == 0).all():
      solution += sep + utils.print(vec)
      sep = ",\\quad"

  solution += "$\n"

solution += "\\end{enumerate}"

    \end{pycode}

    \item $$\py{utils.print(problem)}$$
    \ans

    $$\det(A - \lambda I) = \py{utils.print.print_matrix_det(inter)} = \printobj{det} = 0$$
    $\lambda = \py{", ".join(map(str, roots))}$

    \py{solution}

    Matrix is diagonalizable
% ----------------------
    \begin{pycode}
problem = sympify([[2, 0, 0], [-2, -2, 2], [-5, -10, 7]])
lambda_ = sympy.Symbol("\\lambda")
inter = problem - lambda_ * sympify(np.eye(problem.shape[0]).astype(int))
det, _ = utils.algorithms.det(inter)
det = sympy.factor(det)

roots = sympy.solvers.solve(det, lambda_)

solution = "\\begin{enumerate}[label=\\arabic*.]\n"
for root in roots:
  solution += "\\item$\\lambda = {}$\n".format(root)
  eigen_prep = problem - root * np.eye(problem.shape[0]).astype(int)
  rref, _ = utils.algorithms.rref(np.c_[eigen_prep, np.zeros((problem.shape[0], 1)).astype(int)])
  matrix_sol, _ = utils.algorithms.show_solutions(rref)

  solution += "$$A - ({}) I = ".format(root)
  solution += utils.print(eigen_prep)
  solution += "$$\n"
  solution += "$$" + utils.print(np.c_[eigen_prep, np.zeros((problem.shape[0], 1)).astype(int)])
  solution += "\\xrightarrow{rref}" + utils.print(rref)
  solution += "$$\n"

  solution += "Eigenvectors: $"

  sep = ""
  for vec in matrix_sol['param_vecs']:
    if not (vec == 0).all():
      solution += sep + utils.print(vec)
      sep = ",\\quad"

  solution += "$\n"

solution += "\\end{enumerate}"

    \end{pycode}

    \item $$\py{utils.print(problem)}$$
    \ans

    $$\det(A - \lambda I) = \py{utils.print.print_matrix_det(inter)} = \printobj{det} = 0$$
    $\lambda = \py{", ".join(map(str, roots))}$

    \py{solution}

    Matrix is diagonalizable
% ----------------------
    \begin{pycode}
problem = sympify([[0, -5, 2], [-2, -2, 2], [-7, -15, 9]])
lambda_ = sympy.Symbol("\\lambda")
inter = problem - lambda_ * sympify(np.eye(problem.shape[0]).astype(int))
det, _ = utils.algorithms.det(inter)
det = sympy.factor(det)

roots = sympy.solvers.solve(det, lambda_)

solution = "\\begin{enumerate}[label=\\arabic*.]\n"
for root in roots:
  solution += "\\item$\\lambda = {}$\n".format(root)
  eigen_prep = problem - root * np.eye(problem.shape[0]).astype(int)
  rref, _ = utils.algorithms.rref(np.c_[eigen_prep, np.zeros((problem.shape[0], 1)).astype(int)])
  matrix_sol, _ = utils.algorithms.show_solutions(rref)

  solution += "$$A - ({}) I = ".format(root)
  solution += utils.print(eigen_prep)
  solution += "$$\n"
  solution += "$$" + utils.print(np.c_[eigen_prep, np.zeros((problem.shape[0], 1)).astype(int)])
  solution += "\\xrightarrow{rref}" + utils.print(rref)
  solution += "$$\n"

  solution += "Eigenvectors: $"

  sep = ""
  for vec in matrix_sol['param_vecs']:
    if not (vec == 0).all():
      solution += sep + utils.print(vec)
      sep = ",\\quad"

  solution += "$\n"

solution += "\\end{enumerate}"

    \end{pycode}

    \item $$\py{utils.print(problem)}$$
    \ans

    $$\det(A - \lambda I) = \py{utils.print.print_matrix_det(inter)} = \printobj{det} = 0$$
    $\lambda = \py{", ".join(map(str, roots))}$

    \py{solution}

    Matrix is not diagonalizable
% ----------------------
    \begin{pycode}
problem = sympify([[10, -24, 7], [6, -14, 4], [6, -15, 5]])
lambda_ = sympy.Symbol("\\lambda")
inter = problem - lambda_ * sympify(np.eye(problem.shape[0]).astype(int))
det, _ = utils.algorithms.det(inter)
det = sympy.factor(det)

roots = sympy.solvers.solve(det, lambda_)

solution = "\\begin{enumerate}[label=\\arabic*.]\n"
for root in roots:
  if not root.is_real:
      continue

  solution += "\\item$\\lambda = {}$\n".format(root)
  eigen_prep = problem - root * np.eye(problem.shape[0]).astype(int)
  rref, _ = utils.algorithms.rref(np.c_[eigen_prep, np.zeros((problem.shape[0], 1)).astype(int)])
  matrix_sol, _ = utils.algorithms.show_solutions(rref)

  solution += "$$A - ({}) I = ".format(root)
  solution += utils.print(eigen_prep)
  solution += "$$\n"
  solution += "$$" + utils.print(np.c_[eigen_prep, np.zeros((problem.shape[0], 1)).astype(int)])
  solution += "\\xrightarrow{rref}" + utils.print(rref)
  solution += "$$\n"

  solution += "Eigenvectors: $"

  sep = ""
  for vec in matrix_sol['param_vecs']:
    if not (vec == 0).all():
      solution += sep + utils.print(vec)
      sep = ",\\quad"

  solution += "$\n"

solution += "\\end{enumerate}"

    \end{pycode}

    \item $$\py{utils.print(problem)}$$
    \ans

    $$\det(A - \lambda I) = \py{utils.print.print_matrix_det(inter)} = \printobj{det} = 0$$
    $\lambda = \py{", ".join(map(utils.print, roots))}$

    \py{solution}

    Matrix is not diagonalizable over real numbers
% ----------------------
    \begin{pycode}
problem = sympify([[-2, -1, 1], [-6, -2, 0], [13, 7, -4]])
lambda_ = sympy.Symbol("\\lambda")
inter = problem - lambda_ * sympify(np.eye(problem.shape[0]).astype(int))
det, _ = utils.algorithms.det(inter)
det = sympy.factor(det)

roots = sympy.solvers.solve(det, lambda_)

solution = "\\begin{enumerate}[label=\\arabic*.]\n"
for root in roots:
  if not root.is_real:
      continue

  solution += "\\item$\\lambda = {}$\n".format(root)
  eigen_prep = problem - root * np.eye(problem.shape[0]).astype(int)
  rref, _ = utils.algorithms.rref(np.c_[eigen_prep, np.zeros((problem.shape[0], 1)).astype(int)])
  matrix_sol, _ = utils.algorithms.show_solutions(rref)

  solution += "$$A - ({}) I = ".format(root)
  solution += utils.print(eigen_prep)
  solution += "$$\n"
  solution += "$$" + utils.print(np.c_[eigen_prep, np.zeros((problem.shape[0], 1)).astype(int)])
  solution += "\\xrightarrow{rref}" + utils.print(rref)
  solution += "$$\n"

  solution += "Eigenvectors: $"

  sep = ""
  for vec in matrix_sol['param_vecs']:
    if not (vec == 0).all():
      solution += sep + utils.print(vec)
      sep = ",\\quad"

  solution += "$\n"

solution += "\\end{enumerate}"

    \end{pycode}

    \item $$\py{utils.print(problem)}$$
    \ans

    $$\det(A - \lambda I) = \py{utils.print.print_matrix_det(inter)} = \printobj{det} = 0$$
    $\lambda = \py{", ".join(map(utils.print, roots))}$

    \py{solution}

    Matrix is not diagonalizable over real numbers
% ----------------------
    \begin{pycode}
problem = sympify([[1, 2, 0], [-3, 2, 3], [-1, 2, 2]])
lambda_ = sympy.Symbol("\\lambda")
inter = problem - lambda_ * sympify(np.eye(problem.shape[0]).astype(int))
det, _ = utils.algorithms.det(inter)
det = sympy.factor(det)

roots = sympy.solvers.solve(det, lambda_)

solution = "\\begin{enumerate}[label=\\arabic*.]\n"
for root in roots:
  solution += "\\item$\\lambda = {}$\n".format(root)
  eigen_prep = problem - root * np.eye(problem.shape[0]).astype(int)
  rref, _ = utils.algorithms.rref(np.c_[eigen_prep, np.zeros((problem.shape[0], 1)).astype(int)])
  matrix_sol, _ = utils.algorithms.show_solutions(rref)

  solution += "$$A - ({}) I = ".format(root)
  solution += utils.print(eigen_prep)
  solution += "$$\n"
  solution += "$$" + utils.print(np.c_[eigen_prep, np.zeros((problem.shape[0], 1)).astype(int)])
  solution += "\\xrightarrow{rref}" + utils.print(rref)
  solution += "$$\n"

  solution += "Eigenvectors: $"

  sep = ""
  for vec in matrix_sol['param_vecs']:
    if not (vec == 0).all():
      solution += sep + utils.print(vec)
      sep = ",\\quad"

  solution += "$\n"

solution += "\\end{enumerate}"

    \end{pycode}

    \item $$\py{utils.print(problem)}$$
    \ans

    $$\det(A - \lambda I) = \py{utils.print.print_matrix_det(inter)} = \printobj{det} = 0$$
    $\lambda = \py{", ".join(map(str, roots))}$

    \py{solution}

    Matrix is not diagonalizable
% ----------------------
    \begin{pycode}
problem = sympify([[1, 3, 0, 0], [3, 1, 0, 0], [0, 0, -1, 2], [0, 0, -1, -4]])
lambda_ = sympy.Symbol("\\lambda")
inter = problem - lambda_ * sympify(np.eye(problem.shape[0]).astype(int))
det, _ = utils.algorithms.det(inter)
det = sympy.factor(det)

roots = sympy.solvers.solve(det, lambda_)

solution = "\\begin{enumerate}[label=\\arabic*.]\n"
for root in roots:
  if not root.is_real:
    continue

  solution += "\\item$\\lambda = {}$\n".format(root)
  eigen_prep = problem - root * np.eye(problem.shape[0]).astype(int)
  rref, _ = utils.algorithms.rref(np.c_[eigen_prep, np.zeros((problem.shape[0], 1)).astype(int)])
  matrix_sol, _ = utils.algorithms.show_solutions(rref)

  solution += "$$A - ({}) I = ".format(root)
  solution += utils.print(eigen_prep)
  solution += "$$\n"
  solution += "$$" + utils.print(np.c_[eigen_prep, np.zeros((problem.shape[0], 1)).astype(int)])
  solution += "\\xrightarrow{rref}" + utils.print(rref)
  solution += "$$\n"

  solution += "Eigenvectors: $"

  sep = ""
  for vec in matrix_sol['param_vecs']:
    if not (vec == 0).all():
      solution += sep + utils.print(vec)
      sep = ",\\quad"

  solution += "$\n"

solution += "\\end{enumerate}"

    \end{pycode}

    \item $$\py{utils.print(problem)}$$
    \ans

    $$\det(A - \lambda I) = \py{utils.print.print_matrix_det(inter)} = \printobj{det} = 0$$
    $\lambda = \py{", ".join(map(str, roots))}$

    \py{solution}

    Matrix is diagonalizable
% ----------------------
  \end{enumerate}
\end{question}

\begin{question}{5.11}
  Suppose that $A$ is an $n \times n$ matrix such that $A^{2}=I$ and that $\lambda$ is an eigenvalue for $A .$ Prove that $\lambda=\pm 1$

  \ans 

  \begin{gather*}
    AX = \lambda X \\
    AAX = A \lambda X \\
    IX = \lambda AX \\
    X = \lambda \lambda X \\
    X = \lambda^2 X 
  \end{gather*}
  
  Since $X$ is a nonzero matrix, $\lambda = \pm 1$

\end{question}

\begin{question}{5.12}
  Let $A$ be an $n \times n$ matrix and let $\lambda$ be an eigenvalue for $A .$ Prove that $\lambda^{2}$ is an eigenvalue for $A^{2} . $

  \ans

  \begin{gather*}
    AX = \lambda X \\
    AAX = A \lambda X \\
    A^2X = \lambda AX \\
    A^2X = \lambda \lambda X \\
    A^2X = \lambda^2 X
  \end{gather*}
  
\end{question}

\begin{question}{5.13}
  Suppose that $A$ in Exercise 5.12 is invertible. Prove that $\lambda^{-1}$ is an eigenvalue
  for $A^{-1}$

  \ans

  \begin{gather*}
    AX = \lambda X \\
    A^{-1}AX = A^{-1}\lambda X \\
    IX = \lambda A^{-1}X \\
    \lambda^{-1} X = A^{-1}X \\
  \end{gather*}
  
\end{question}

\end{document}