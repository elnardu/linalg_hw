\documentclass[letterpaper]{article}

%% Sets page size and margins
\usepackage[letterpaper,top=3cm,bottom=2cm,left=3cm,right=3cm,marginparwidth=1.75cm]{geometry}
 
%% Useful packages
\usepackage{amsmath}
\usepackage{amsfonts}
\usepackage{mathtools}
\usepackage{graphicx}
\usepackage{enumitem}
\usepackage[colorlinks=true, allcolors=blue]{hyperref}
\usepackage{pythontex}
\usepackage{lastpage}
\usepackage{fancyhdr}

\pagestyle{fancy}
\graphicspath{ {images/} }
\newcommand{\true}{$T$}
\newcommand{\false}{$F$}
\newcommand{\ans}{\textit{Answer: }}
\newcommand{\prf}{\textbf{Proof:}}
\newenvironment{question}[2][Question]{\begin{trivlist}
\item[\hskip \labelsep {\bfseries #1}\hskip \labelsep {\bfseries #2.}]}{\end{trivlist}}

\allowdisplaybreaks

% Fancy matrix
\makeatletter
\renewcommand*\env@matrix[1][*\c@MaxMatrixCols c]{%
  \hskip -\arraycolsep
  \let\@ifnextchar\new@ifnextchar
  \array{#1}}
\makeatother

% My scripts setup
\begin{pycode}
import sympy
import numpy as np

import sys
sys.path.append('..')

from utils import Utils
utils = Utils(pytex)
sympify = np.vectorize(sympy.sympify)

# display det
ddet = utils.print.print_matrix_det
\end{pycode}

\newcommand{\printobj}[1]{\py{utils.print.print_object(#1)}}

\title{MA 351, HW 8} 
\lhead{MA 351, HW 8}

\author{Elnard Utiushev}
\rhead{Elnard Utiushev}
\cfoot{\thepage\ of \pageref{LastPage}}

\begin{document}

\maketitle

Section 5.2: Exercises: 5.29 (a)(c)(e)(g), 5.32, 5.33, 5.34, 5.36, 5.37, 5.39 \\
Section 5.3: Exercises: 5.44, 5.45, 5.46, 5.57

\section{Section 5.2}
\subsection{Exercises}

\begin{question}{5.29 (a)(c)(e)(g)}
  For the following matrices, find (if possible) an invertible matrix $Q$ and a
  diagonal matrix $D$ such that $A=Q D Q^{-1} .$

  \ans 
  \begin{pycode}
problems = [
  ('a', [[-2, -1, 1], [-6, -2, 0], [13, 7, -4]]),
  ('c', [[1, 3, 0, 0], [3, 1, 0, 0], [0, 0, -1, 2], [0, 0, -1, -4]]),
  ('e', [[7, -1], [9, 1]]),
  ('g', [[1, -4, 2], [-4, 1, -2], [2, -2, -2]])
]

solution = "\\begin{enumerate}\n"
for letter, problem in problems:
  solution += f"\\item[{letter}.]\n"

  problem = sympify(problem)

  # finding the determinant 
  lambda_ = sympy.Symbol("\\lambda")
  inter = problem - lambda_ * sympify(np.eye(problem.shape[0]).astype(int))
  det, _ = utils.algorithms.det(inter)
  det = sympy.factor(det)

  # finding the eigenvalues
  roots = sympy.solvers.solve(det, lambda_)

  solution += f"$$\\det(A - \\lambda I) = {utils.print.print_matrix_det(inter)} = {utils.print(det)}$$\n"
  solution += f"$\\lambda = {', '.join(map(utils.print, roots))}$\n"

  if not all(map(lambda x: x.is_real, roots)):
    solution += "\\\\Not diagonalizable over real numbers\n"
    continue

  eigenvectors = []

  # finding the eigenvectors 
  solution += "\\begin{enumerate}[label=\\arabic*.]\n"
  for root in roots:
    solution += "\\item$\\lambda = {}$\n".format(root)
    eigen_prep = problem - root * np.eye(problem.shape[0]).astype(int)
    rref, _ = utils.algorithms.rref(np.c_[eigen_prep, np.zeros((problem.shape[0], 1)).astype(int)])
    matrix_sol, _ = utils.algorithms.show_solutions(rref)

    solution += "$$A - ({}) I = ".format(root)
    solution += utils.print(eigen_prep)
    solution += "$$\n"
    solution += "$$" + utils.print(np.c_[eigen_prep, np.zeros((problem.shape[0], 1)).astype(int)])
    solution += "\\xrightarrow{rref}" + utils.print(rref)
    solution += "$$\n"

    solution += "Eigenvectors: $"

    sep = ""
    for vec in matrix_sol['param_vecs']:
      if not (vec == 0).all():
        eigenvectors.append((root, vec))
        solution += sep + utils.print(vec)
        sep = ",\\quad"

    solution += "$\n"
  solution += "\\end{enumerate}\n"

  if len(eigenvectors) != problem.shape[0]:
    solution += "Not diagonalizable\n"
    continue

  eigenvalues, eigenvectors = zip(*eigenvectors)

  matrix_Q = np.column_stack(eigenvectors)
  matrix_D = np.diagflat(eigenvalues)

  solution += f"$$Q = {utils.print(matrix_Q)}, \\quad D = {utils.print(matrix_D)}$$\n"

solution += "\\end{enumerate}"
  \end{pycode}

  \py{solution}
  
\end{question}

\begin{question}{5.32}
  In Example $5.6,$ on page $288,$ find a matrix $B$ such that $B^{2}=A .$ Check your
  answer by direct computation.

  \begin{pycode}
D = 0.5 * np.array([
  [2 * 3 ** 0.5, -1 + 3 ** 0.5, 0],
  [0, 2, 0],
  [-4 + 4 * (3 ** 0.5), -2 + 2 * (3 ** 0.5), 2]
])
  \end{pycode}
  
  \ans Using the formula obtained in Example $5.6$, we get 
  $$
  A^{k}=Q D^{k} Q^{-1}=\frac{1}{2} \left[ \begin{array}{ccc}{2\left(3^{k}\right)} & {-1+3^{k}} & {0} \\ {0} & {2} & {0} \\ {-4+4\left(3^{k}\right)} & {-2+2\left(3^{k}\right)} & {2}\end{array}\right]
  $$
  $$
  B = A^{0.5} = Q D^{0.5} Q^{-1}=\frac{1}{2} \left[ \begin{array}{ccc}{2\left(3^{0.5}\right)} & {-1+3^{0.5}} & {0} \\ {0} & {2} & {0} \\ {-4+4\left(3^{0.5}\right)} & {-2+2\left(3^{0.5}\right)} & {2}\end{array}\right]
  $$
  $$B^2 = \printobj{D @ D} = A$$
\end{question}

\begin{question}{5.33}
  Suppose that $A$ is diagonalizable over $\mathbb{R}$ and $A$ has only $\pm 1$ as eigenvalues.
  Show that $A^{2}=I$

  \ans Since $A$ is diagonalizable over $\mathbb{R}$, we can write
  $A^k = QD^kQ^{-1}$
  ,where 
  $$D = \printobj{[[1, 0], [0, -1]]}$$
  Therefore, 
  $$A^2 = QD^2Q^{-1} = Q\left(\printobj{[[1, 0], [0, -1]]}\right)^2Q^{-1} =  QIQ^{-1} = I$$
\end{question}

\begin{question}{5.34}
  Suppose that $A$ is diagonalizable over $\mathbb{R}$ and $A$ has only 0 and 1 as eigenvalues.
  Show that $A^{2}=A$

  \ans Since $A$ is diagonalizable over $\mathbb{R}$, we can write
  $A^k = QD^kQ^{-1}$
  ,where 
  $$D = \printobj{[[0, 0], [0, 1]]}$$
  Therefore, 
  $$A^2 = QD^2Q^{-1} = Q\left(\printobj{[[0, 0], [0, 1]]}\right)^2Q^{-1} =  QDQ^{-1} = A$$
  
\end{question}

\begin{question}{5.36}
  Suppose that $A$ is diagonalizable over $\mathbb{R}$ with eigenvalues $\lambda_{i} .$ Suppose also
  that
  $$q(\lambda)=a_{n} \lambda^{n}+a_{n-1} \lambda^{n-1}+\cdots+a_{0}$$
  is some polynomial such that $q\left(\lambda_{i}\right)=0$ for all $i .$ Prove that
  $$a_{n} A^{n}+a_{n-1} A^{n-1}+\cdots+a_{0} I=\mathbf{0}$$

  \ans Since $A$ is diagonalizable over $\mathbb{R}$

  \begin{align*}
    a_{n} A^{n}+a_{n-1} A^{n-1}+\cdots+a_{0} I&=a_{n} QD^nQ^{-1}+a_{n-1} QD^{n-1}Q^{-1}+\cdots+a_{0} I \\
    &= a_{n} Q
    \left[ \begin{array}{ccc}{\lambda_0} & {\cdots} & {0} \\ {\vdots} & {\ddots} & {\vdots} \\ {0} & {\cdots} & {\lambda_i}\end{array}\right]^n
    Q^{-1}+a_{n-1} Q
    \left[ \begin{array}{ccc}{\lambda_0} & {\cdots} & {0} \\ {\vdots} & {\ddots} & {\vdots} \\ {0} & {\cdots} & {\lambda_i}\end{array}\right]^{n-1}
    Q^{-1}+\cdots+a_{0} I \\
    &= a_{n} Q
    \left[ \begin{array}{ccc}{\lambda_0^n} & {\cdots} & {0} \\ {\vdots} & {\ddots} & {\vdots} \\ {0} & {\cdots} & {\lambda_i^n}\end{array}\right]
    Q^{-1}+a_{n-1} Q
    \left[ \begin{array}{ccc}{\lambda_0^{n-1}} & {\cdots} & {0} \\ {\vdots} & {\ddots} & {\vdots} \\ {0} & {\cdots} & {\lambda_i^{n-1}}\end{array}\right]
    Q^{-1}+\cdots+a_{0} I \\
    &= Q ( a_{n}
    \left[ \begin{array}{ccc}{\lambda_0^n} & {\cdots} & {0} \\ {\vdots} & {\ddots} & {\vdots} \\ {0} & {\cdots} & {\lambda_i^n}\end{array}\right]
    +a_{n-1}
    \left[ \begin{array}{ccc}{\lambda_0^{n-1}} & {\cdots} & {0} \\ {\vdots} & {\ddots} & {\vdots} \\ {0} & {\cdots} & {\lambda_i^{n-1}}\end{array}\right]
    +\cdots ) Q^{-1}+a_{0} I \\
    &= Q
    \left[ \begin{array}{ccc}{ a_{n} \lambda_0^n + a_{n-1} \lambda_0^{n-1} + \cdots} & {\cdots} & {0} \\ {\vdots} & {\ddots} & {\vdots} \\ {0} & {\cdots} & { a_{n} \lambda_i^n + a_{n-1} \lambda_i^{n-1} + \cdots}\end{array}\right]
    Q^{-1}+a_{0}IQQ^{-1} \\
    &= Q
    \left[ \begin{array}{ccc}{ a_{n} \lambda_0^n + a_{n-1} \lambda_0^{n-1} + \cdots +a_{0}} & {\cdots} & {0} \\ {\vdots} & {\ddots} & {\vdots} \\ {0} & {\cdots} & { a_{n} \lambda_i^n + a_{n-1} \lambda_i^{n-1} + \cdots +a_{0}}\end{array}\right]
    Q^{-1} \\
    &= Q
    \left[ \begin{array}{ccc}{0} & {\cdots} & {0} \\ {\vdots} & {\ddots} & {\vdots} \\ {0} & {\cdots} & {0}\end{array}\right]
    Q^{-1} \\
    &= 0
  \end{align*}
  
\end{question}

\begin{question}{5.37}
  Find values of $a, b,$ and $c,$ all nonzero, such that the matrix $A$ below is
  diagonalizable over $\mathbb{R} :$

  $$
  A=\left[ \begin{array}{rrr}{2} & {a} & {b} \\ {0} & {-5} & {c} \\ {0} & {0} & {2}\end{array}\right]
  $$

  \ans The eigenvalues are 2 with multiplicity 2 and -5 with multiplicity 1. The dimension
  of an eigenspace $\lambda = 2$ must be 2.  

  $$A-2I=\left[ \begin{array}{rrr}{0} & {a} & {b} \\ {0} & {-7} & {c} \\ {0} & {0} & {0}\end{array}\right]$$
  
  Therefore, $[a, b] = t [-7, c]$, where t is a scalar 


\end{question}

\begin{question}{5.39}
  Suppose that $A$ and $B$ are $n \times n$ matrices such that $A=Q B Q^{-1}$ for
  some invertible matrix $Q .$ Prove that $A$ and $B$ have the same characteristic polynomials.
  Suppose that $X$ is an eigenvector for $B .$ Show that $Q X$ is an eigenvector for $A$

  \ans 

  \begin{gather*}
    \det(A-\lambda I) = \det(Q B Q^{-1} - Q \lambda I Q^{-1})
    = \det(Q (B - \lambda I) Q^{-1}) = \det(Q)\det(B - \lambda I)\det(Q^{-1}) \\
    = \det(QQ^{-1})\det(B - \lambda I) = \det(B - \lambda I)
  \end{gather*}

  \begin{gather*}
    BX = \lambda X \\
    BQ^{-1}QX = \lambda X \\
    QBQ^{-1}QX = Q \lambda X \\
    AQX = \lambda Q X 
  \end{gather*}

\end{question}


\section{Section 5.3}
\subsection{Exercises}

\begin{question}{5.44}
  Compute $A B$ and $B A$ for the matrices $A$ and $B$ .

  $$
  A=\left[ \begin{array}{cc}{1+i} & {2 i} \\ {2} & {3 i}\end{array}\right], \quad B=\left[ \begin{array}{cc}{-i} & {3} \\ {2+i} & {4 i}\end{array}\right]
  $$

  \begin{pycode}
A = sympy.Matrix(sympify(np.array([
  ['1+I', '2*I'],
  [2, '3*I']
])))
B = sympy.Matrix(sympify(np.array([
  ['-I', 3],
  ['2+I', '4*I']
])))
  \end{pycode}

  \ans 
  $$AB = \printobj{sympy.expand(sympy.MatMul(A, B).doit())}$$
  $$BA = \printobj{sympy.expand(sympy.MatMul(B, A).doit())}$$
  
\end{question}

\begin{question}{5.45}
  For the matrix $A$ below, find complex matrices $Q$ and $D$ where $D$ is diagonal
  such that $A=Q D Q^{-1}.$ Use your answer to find $A^{20}$. (Use a calculator to
  approximate the answer.)

  $$
  A=\left[ \begin{array}{rr}{1} & {-4} \\ {1} & {1}\end{array}\right]
  $$

  \begin{pycode}
problem = np.array([
  [1, -4],
  [1, 1]
])

problem = sympify(problem)

solution = ""

# finding the determinant 
lambda_ = sympy.Symbol("\\lambda")
inter = problem - lambda_ * sympify(np.eye(problem.shape[0]).astype(int))
det, _ = utils.algorithms.det(inter)
det = sympy.factor(det)

# finding the eigenvalues
roots = sympy.solvers.solve(det, lambda_)

solution += f"$$\\det(A - \\lambda I) = {utils.print.print_matrix_det(inter)} = {utils.print(det)}$$\n"
solution += f"$\\lambda = {', '.join(map(utils.print, roots))}$\n"

eigenvectors = []

# finding the eigenvectors 
solution += "\\begin{enumerate}[label=\\arabic*.]\n"
for root in roots:
  solution += "\\item$\\lambda = {}$\n".format(utils.print(root))
  eigen_prep = problem - root * np.eye(problem.shape[0]).astype(int)
  rref, _ = utils.algorithms.rref(np.c_[eigen_prep, np.zeros((problem.shape[0], 1)).astype(int)])
  matrix_sol, _ = utils.algorithms.show_solutions(rref)

  solution += "$$A - ({}) I = ".format(utils.print(root))
  solution += utils.print(eigen_prep)
  solution += "$$\n"
  solution += "$$" + utils.print(np.c_[eigen_prep, np.zeros((problem.shape[0], 1)).astype(int)])
  solution += "\\xrightarrow{rref}" + utils.print(rref)
  solution += "$$\n"

  solution += "Eigenvectors: $"

  sep = ""
  for vec in matrix_sol['param_vecs']:
    if not (vec == 0).all():
      eigenvectors.append((root, vec))
      solution += sep + utils.print(vec)
      sep = ",\\quad"

  solution += "$\n"
solution += "\\end{enumerate}\n"

if len(eigenvectors) != problem.shape[0]:
  solution += "Not diagonalizable\n"

eigenvalues, eigenvectors = zip(*eigenvectors)

matrix_Q = np.column_stack(eigenvectors)
matrix_D = np.diagflat(eigenvalues)

solution += f"$$Q = {utils.print(matrix_Q)}, \\quad D = {utils.print(matrix_D)}$$\n"

matrix_Q = sympy.Matrix(matrix_Q)
matrix_D = sympy.Matrix(matrix_D)
matrix_Q_inv = matrix_Q.inv()
  \end{pycode}

  \ans 

  \py{solution}

  \begin{gather*}
    A^{20} = QD^{20}Q^{-1} 
    = \printobj{sympy.MatMul(matrix_Q, matrix_D ** 20, matrix_Q_inv)} \\
    = \printobj{sympy.MatMul(matrix_Q, matrix_D ** 20, matrix_Q_inv).doit().evalf()}
  \end{gather*}
  
\end{question}

\begin{question}{5.46}
  Find all eigenvalues for the following matrix $A :$

  $$
  A=\left[ \begin{array}{rrr}{2} & {3} & {6} \\ {6} & {2} & {-3} \\ {3} & {-6} & {2}\end{array}\right]
  $$

  It might help to know that $-7$ is one eigenvalue.

  \begin{pycode}
problem = np.array([
  [2, 3, 6],
  [6, 2, -3],
  [3, -6, 2]
])

problem = sympify(problem)

solution = ""

# finding the determinant 
lambda_ = sympy.Symbol("\\lambda")
inter = problem - lambda_ * sympify(np.eye(problem.shape[0]).astype(int))
det, _ = utils.algorithms.det(inter)
det = sympy.factor(det)

# finding the eigenvalues
roots = sympy.solvers.solve(det, lambda_)

solution += f"$$\\det(A - \\lambda I) = {utils.print.print_matrix_det(inter)} = {utils.print(det)}$$\n"
solution += f"$\\lambda = {', '.join(map(utils.print, roots))}$\n"
  \end{pycode}

  \ans 
  \py{solution}

\end{question}

\begin{question}{5.47}
  Consider the transformation $T : \mathbb{C} \rightarrow \mathbb{C}$ given by $T(z)=(2+3 i) z .$ We
  commented that we may interpret the complex numbers as being $\mathbb{R}^{2} .$ Thus, we may
  think of $T$ as transforming $\mathbb{R}^{2}$ into $\mathbb{R}^{2} .$ Explicitly, for example, $T$ transforms
  the point $[1,2]^{t}$ into $[-4,7]^{t}$ since $(2+3 i)(1+2 i)=-4+7 i$

  Show that as a transformation of $\mathbb{R}^{2}$ into $\mathbb{R}^{2}, T$ is linear. Find a real $2 \times 2$
  matrix $A$ such that $T=T_{A} .$ Find all eigenvalues of this matrix.

  \ans 
  \begin{gather*}
    T(z_1 + z_2) = (2+3 i)(z_1 + z_2) = (2+3 i)z_1 + (2+3 i)z_2 = T(z_1) + T(z_2) \\
    T(az) = (2+3 i)(az) = a(2+3 i)z = aT(z)
  \end{gather*}

  Therefore, the transformation $T$ is linear. 

  $$A = \printobj{[[2, -3], [3, 2]]}$$

  \begin{pycode}
problem = np.array([[2, -3], [3, 2]])

problem = sympify(problem)

solution = ""

# finding the determinant 
lambda_ = sympy.Symbol("\\lambda")
inter = problem - lambda_ * sympify(np.eye(problem.shape[0]).astype(int))
det, _ = utils.algorithms.det(inter)
det = sympy.factor(det)

# finding the eigenvalues
roots = sympy.solvers.solve(det, lambda_)

solution += f"$$\\det(A - \\lambda I) = {utils.print.print_matrix_det(inter)} = {utils.print(det)}$$\n"
solution += f"$\\lambda = {', '.join(map(utils.print, roots))}$\n"
  \end{pycode}

  \py{solution}
  
  
\end{question}


\end{document}