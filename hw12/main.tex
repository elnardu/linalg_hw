\documentclass[letterpaper]{article}

%% Sets page size and margins
\usepackage[letterpaper,top=3cm,bottom=2cm,left=3cm,right=3cm,marginparwidth=1.75cm]{geometry}
 
%% Useful packages
\usepackage{amsmath}
\usepackage{amsfonts}
\usepackage{mathtools}
\usepackage{graphicx}
\usepackage{enumitem}
\usepackage[colorlinks=true, allcolors=blue]{hyperref}
\usepackage{pythontex}
\usepackage{lastpage}
\usepackage{fancyhdr}

\pagestyle{fancy}
\graphicspath{ {images/} }
\newcommand{\true}{$T$}
\newcommand{\false}{$F$}
\newcommand{\ans}{\textit{Answer: }}
\newcommand{\prf}{\textbf{Proof:}}
\newenvironment{question}[2][Question]{\begin{trivlist}
\item[\hskip \labelsep {\bfseries #1}\hskip \labelsep {\bfseries #2.}]}{\end{trivlist}}

\allowdisplaybreaks

% Fancy matrix
\makeatletter
\renewcommand*\env@matrix[1][*\c@MaxMatrixCols c]{%
  \hskip -\arraycolsep
  \let\@ifnextchar\new@ifnextchar
  \array{#1}}
\makeatother

% My scripts setup
\begin{pycode}
import sympy
import numpy as np

import sys
sys.path.append('..')

from utils import Utils
utils = Utils(pytex)
sympify = np.vectorize(sympy.sympify)

# display det
ddet = utils.print.print_matrix_det


def proj(basis, vector):
  solution = []
  solution_vec = 0
  for b in basis:
    solution.append(f"\\frac{{{utils.print(vector)}\\cdot {utils.print(b)}}}{{{utils.print(b)} \cdot {utils.print(b)}}}{utils.print(b)}")
    solution_vec += (np.dot(vector, b) / np.dot(b, b)) * b

  solution = "+".join(solution)
  solution += f"= {utils.print(solution_vec)}"
  return solution, solution_vec


def gram_schi(basis):
  solution = "\\begin{itemize}\n"
  p1 = basis[0]
  new_basis = [p1]

  solution += f"\\item $P_1 = X_1 = {utils.print(p1)}$\n"
  for i, b in enumerate(basis[1:]):
    new_p = b - proj(new_basis, b)[1]

    solution += f"\\item $Y_{i+2} = {proj(new_basis, b)[0]}$, "
    solution += f"$P_{i+2} = X_{i+2} - Y_{i+2} = {utils.print(new_p)}$"
    solution += "\n"

    if (new_p == 0).all():
      continue

    new_basis.append(new_p)

  solution += "\\end{itemize}\n\n"
  return solution, new_basis

\end{pycode}

\newcommand{\printobj}[1]{\py{utils.print.print_object(#1)}}

\title{MA 351, HW 12} 
\lhead{MA 351, HW 12}

\author{Elnard Utiushev}
\rhead{Elnard Utiushev}
\cfoot{\thepage\ of \pageref{LastPage}}

\begin{document}

\maketitle

Section 6.5: Exercises: 6.91, 6.92, 6.95, 6.100 \\
Section 6.6: True/False: 6.24, 6.26, 6.27, 6.28, 6.29, 6.33; 
Exercises: 6.106, 6.108, 6.109, 6.110, 6.115, 6.122

\section{Section 6.5}
\subsection{Exercises}

\begin{question}{6.91}
  Suppose that in the context of Example 6.16 it is proposed that our data would
  be better described by an equation of the form

  $$
  T=a+b t+c t^{2}
  $$

  \begin{enumerate}
    \item Explain how you would use the techniques of this section to find
    approximate values for $a, b,$ and $c .$ Specifically, in the normal 
    equation $A^{t} B=A^{t} A X,$ what are appropriate choices for $A$ and $B ?$

    \begin{pycode}
A_orig = np.array([0.5, 1.1, 1.5, 2.1, 2.3])
B = np.array([32.0, 33.0, 34.2, 35.1, 35.7])
    \end{pycode}

    \ans 
    $$A = [1, t, t^2] = \printobj{np.column_stack([np.ones(A_orig.shape[0]), A_orig, A_orig ** 2])}$$
    $$B = \printobj{B}$$

    \item If you have appropriate software available, compute the solution to
    the normal equation. Note the size of the constant $c .$ What is your
    interpretation of this result?

    \ans

    \begin{pycode}
A = np.column_stack([np.ones(A_orig.shape[0]), A_orig, A_orig ** 2])
A_rref, _ = utils.algorithms.rref(np.column_stack([A.T @ A, A.T @ B]))
sol, sol_str = utils.algorithms.show_solutions(A_rref)
    \end{pycode}

    $$A^t A = \printobj{A.T @ A}$$
    $$A^t B = \printobj{A.T @ B}$$
    \py{sol_str}
  \end{enumerate}
\end{question}

\begin{question}{6.92}
  The data in the chart below is the estimated population of the United States
  (in millions), rounded to three digits, from 1980 to 2000.? Your goal in this
  exercise is to predict the U.S. population in the year 2010.

  $$
  \begin{array}{cccccc}{\text { Year }} & {1980} & {1985} & {1990} & {1995} & {2000} \\ {\text { Population }} & {227} & {238} & {249} & {263} & {273}\end{array}
  $$

  For this, let $t$ denote "years after 1980$"$ and $I$ represent the increase in
  population over the 1980 level (see the chart below). Use the method of least
  squares to find constants $a$ and $b$ such that $I$ is approximately equal to $a t+b .$
  Then use your formula to predict the 2010 population.

  $$
  \begin{array}{cccccc}{\text { Years after } 1980} & {0} & {5} & {10} & {15} & {20} \\ {\text { Increase over } 1980} & {0} & {11} & {22} & {36} & {46}\end{array}
  $$

  \ans 

  \begin{pycode}
X = np.array([0, 5, 10, 15, 20]) 
X = np.column_stack([np.ones(X.shape[0]), X])
Y = np.array([0, 11, 22, 36, 46])

A = X
B = Y

AtA = A.T @ A
AtB = A.T @ B

AtA = AtA.astype(int)
AtB = AtB.astype(int)

A_rref, _ = utils.algorithms.rref(np.column_stack([AtA, AtB]))
sol, sol_str = utils.algorithms.show_solutions(A_rref)
  \end{pycode}

  $$A = \printobj{X}, \quad B = \printobj{Y}$$
  $$A^t A = \printobj{AtA}, \quad A^t B = \printobj{AtB}$$
  \py{sol_str}

  Prediction: $349/5 + 227 = 69.8 + 227 = 296.8$
  $$\printobj{np.array([1, 30]).reshape(-1, 1).T}\printobj{sol['param_vecs'][0]}
  = \printobj{sympy.MatMul(sympy.Matrix(np.array([1, 30]).reshape(-1, 1).T), sympy.Matrix(sol['param_vecs'][0])).doit()}$$
\end{question}

\begin{question}{6.95}
  Let $\mathcal{W}$ be the subspace of $\mathbb{R}^{4}$ spanned by $A_{1}=[1,2,0,1,0]^{t}$ and $A_{2}=$
  $[1,1,1,1,1]^{t} .$ Use Theorem 6.21 on page 376 to find the projection matrix $P_{w}$ ,
  $\operatorname{Proj}_{\mathcal{W}}(B)$ and $\operatorname{Orth}_{\mathcal{W}}(B),$ where $B=[1,2,3,4,5]^{t} .$ Show by direct calculation
  that Orth $_{\mathcal{W}}(B)$ is orthogonal to $A_{1}$ and $A_{2}$

  \begin{pycode}
A = sympy.Matrix(np.column_stack([
    np.array([1,2,0,1,0]),
    np.array([1,1,1,1,1])
  ]))

B = sympy.Matrix([1,2,3,4,5])

P = sympy.MatMul(sympy.MatMul(A, sympy.MatMul(A.T, A).inv()), A.T)
  \end{pycode}

  \ans 

  $$P_{\mathcal{W}} = \printobj{P} = \printobj{P.doit()}$$
  $$\operatorname{Proj}_{\mathcal{W}}(B) = \printobj{sympy.MatMul(P, B).doit()}$$
  $$\operatorname{Orth}_{\mathcal{W}}(B) = \printobj{(B - sympy.MatMul(P, B)).doit()}$$
\end{question}

\begin{question}{6.100}
  From Exercises $3.80-3.82,(A B)^{-1}=B^{-1} A^{-1} .$ What is wrong with the
  following calculation?
  
  $$
  P_{\mathcal{W}}=A\left(A^{t} A\right)^{-1} A^{t}=A A^{-1}\left(A^{t}\right)^{-1} A^{t}=I I=I
  $$

  \ans $A$ may not be invertible
\end{question}

\section{Section 6.6}
\subsection{True/False}

\begin{question}{6.24}
  The matrix $A$ is symmetric and has the characteristic polynomial $p(\lambda)=$
  $\lambda^{3}(\lambda-1)^{2}(\lambda+3) .$ Then the nullspace of $A$ might have dimension 2 .

  \ans True, since $\lambda = 0$ has multiplicity of 3
\end{question}

\begin{question}{6.26}
  The polynomial $p(\lambda)=(\lambda-1)(\lambda-2)^{3}\left(\lambda^{2}+1\right)$ 
  could be the characteristic polynomial of a symmetric matrix.

  \ans True
\end{question}

\begin{question}{6.27}
  It is impossible for a symmetric $3 \times 3$ matrix $A$ to have the vectors $[1,2,3]^{t}$ and
  $[1,1,1]^{t}$ as eigenvectors corresponding to the eigenvalues 3 and $5,$ respectively.

  \ans True, $[1,2,3]^t \cdot [1,1,1]^t \neq 0$
\end{question}

\begin{question}{6.28}
  It is impossible for a symmetric $3 \times 3$ matrix $A$ to have the vectors $[1,2,3]^{t}$ and
  $[1,1,1]^{t}$ as eigenvectors corresponding to the eigenvalue $3 .$

  \ans False
\end{question}

\begin{question}{6.29}
  Given that the matrix $A$ below has $-5$ and $-10$ as eigenvalues, it follows that
  the quadratic curve described by $-9 x^{2}+4 x y-6 y^{2}=1$ is an ellipse.

  $$
  A=\left[ \begin{array}{rr}{-9} & {2} \\ {2} & {-6}\end{array}\right]
  $$

  \ans False, $-5(x')^2 -10(y')^2 = 1$
\end{question}

\begin{question}{6.33}
  Suppose that $A$ is a $3 \times 3$ symmetric matrix with eigenvalues $1,2$ and $3 .$ Then
there are exactly eight different orthogonal matrices $Q$ such that $A=Q D Q^{t}$ is a
diagonal matrix.
  
  \ans True
\end{question}

\subsection{Exercises}

\begin{question}{6.106}
  Let
  $$
  A=\left[ \begin{array}{rrr}{10} & {2} & {2} \\ {2} & {13} & {4} \\ {2} & {4} & {13}\end{array}\right]
  $$

  Find an orthogonal matrix $Q$ and a diagonal matrix $D$ such that $A=Q D Q^{t}$
  given that the eigenvalues of $A$ are $\lambda=9$ and $\lambda=18 .$

  \begin{pycode}
D = np.array([
  [18, 0, 0],
  [0, 18, 0],
  [0, 0, 9]
])

problem = [[10, 2, 2], [2, 13, 4], [2, 4, 13]]

solution = ""

problem = sympify(problem)

# finding the determinant 
lambda_ = sympy.Symbol("\\lambda")
inter = problem - lambda_ * sympify(np.eye(problem.shape[0]).astype(int))
det, _ = utils.algorithms.det(inter)
det = sympy.factor(det)

# finding the eigenvalues
roots = sympy.solvers.solve(det, lambda_)

solution += f"$$\\det(A - \\lambda I) = {utils.print.print_matrix_det(inter)} = {utils.print(det)}$$\n"
solution += f"$\\lambda = {', '.join(map(utils.print, roots))}$\n"

if not all(map(lambda x: x.is_real, roots)):
  solution += "\\\\Not diagonalizable over real numbers\n"

eigenvectors = []

# finding the eigenvectors 
solution += "\\begin{enumerate}[label=\\arabic*.]\n"
for root in roots:
  solution += "\\item$\\lambda = {}$\n".format(root)
  eigen_prep = problem - root * np.eye(problem.shape[0]).astype(int)
  rref, _ = utils.algorithms.rref(np.c_[eigen_prep, np.zeros((problem.shape[0], 1)).astype(int)])
  matrix_sol, _ = utils.algorithms.show_solutions(rref)

  solution += "$$A - ({}) I = ".format(root)
  solution += utils.print(eigen_prep)
  solution += "$$\n"
  solution += "$$" + utils.print(np.c_[eigen_prep, np.zeros((problem.shape[0], 1)).astype(int)])
  solution += "\\xrightarrow{rref}" + utils.print(rref)
  solution += "$$\n"

  solution += "Eigenvectors: $"

  sep = ""
  for vec in matrix_sol['param_vecs']:
    if not (vec == 0).all():
      eigenvectors.append((root, vec))
      solution += sep + utils.print(vec)
      sep = ",\\quad"

  solution += "$\n"
solution += "\\end{enumerate}\n"

if len(eigenvectors) != problem.shape[0]:
  solution += "Not diagonalizable\n"

eigenvalues, eigenvectors = zip(*eigenvectors)

sol, new_eigen = gram_schi(eigenvectors[:2])
matrix_Q = np.column_stack([
  sympy.Matrix(new_eigen[0]).normalized(),
  sympy.Matrix(new_eigen[1]).normalized(),
  sympy.Matrix(eigenvectors[2]).normalized()
  ])
matrix_D = np.diagflat(eigenvalues)

solution += f"$$Q = {utils.print(matrix_Q)}, \\quad D = {utils.print(matrix_D)}$$\n"

  \end{pycode}

  \ans
  $$D = \printobj{D}$$

  \py{solution}
\end{question}

\begin{question}{6.108}
  For each variety find a quadratic form in standard form that describes the
  variety relative to the $B$ coordinates for some orthonormal basis $\mathcal{B}$ . (You
  need not find the basis $B .$ ) What, geometrically, does each variety represent?

  \begin{enumerate}[label=(\alph*)]
    \item $x^{2}+x y+2 y^{2}=1$
    \begin{pycode}
problem = np.array([
  [1, sympy.sympify(1) / 2],
  [sympy.sympify(1) / 2, 2]
])

solution = ""

problem = sympify(problem)

# finding the determinant 
lambda_ = sympy.Symbol("\\lambda")
inter = problem - lambda_ * sympify(np.eye(problem.shape[0]).astype(int))
det, _ = utils.algorithms.det(inter)
det = sympy.factor(det)

# finding the eigenvalues
roots = sympy.solvers.solve(det, lambda_)

solution += f"$$\\det(A - \\lambda I) = {utils.print.print_matrix_det(inter)} = {utils.print(det)}$$\n"
solution += f"$\\lambda = {', '.join(map(utils.print, roots))}$\n"

    \end{pycode}

    \ans
    \py{solution}

    Ellipse
    $$(3 - \sqrt{2})(x')^2 + (3 + \sqrt{2})(y')^2 = 1$$


    \item $x^{2}+4 x y+2 y^{2}=1$
    \begin{pycode}
problem = np.array([
  [1, 2],
  [2, 2]
])

solution = ""

problem = sympify(problem)

# finding the determinant 
lambda_ = sympy.Symbol("\\lambda")
inter = problem - lambda_ * sympify(np.eye(problem.shape[0]).astype(int))
det, _ = utils.algorithms.det(inter)
det = sympy.factor(det)

# finding the eigenvalues
roots = sympy.solvers.solve(det, lambda_)

solution += f"$$\\det(A - \\lambda I) = {utils.print.print_matrix_det(inter)} = {utils.print(det)}$$\n"
solution += f"$\\lambda = {', '.join(map(utils.print, roots))}$\n"

    \end{pycode}

    \ans
    \py{solution}
    
    Ellipse
    $$(3 + \sqrt{17})(x')^2 + (3 - \sqrt{17})(y')^2 = 1$$

    \item $x^{2}+4 x y+4 y^{2}=1$
    \begin{pycode}
problem = np.array([
  [1, 2],
  [2, 4]
])

solution = ""

problem = sympify(problem)

# finding the determinant 
lambda_ = sympy.Symbol("\\lambda")
inter = problem - lambda_ * sympify(np.eye(problem.shape[0]).astype(int))
det, _ = utils.algorithms.det(inter)
det = sympy.factor(det)

# finding the eigenvalues
roots = sympy.solvers.solve(det, lambda_)

solution += f"$$\\det(A - \\lambda I) = {utils.print.print_matrix_det(inter)} = {utils.print(det)}$$\n"
solution += f"$\\lambda = {', '.join(map(utils.print, roots))}$\n"

    \end{pycode}

    \ans
    \py{solution}
    
    Lines
    $$5(y')^2 = 1$$

    \item $14 x^{2}+4 x y+11 y^{2}=1$
    \begin{pycode}
problem = np.array([
  [14, 2],
  [2, 11]
])

solution = ""

problem = sympify(problem)

# finding the determinant 
lambda_ = sympy.Symbol("\\lambda")
inter = problem - lambda_ * sympify(np.eye(problem.shape[0]).astype(int))
det, _ = utils.algorithms.det(inter)
det = sympy.factor(det)

# finding the eigenvalues
roots = sympy.solvers.solve(det, lambda_)

solution += f"$$\\det(A - \\lambda I) = {utils.print.print_matrix_det(inter)} = {utils.print(det)}$$\n"
solution += f"$\\lambda = {', '.join(map(utils.print, roots))}$\n"

    \end{pycode}

    \ans
    \py{solution}
    
    Ellipse
    $$10(x')^2 + 15(y')^2 = 1$$ 
  \end{enumerate}
  
\end{question}

\begin{question}{6.109}
  For each variety in parts $(\mathrm{a})-(\mathrm{c})$ in Exercise $6.108,$ find an orthonormal basis
  $B$ for $\mathbb{R}^{2}$ for which the variety is described by a quadratic form in standard
  form in the $\mathcal{B}$ coordinates.

  \begin{enumerate}[label=(\alph*)]
    \item $x^{2}+x y+2 y^{2}=1$
    \begin{pycode}
problem = np.array([
  [1, sympy.sympify(1) / 2],
  [sympy.sympify(1) / 2, 2]
])

solution = ""

problem = sympify(problem)

# finding the determinant 
lambda_ = sympy.Symbol("\\lambda")
inter = problem - lambda_ * sympify(np.eye(problem.shape[0]).astype(int))
det, _ = utils.algorithms.det(inter)
det = sympy.factor(det)

# finding the eigenvalues
roots = sympy.solvers.solve(det, lambda_)

solution += f"$$\\det(A - \\lambda I) = {utils.print.print_matrix_det(inter)} = {utils.print(det)}$$\n"
solution += f"$\\lambda = {', '.join(map(utils.print, roots))}$\n"

# eigenvectors = []

# # finding the eigenvectors 
# solution += "\\begin{enumerate}[label=\\arabic*.]\n"
# for root in roots:
#   solution += "\\item$\\lambda = {}$\n".format(utils.print(root))
#   eigen_prep = problem - root * np.eye(problem.shape[0]).astype(int)
#   # rref, _ = utils.algorithms.rref(np.c_[eigen_prep, np.zeros((problem.shape[0], 1)).astype(int)])
#   rref = sympy.Matrix(np.c_[eigen_prep, np.zeros((problem.shape[0], 1)).astype(int)])
#   print(rref, file=sys.stderr)
#   rref = np.array(rref.rref()[0])
#   # rref, _ = utils.algorithms.rref(rref)
#   rref[1, :] = sympify([0, 1, 0])
#   matrix_sol, _ = utils.algorithms.show_solutions(rref)

#   solution += "$$A - ({}) I = ".format(utils.print(root))
#   solution += utils.print(eigen_prep)
#   solution += "$$\n"
#   solution += "$$" + utils.print(np.c_[eigen_prep, np.zeros((problem.shape[0], 1)).astype(int)])
#   solution += "\\xrightarrow{rref}" + utils.print(rref)
#   solution += "$$\n"

#   solution += "Eigenvectors: $"

#   sep = ""
#   for vec in matrix_sol['param_vecs']:
#     if not (vec == 0).all():
#       eigenvectors.append((root, vec))
#       solution += sep + utils.print(vec)
#       sep = ",\\quad"

#   solution += " $\n"
# solution += "\\end{enumerate}\n"

# if len(eigenvectors) != problem.shape[0]:
#   solution += "Not diagonalizable\n"

# eigenvalues, eigenvectors = zip(*eigenvectors)

# matrix_Q = np.column_stack(eigenvectors)
# matrix_D = np.diagflat(eigenvalues)

# solution += f"$$Q = {utils.print(matrix_Q)}, \\quad D = {utils.print(matrix_D)}$$\n"

# matrix_Q = sympy.Matrix(matrix_Q)
# matrix_D = sympy.Matrix(matrix_D)
# matrix_Q_inv = matrix_Q.inv()

eigenvectors = [
  sympify([-1 + sympy.functions.sqrt(sympy.sympify(2)), 1]),
  sympify([-1 - sympy.functions.sqrt(sympy.sympify(2)), 1])
]

Q = np.column_stack([
  sympy.Matrix(eigenvectors[0]).normalized(),
  sympy.Matrix(eigenvectors[1]).normalized()
])

D = np.diagflat(roots[::-1])

# Q = sympy.simplify(Q)
    \end{pycode}

    \ans
    \py{solution}

    Eigenvectors: $\printobj{eigenvectors[0]}, \printobj{eigenvectors[1]}$

    % $$Q = \printobj{Q}, D = \printobj{D}$$

    $$Q = \printobj{Q}$$

    % ------------------

    \item $x^{2}+4 x y+2 y^{2}=1$
    \begin{pycode}
problem = np.array([
  [1, 2],
  [2, 2]
])

solution = ""

problem = sympify(problem)

# finding the determinant 
lambda_ = sympy.Symbol("\\lambda")
inter = problem - lambda_ * sympify(np.eye(problem.shape[0]).astype(int))
det, _ = utils.algorithms.det(inter)
det = sympy.factor(det)

# finding the eigenvalues
roots = sympy.solvers.solve(det, lambda_)

solution += f"$$\\det(A - \\lambda I) = {utils.print.print_matrix_det(inter)} = {utils.print(det)}$$\n"
solution += f"$\\lambda = {', '.join(map(utils.print, roots))}$\n"

# eigenvectors = []

# # finding the eigenvectors 
# solution += "\\begin{enumerate}[label=\\arabic*.]\n"
# for root in roots:
#   solution += "\\item$\\lambda = {}$\n".format(utils.print(root))
#   eigen_prep = problem - root * np.eye(problem.shape[0]).astype(int)
#   # rref, _ = utils.algorithms.rref(np.c_[eigen_prep, np.zeros((problem.shape[0], 1)).astype(int)])
#   rref = sympy.Matrix(np.c_[eigen_prep, np.zeros((problem.shape[0], 1)).astype(int)])
#   print(rref, file=sys.stderr)
#   rref = np.array(rref.rref()[0])
#   # rref, _ = utils.algorithms.rref(rref)
#   rref[1, :] = sympify([0, 1, 0])
#   matrix_sol, _ = utils.algorithms.show_solutions(rref)

#   solution += "$$A - ({}) I = ".format(utils.print(root))
#   solution += utils.print(eigen_prep)
#   solution += "$$\n"
#   solution += "$$" + utils.print(np.c_[eigen_prep, np.zeros((problem.shape[0], 1)).astype(int)])
#   solution += "\\xrightarrow{rref}" + utils.print(rref)
#   solution += "$$\n"

#   solution += "Eigenvectors: $"

#   sep = ""
#   for vec in matrix_sol['param_vecs']:
#     if not (vec == 0).all():
#       eigenvectors.append((root, vec))
#       solution += sep + utils.print(vec)
#       sep = ",\\quad"

#   solution += " $\n"
# solution += "\\end{enumerate}\n"

# if len(eigenvectors) != problem.shape[0]:
#   solution += "Not diagonalizable\n"

# eigenvalues, eigenvectors = zip(*eigenvectors)

# matrix_Q = np.column_stack(eigenvectors)
# matrix_D = np.diagflat(eigenvalues)

# solution += f"$$Q = {utils.print(matrix_Q)}, \\quad D = {utils.print(matrix_D)}$$\n"

# matrix_Q = sympy.Matrix(matrix_Q)
# matrix_D = sympy.Matrix(matrix_D)
# matrix_Q_inv = matrix_Q.inv()

eigenvectors = [
  sympify([(-1 + sympy.functions.sqrt(sympy.sympify(17)))/4, 1]),
  sympify([-(1 - sympy.functions.sqrt(sympy.sympify(2)))/4, 1])
]

Q = np.column_stack([
  sympy.Matrix(eigenvectors[0]).normalized(),
  sympy.Matrix(eigenvectors[1]).normalized()
])

D = np.diagflat(roots)
    \end{pycode}
    \py{solution}

    Eigenvectors: $\printobj{eigenvectors[0]}, \printobj{eigenvectors[1]}$

    $$Q = \printobj{Q}$$

    % ------------------

    \item $x^{2}+4 x y+4 y^{2}=1$
    \begin{pycode}
problem = np.array([
  [1, 2],
  [2, 4]
])

solution = ""

problem = sympify(problem)

# finding the determinant 
lambda_ = sympy.Symbol("\\lambda")
inter = problem - lambda_ * sympify(np.eye(problem.shape[0]).astype(int))
det, _ = utils.algorithms.det(inter)
det = sympy.factor(det)

# finding the eigenvalues
roots = sympy.solvers.solve(det, lambda_)

solution += f"$$\\det(A - \\lambda I) = {utils.print.print_matrix_det(inter)} = {utils.print(det)}$$\n"
solution += f"$\\lambda = {', '.join(map(utils.print, roots))}$\n"

# eigenvectors = []

# # finding the eigenvectors 
# solution += "\\begin{enumerate}[label=\\arabic*.]\n"
# for root in roots:
#   solution += "\\item$\\lambda = {}$\n".format(utils.print(root))
#   eigen_prep = problem - root * np.eye(problem.shape[0]).astype(int)
#   # rref, _ = utils.algorithms.rref(np.c_[eigen_prep, np.zeros((problem.shape[0], 1)).astype(int)])
#   rref = sympy.Matrix(np.c_[eigen_prep, np.zeros((problem.shape[0], 1)).astype(int)])
#   print(rref, file=sys.stderr)
#   rref = np.array(rref.rref()[0])
#   # rref, _ = utils.algorithms.rref(rref)
#   rref[1, :] = sympify([0, 1, 0])
#   matrix_sol, _ = utils.algorithms.show_solutions(rref)

#   solution += "$$A - ({}) I = ".format(utils.print(root))
#   solution += utils.print(eigen_prep)
#   solution += "$$\n"
#   solution += "$$" + utils.print(np.c_[eigen_prep, np.zeros((problem.shape[0], 1)).astype(int)])
#   solution += "\\xrightarrow{rref}" + utils.print(rref)
#   solution += "$$\n"

#   solution += "Eigenvectors: $"

#   sep = ""
#   for vec in matrix_sol['param_vecs']:
#     if not (vec == 0).all():
#       eigenvectors.append((root, vec))
#       solution += sep + utils.print(vec)
#       sep = ",\\quad"

#   solution += " $\n"
# solution += "\\end{enumerate}\n"

# if len(eigenvectors) != problem.shape[0]:
#   solution += "Not diagonalizable\n"

# eigenvalues, eigenvectors = zip(*eigenvectors)

# matrix_Q = np.column_stack(eigenvectors)
# matrix_D = np.diagflat(eigenvalues)

# solution += f"$$Q = {utils.print(matrix_Q)}, \\quad D = {utils.print(matrix_D)}$$\n"

# matrix_Q = sympy.Matrix(matrix_Q)
# matrix_D = sympy.Matrix(matrix_D)
# matrix_Q_inv = matrix_Q.inv()

eigenvectors = [
  sympify([sympy.sympify(-2), 1]),
  sympify([sympy.sympify(1)/2, 1])
]

Q = np.column_stack([
  sympy.Matrix(eigenvectors[0]).normalized(),
  sympy.Matrix(eigenvectors[1]).normalized()
])

D = np.diagflat(roots)
    \end{pycode}
    \py{solution}

    Eigenvectors: $\printobj{eigenvectors[0]}, \printobj{eigenvectors[1]}$

    $$Q = \printobj{Q}$$
    
  \end{enumerate}
  
\end{question}

\begin{question}{6.110}
  Give an equation in the standard coordinates for $\mathbb{R}^{2}$ that describes an ellipse
  centered at the origin with a length 4 major cord parallel to the vector $[3,4]^{t}$
  and a length 2 minor axis. (The major cord is the longest line segment that
  can be inscribed in the ellipse.)
  
\end{question}

\begin{question}{6.115}
  Let $A$ be as shown. Find an orthogonal matrix $Q$ and a diagonal mattrix $D$
  such that $Q^{\prime} A Q=D .$ To save you time, we have provided the eigenvalues
  (there only two).

  $$
  A=\left[ \begin{array}{llll}{3} & {1} & {1} & {2} \\ {1} & {3} & {1} & {2} \\ {1} & {1} & {3} & {2} \\ {2} & {2} & {2} & {6}\end{array}\right], \quad \lambda_{1}=9, \quad \lambda_{2}=2
  $$
\end{question}

\begin{question}{6.122}
  Let $A$ be an $n \times n$ matrix. Suppose that there is an orthogonal matrix $Q$ and
  a diagonal matrix $D$ such that $A=Q D Q^{-1} .$ Prove that $A$ must be symmetric.
  Thus, symmetric matrices are only matrices that may be diagonalized
  using an orthogonal diagonalizing basis.
  
  \ans $A = QDQ^{-1} = QDQ^t = Q^tD^t(Q^t)^t = (QDQ^t)^t = (QDQ^{-1})^t = A^t$
\end{question}

\end{document}