\documentclass[letterpaper]{article}

%% Sets page size and margins
\usepackage[letterpaper,top=3cm,bottom=2cm,left=3cm,right=3cm,marginparwidth=1.75cm]{geometry}
 
%% Useful packages
\usepackage{amsmath}
\usepackage{amsfonts}
\usepackage{mathtools}
\usepackage{graphicx}
\usepackage{enumitem}
\usepackage[colorlinks=true, allcolors=blue]{hyperref}
\usepackage{pythontex}
\usepackage{lastpage}
\usepackage{fancyhdr}

\pagestyle{fancy}
\graphicspath{ {images/} }
\newcommand{\true}{$T$}
\newcommand{\false}{$F$}
\newcommand{\ans}{\textit{Answer: }}
\newcommand{\prf}{\textbf{Proof:}}
\newenvironment{question}[2][Question]{\begin{trivlist}
\item[\hskip \labelsep {\bfseries #1}\hskip \labelsep {\bfseries #2.}]}{\end{trivlist}}

\allowdisplaybreaks

% Fancy matrix
\makeatletter
\renewcommand*\env@matrix[1][*\c@MaxMatrixCols c]{%
  \hskip -\arraycolsep
  \let\@ifnextchar\new@ifnextchar
  \array{#1}}
\makeatother

% My scripts setup
\begin{pycode}
import sympy
import numpy as np

import sys
sys.path.append('..')

from utils import Utils
utils = Utils(pytex)
sympify = np.vectorize(sympy.sympify)

# display det
ddet = utils.print.print_matrix_det
\end{pycode}

\newcommand{\printobj}[1]{\py{utils.print.print_object(#1)}}

\title{MA 351, HW 6} 
\lhead{MA 351, HW 6}

\author{Elnard Utiushev}
\rhead{Elnard Utiushev}
\cfoot{\thepage\ of \pageref{LastPage}}

\begin{document}

\maketitle

Section 4.1: True/False: 4.1, 4.2, 4.3, 4.4, 4.5, 4.6, 4.7; Exercises: 4.1, 4.6 \\
Section 4.2: True/False: 4.9, 4.10; Exercises: 4.12, 4.15, 4.16, 4.18, 4.25, 4.26

\section{Section 4.1}
\subsection{True/False}

\begin{question}{4.1}
  The following matrices have the same determinant.

  \begin{pycode}
problems = [
  sympify([[2, 4, 2, 6], [3, 3, 27, 33], [2, 1, 5, 2], [6, 1, -3, 3]]),
  sympify([[3, 6, 3, 9], [2, 1, 5, 2], [6, 1, -3, 3], [2, 2, 18, 22]])
]

det_0, sol_0 = utils.algorithms.det(problems[0])
det_1, sol_1 = utils.algorithms.det(problems[1])
  \end{pycode}

  $$\py{utils.print.print_list_of_arrays(problems)}$$

  \ans True

  $\py{utils.print.print_matrix_det(problems[0])} = \py{sol_0} = \py{det_0}$

  $\py{utils.print.print_matrix_det(problems[1])} = \py{sol_1} = \py{det_1}$
\end{question}

\begin{question}{4.2}
  Let $A$ be a $3 \times 3$ matrix. Then det $(5 A)=5 \operatorname{det}(A)$

  \ans False, for example
  $$\py{utils.print.print_matrix_det(5 * np.eye(3).astype(int))} = 125$$
  $$5 * \py{utils.print.print_matrix_det(np.eye(3).astype(int))} = 5$$
\end{question}

\begin{question}{4.3}
  Let $A$ and $B$ be $3 \times 3$ matrices. Then $\operatorname{det}(A+B)=\operatorname{det}(A)+\operatorname{det}(B)$
  
  \ans False, for example,

  $$|A + B|
  = \py{utils.print.print_matrix_det(5 * np.eye(3).astype(int) + np.eye(3).astype(int))}
  = 216$$

  $$
  |A| + |B| = \py{utils.print.print_matrix_det(5 * np.eye(3).astype(int))} + \py{utils.print.print_matrix_det(np.eye(3).astype(int))} = 125 + 1 = 126
  $$

\end{question}

\begin{question}{4.4}
  The following statement is true:

  \begin{pycode}
A = sympify([[2, 4, 2, 6], [3, 3, 27, 33], [2, 1, 5, 2], [6, 1, -3, 3]])    
B = sympify([[2, 4, 2, 6], [1, 1, 2, 0], [2, 1, 5, 2], [6, 1, -3, 3]])
C = sympify([[2, 4, 2, 6], [1, 1, 23, 33], [2, 1, 5, 2], [6, 1, -3, 3]])

D = B.copy()
D[1] *= 2
  \end{pycode}

  $$
  \left| \begin{array}{rrrr}{2} & {4} & {2} & {6} \\ {3} & {3} & {27} & {33} \\ {2} & {1} & {5} & {2} \\ {6} & {1} & {-3} & {3}\end{array}\right|=2 \left| \begin{array}{rrrr}{2} & {4} & {2} & {6} \\ {1} & {1} & {2} & {0} \\ {2} & {1} & {5} & {2} \\ {6} & {1} & {-3} & {3}\end{array}\right|+\left| \begin{array}{cccc}{2} & {4} & {2} & {6} \\ {1} & {1} & {23} & {33} \\ {2} & {1} & {5} & {2} \\ {6} & {1} & {-3} & {3}\end{array}\right|
  $$

  \ans True

  $$
  2 * \py{ddet(B)} + \py{ddet(C)} = \py{ddet(D)} + \py{ddet(C)} = \py{ddet(A)}
  $$
  
\end{question}

\begin{question}{4.5}
  The following statement is true:

  $$
\left| \begin{array}{rrrr}{2} & {4} & {2} & {6} \\ {3} & {3} & {7} & {8} \\ {2} & {1} & {5} & {2} \\ {6} & {1} & {-3} & {3}\end{array}\right|=4 \left| \begin{array}{rrr}{3} & {7} & {8} \\ {2} & {5} & {2} \\ {6} & {-3} & {3}\end{array}\right|-3 \left| \begin{array}{rrr}{2} & {2} & {6} \\ {2} & {5} & {2} \\ {6} & {-3} & {3}\end{array}\right|
+\left| \begin{array}{rrr}{2} & {2} & {6} \\ {3} & {7} & {8} \\ {6} & {-3} & {3}\end{array}\right|-\left| \begin{array}{lll}{2} & {2} & {6} \\ {3} & {7} & {8} \\ {2} & {5} & {2}\end{array}\right|
$$

  \ans False, since the expansion started from an even column, signs should be inverted.
    
\end{question}

\begin{question}{4.6}
  The following matrices have the same determinant:
  $$\left[ \begin{array}{cccc}{1753} & {0} & {0} & {0} \\ {27} & {33} & {0} & {0} \\ {13} & {911} & {1411} & {0} \\ {-15} & {44} & {32} & {1001}\end{array}\right], \quad \left[ \begin{array}{rrrr}{1753} & {27} & {13} & {-15} \\ {0} & {33} & {911} & {44} \\ {0} & {0} & {1411} & {32} \\ {0} & {0} & {0} & {1001}\end{array}\right]$$

  \ans True, since the determinant of upper (or lower) triangular matrix is equal to 
  the product of the diagonal elements.
  
\end{question}

\begin{question}{4.7}
  The following matrices have the same determinant:

  $$
  \left[ \begin{array}{cccc}{1753} & {0} & {0} & {0} \\ {27} & {33} & {0} & {0} \\ {13} & {911} & {1411} & {0} \\ {-15} & {44} & {32} & {1001}\end{array}\right], \quad \left[ \begin{array}{rrrr}{0} & {0} & {0} & {1753} \\ {0} & {0} & {33} & {27} \\ {0} & {1411} & {911} & {13} \\ {1001} & {32} & {44} & {-15}\end{array}\right]
  $$

  \ans True, we can transform the second matrix to upper triangular matrix by switching the rows (1 and 4, 2 and 3). Since we switched rows twice, the determinant for the second matrix did not change. By applying the same reasoning as in 4.6, we prove that the matrices have the same determinants.
\end{question}

\subsection{Exercises}

\begin{question}{4.1}
  Compute the following determinants:

  \begin{pycode}
problems = [
  [[1, 4], [-3, 2]],
  [[1, 4], [2, 8]],
  [[2, 0, 1], [-1, 1, 1], [-2, 2, 2]],
  [[7, 1, 1], [0, 'a', 'b'], [0, 'd', 'c']],
  [[0, 5, 1], [-1, 1, 3], [-2, -2, 2]],
  [[2, 1, 1], [5, 4, 3], [7, 5, 4]],
  [[1, 0, 1], [2, 1, 1], [3, 2, 1]],
  [[-3, 2, 2], [1, 4, 1], [7, 6, -2]],
  [[2, 0, 2, 0], [1, 1, 1, 1], [0, 0, 3, 2], [1, 0, 0, 5]],
  [[3, 1, 3, 0], [3, 1, 3, 1], [0, 0, 2, 1], [6, 3, 4, 5]]
]

problems = list(map(sympify, problems))

solution = "\\begin{enumerate}[label=\\textbf{(\\alph*)}]\n"
for problem in problems:
  det, sol = utils.algorithms.det(problem)
  solution += f"\item ${utils.print.print_matrix_det(problem)}={sol}={det}$\n"

solution += "\\end{enumerate}\n"
  \end{pycode}
  
  \py{solution}


\end{question}

\begin{question}{4.6}
  This exercise discusses the proof of the statement that a matrix with integral
  entries has integral determinant.

  \begin{enumerate}[label=\textbf{(\alph*)}]
    \item Prove that if all the entries of a $2 \times 2$ matrix are integers, then its determinant must be an integer.
    
    \ans Let $a, b, c, d$ to be integers.
    $\py{ddet([['a', 'b'], ['c', 'd']])} = ad - bc$
    Since integer set is closed under addition and multiplication, $ad - bc$ is an integer

    \item Use part (a) and formula $(4.2)$ on page 239 to prove the statement in part
    (a) for $3 \times 3$ matrices.
    
    \ans Since integer set is closed under addition and multiplication, using the same reasoning as in part (a), the determinant of 3 by 3 matrix is an integer if all elements of the matrix are integers.
    $$
    \left| \begin{array}{lll}{a_{11}} & {a_{12}} & {a_{13}} \\ {a_{21}} & {a_{22}} & {a_{23}} \\ {a_{31}} & {a_{32}} & {a_{33}}\end{array}\right|=a_{11} \left| \begin{array}{cc}{a_{22}} & {a_{23}} \\ {a_{32}} & {a_{33}}\end{array}\right|-a_{12} \left| \begin{array}{cc}{a_{21}} & {a_{23}} \\ {a_{31}} & {a_{33}}\end{array}\right|+a_{13} \left| \begin{array}{cc}{a_{21}} & {a_{22}} \\ {a_{31}} & {a_{32}}\end{array}\right|
    $$

    \item Use part (b) to prove the statement in part (a) for $4 \times 4$ matrices. If you
    are familiar with mathematical induction, prove the statement in part (a)
    for all $n \times n$ matrices.
    
    \ans Since integer set is closed under addition and multiplication, using the same reasoning as in part (b), the determinant of 4 by 4 matrix is an integer if all elements of the matrix are integers.
  \end{enumerate}
\end{question}

\section{Section 4.2}
\subsection{True/False}

\begin{question}{4.9}
  For all $n \times n$ matrices $A$ and $B, \operatorname{det}(A B)=\operatorname{det}(B A)$

  \ans True
  $$\det(AB) = \det(A)\det(B) = \det(B)\det(A) = \det(BA)$$
  
\end{question}

\begin{question}{4.10}
  Suppose that $\operatorname{det}(A+I)=3,$ and $\operatorname{det}(A-I)=5 .$ Then

  $$
  \operatorname{det}\left(A^{2}-I\right)=20
  $$

  \ans False
  $$\det(A^2-I) = \det(A^2-I^2) = \det((A-I)(A+I)) = \det(A-I)\det(A+I) = 3*5 = 15$$
\end{question}

\subsection{Exercises}

\begin{question}{4.12}
  Use row reduction to compute the following determinants:
  
  \begin{pycode}
problems = [
  [[1, 0, 1], [2, 1, 1], [3, 2, 1]],
  [[-3, 2, 2], [1, 4, 1], [7, 6, -2]],
  [[2, 0, 2, 0], [1, 1, 1, 1], [0, 0, 3, 2], [1, 0, 0, 5]],
  [[3, 1, 3, 0], [3, 1, 3, 1], [0, 0, 2, 1], [6, 3, 4, 5]],
  [[1, 2, 3, 4], [5, 6, 7, 8], [9, 10, 11, 12], [13, 14, 15, 16]],
  [[2, 3, 2, 0], [9, 0, 1, 1], [1, 0, 1, 4], [13, 10, 0, 9]]
]

solution = "\\begin{enumerate}[label=\\textbf{(\\alph*)}]\n"
for problem in problems:
  A, sol, mul = utils.algorithms.det_rref(problem)
  pivots = utils.algorithms.get_pivots(A)
  solution += "\\item\\begin{align*}\n"
  solution += sol
  if len(pivots) == A.shape[0]:
    solution += f"= {mul}\n"
  else:
    solution += f"= 0\n"
  solution += "\\end{align*}\n"

solution += "\\end{enumerate}\n"
  \end{pycode}

  \py{solution}
  
\end{question}

\begin{question}{4.15}
  Suppose that

  $$
  \left| \begin{array}{lll}{a} & {b} & {c} \\ {d} & {e} & {f} \\ {g} & {h} & {i}\end{array}\right|=5
  $$
  
  Compute the following determinants:

  \begin{enumerate}[label=\textbf{(\alph*)}]
    \item $\left| \begin{array}{ccc}{2 a} & {2 b} & {2 c} \\ {3 d-a} & {3 e-b} & {3 f-c} \\ {4 g+3 a} & {4 h+3 b} & {4 i+3 c}\end{array}\right|$
    
    \ans 

    \begin{align*}
    \left| \begin{array}{ccc}{2 a} & {2 b} & {2 c} \\ {3 d-a} & {3 e-b} & {3 f-c} \\ {4 g+3 a} & {4 h+3 b} & {4 i+3 c}\end{array}\right|
    = 2 * 3 * 4 \left| \begin{array}{ccc}{a} & {b} & {c} \\ {d} & {e} & {f} \\ {g} & {h} & {i}\end{array}\right| = 2 * 3 * 4 * 5 = 120
    \end{align*}
    
    \item $\left| \begin{array}{ccc}{a+2 d} & {b+2 e} & {c+2 f} \\ {g} & {h} & {i} \\ {d} & {e} & {f}\end{array}\right|$

    \ans

    \begin{align*}
    \left| \begin{array}{ccc}{a+2 d} & {b+2 e} & {c+2 f} \\ {g} & {h} & {i} \\ {d} & {e} & {f}\end{array}\right| 
    = \left| \begin{array}{ccc}{a} & {b} & {c} \\ {g} & {h} & {i} \\ {d} & {e} & {f}\end{array}\right|
    = -\left| \begin{array}{ccc}{a} & {b} & {c} \\ {d} & {e} & {f} \\ {g} & {h} & {i} \end{array}\right|
    = -5
    \end{align*}
  \end{enumerate}
\end{question}

\begin{question}{4.16}
  Use row reduction to prove that for all numbers $x, y,$ and $z$
  $$
  \left| \begin{array}{ccc}{1} & {1} & {1} \\ {x} & {y} & {z} \\ {x^{2}} & {y^{2}} & {z^{2}}\end{array}\right|=(y-x)(z-x)(z-y)
  $$

  reduction.

  \ans
  \begin{align*}
  \left| \begin{array}{ccc}{1} & {1} & {1} \\ {x} & {y} & {z} \\ {x^{2}} & {y^{2}} & {z^{2}}\end{array}\right|
  = \left| \begin{array}{ccc}{1} & {1} & {1} \\ {0} & {y-x} & {z-x} \\ {0} & {y^{2}-x^2} & {z^{2}-x^2}\end{array}\right|
  = \left| \begin{array}{ccc}{1} & {1} & {1} \\ {0} & {y-x} & {z-x} \\ {0} & {0} & {z^{2}-x^2-(y+x)(z-x)}\end{array}\right|
  = (y-x)(z-x)(z-y)
  \end{align*}
  
\end{question}

\begin{question}{4.18}
  Use the row interchange, scalar, and additive properties to prove that any
  $3 \times 3$ matrix with linearly dependent rows has a zero determinant.

  \ans 
  $$
  \py{ddet(['A_1', 'A_2', 'n*A_1 + k*A_2'])}
  = -\py{ddet(['n*A_1 + k*A_2', 'A_1', 'A_2'])}
  % = -\py{ddet(['n*A_1 + k*A_2 - n*A_1', 'A_1', 'A_2'])}
  = \left| \begin{array}{c}{nA_1 + kA_2 - nA_1} \\ {A_1} \\ {A_2}\end{array}\right|
  = \left| \begin{array}{c}{kA_2 - kA_2} \\ {A_1} \\ {A_2}\end{array}\right|
  = \left| \begin{array}{c}{0} \\ {A_1} \\ {A_2}\end{array}\right|
  = 0
  $$
  
\end{question}

\begin{question}{4.25}
  Two $n \times n$ matrices $A$ and $B$ are said to be similar if there is an invertible
  matrix $Q$ such that $A=Q B Q^{-1} .$ Prove that similar matrices have the same
  determinant.

  \ans 
  $$\det(Q B Q^{-1}) = \det(Q)\det(B)\det(Q^{-1}) = \det(QQ^{-1})\det(B) = \det(B)$$
  
\end{question}

\begin{question}{4.26}
  Later we shall study $n \times n$ matrices with the property that $A A^{t}=I .$ What are
  the possible values of the determinant of such a matrix?

  \ans
  $$\det(I) = 1 = \det(AA^t) = \det(A) \det(A^t)$$
  Since this must hold even if $A$ consists of integers, $\det(A) = 1$
  
\end{question}

\end{document}