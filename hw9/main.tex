\documentclass[letterpaper]{article}

%% Sets page size and margins
\usepackage[letterpaper,top=3cm,bottom=2cm,left=3cm,right=3cm,marginparwidth=1.75cm]{geometry}
 
%% Useful packages
\usepackage{amsmath}
\usepackage{amsfonts}
\usepackage{mathtools}
\usepackage{graphicx}
\usepackage{enumitem}
\usepackage[colorlinks=true, allcolors=blue]{hyperref}
\usepackage{pythontex}
\usepackage{lastpage}
\usepackage{fancyhdr}

\pagestyle{fancy}
\graphicspath{ {images/} }
\newcommand{\true}{$T$}
\newcommand{\false}{$F$}
\newcommand{\ans}{\textit{Answer: }}
\newcommand{\prf}{\textbf{Proof:}}
\newenvironment{question}[2][Question]{\begin{trivlist}
\item[\hskip \labelsep {\bfseries #1}\hskip \labelsep {\bfseries #2.}]}{\end{trivlist}}

\allowdisplaybreaks

% Fancy matrix
\makeatletter
\renewcommand*\env@matrix[1][*\c@MaxMatrixCols c]{%
  \hskip -\arraycolsep
  \let\@ifnextchar\new@ifnextchar
  \array{#1}}
\makeatother

% My scripts setup
\begin{pycode}
import sympy
import numpy as np

import sys
sys.path.append('..')

from utils import Utils
utils = Utils(pytex)
sympify = np.vectorize(sympy.sympify)

# display det
ddet = utils.print.print_matrix_det
\end{pycode}

\newcommand{\printobj}[1]{\py{utils.print.print_object(#1)}}

\title{MA 351, HW 9} 
\lhead{MA 351, HW 9}

\author{Elnard Utiushev}
\rhead{Elnard Utiushev}
\cfoot{\thepage\ of \pageref{LastPage}}

\begin{document}

\maketitle

Section 6.1: True/False: 6.1, 6.4, 6.5, 6.6, 6.7 \\
Exercises: 6.1, 6.3, 6.9, 6.13, 6.14, 6.15

\section{Section 6.1}
\subsection{True/False}

\begin{question}{6.1}
  Let $\left\{P_{1}, P_{2}, P_{3}\right\}$ be an orthogonal subset of 
  $\mathbb{R}^{3} .$ Suppose that $X=P_{1}-2 P_{2}+$
  3$P_{3}$ satisfies $X \cdot P_{3}=6 .$ Then $\left|P_{3}\right|=2$

  \ans False, 
  $$X \cdot P_3 = (P_1 - 2P_2 + 3P_3) P_3 = P_1\cdot P_3 - 2P_2\cdot P_3 + 3P_3\cdot P_3
  = 3 |P_3|^2 = 6$$
  $$|P_3| = \sqrt{6/3} = \sqrt{2}$$
\end{question}

\begin{question}{6.4}
  The vectors $P_{1}=[1,1]^{t}$ and $P_{2}=[1,3]^{t}$ are perpendicular.

  \ans False,

  $$1*1 + 1*3 = 1+3 \neq 0$$
\end{question}

\begin{question}{6.5}
  If $X \cdot Y=0,$ then either $X=\mathbf{0}$ or $Y=\mathbf{0}$
  
  \ans False, $Y = [1, 0]^t$, $X = [0, 1]^t$, $X \cdot Y = 0$
\end{question}

\begin{question}{6.6}
  $X \cdot(Y \cdot Z)=(X \cdot Y) \cdot Z$
  
  \ans False, dot product is not associative, since dot product produces a scalar 
  and is not defined for a scalar and a vector.
\end{question}

\begin{question}{6.7}
  $(X-Y) \cdot(X+Y)=|X|^{2}-|Y|^{2}$

  \ans True,

  \begin{align*}
    (X-Y) \cdot(X+Y) &= (X-Y) \cdot X + (X-Y) \cdot Y \\
    &= |X|^{2} - Y \cdot X + X \cdot Y - |Y|^{2} = |X|^{2}-|Y|^{2}
  \end{align*}
  
\end{question}

\subsection{Exercises}

\begin{question}{6.1}
  For each pair of vectors $X$ and $Y$ below, find (i) the distance between $X$ and $Y$
  (ii) $|X|$ and $|Y|,$ (iii) $X \cdot Y,$ and (iv) the angle between $X$ and $Y .$
  
  \ans 

  \begin{pycode}
problems = [
  ([3, 4], [-1, 2]),
  ([-3, 2], [-4, 7]),
  ([1, 2, 3], [-1, 1, 2]),
  ([1, 1, 0, 2], [1, 1, 1, -3])
]

solution = "\\begin{enumerate}[label=(\\alph*)]\n"
for problem in problems:
  solution += f"\\item ${utils.print.print_list_of_arrays(problem)}$\n"
  solution += "\\begin{itemize}\n"
  
  X, Y = problem
  X = np.array(X)
  Y = np.array(Y)

  distance_sol = []
  for x, y in zip(*problem):
    if x < 0:
      x = f"({x})"
    if y < 0:
      y = f"({y})"
    distance_sol.append(f"({x} - {y})^2")

  distance_sol = "+".join(distance_sol)
  distance_sol = f"\\sqrt{{{distance_sol}}} = {np.linalg.norm(X - Y)}"

  solution += f"\\item Distance ${distance_sol}$\n"

  X_len_sol = []
  for x in X:
    X_len_sol.append(f"({x})^2")

  X_len_sol = "+".join(X_len_sol)
  X_len_sol = f"\\sqrt{{{X_len_sol}}} = {np.linalg.norm(X)}"

  solution += f"\\item $|X| = {X_len_sol}$\n"

  Y_len_sol = []
  for y in Y:
    Y_len_sol.append(f"({y})^2")

  Y_len_sol = "+".join(Y_len_sol)
  Y_len_sol = f"\\sqrt{{{Y_len_sol}}} = {np.linalg.norm(Y)}"

  solution += f"\\item $|Y| = {Y_len_sol}$\n"

  dot_sol = []
  for x, y in zip(*problem):
    if x < 0:
      x = f"({x})"
    if y < 0:
      y = f"({y})"
    dot_sol.append(f"{x} \cdot {y}")

  dot_sol = "+".join(dot_sol)
  dot_sol = f"{dot_sol} = {X @ Y}"

  solution += f"\\item $X \\cdot Y = {dot_sol}$\n"

  angle_sol = f"\\text{{arccos}}\\dfrac{{{X @ Y}}}{{{np.linalg.norm(X)} \cdot {np.linalg.norm(Y)}}}"
  angle = np.arccos((X @ Y) / (
    np.linalg.norm(X)
    * np.linalg.norm(Y)
  ))

  solution += f"\\item Angle ${angle_sol} = {angle}$ rad\n"


  solution += "\\end{itemize}\n"
solution += "\\end{enumerate}\n"
  \end{pycode}

  \py{solution}

\end{question}

\begin{question}{6.3}
  Find $c, d, e,$ and $f$ such that $[c, d, e, f]^{t}$ is perpendicular to $[a, b, a, b]^{t}$.

  \ans These 2 vectors will be perpendicular if their dot product is zero

  \begin{align*}
    ca + db + ea + fb &= 0 \\
    (c + e)a + (d + f)b &= 0 \\
    c = -e \\
    d = -f
  \end{align*}
\end{question}

\begin{question}{6.9}
  Show that the following set $\mathcal{B}$ is an orthogonal basis for $\mathbb{R}^{4} .$ Find the $\mathcal{B}$
  coordinate vector for $X=[1,2,-1,-3]^{t}$

  $$
  \mathcal{B}=\left\{[2,-1,-1,-1]^{t},[1,3,3,-4]^{t},[1,1,0,1]^{t},[1,-2,3,1]^{t}\right\}
  $$

  \ans 

  \begin{pycode}
import itertools

X = np.array([1, 2, -1, -3])

basis = [
  [2, -1, -1, -1],
  [1, 3, 3, -4],
  [1, 1, 0, 1],
  [1, -2, 3, 1]
]

basis = list(map(np.array, basis))

solution = "\\begin{gather*}\n"

for i, j in itertools.combinations(range(4), 2):
  solution += f"P_{i + 1} \cdot P_{j+i} = {utils.print(basis[i])} \cdot {utils.print(basis[j])}"
  solution += f" = {basis[0] @ basis[1]}\\\\\n"

solution += "\\end{gather*}\n"
  \end{pycode}

  \py{solution}

  Therefore, $\mathcal{B}$ is an orthogonal basis. 

  \begin{align*}
    x_1' &= \dfrac{X\cdot P_1}{|P_1|^2} 
    = \dfrac{\printobj{X.dot(basis[0])}}{\printobj{basis[0].dot(basis[0])}} \\
    x_2' &= \dfrac{X\cdot P_1}{|P_1|^2} 
    = \dfrac{\printobj{X.dot(basis[1])}}{\printobj{basis[1].dot(basis[1])}} \\
    x_3' &= \dfrac{X\cdot P_1}{|P_1|^2} 
    = \dfrac{\printobj{X.dot(basis[2])}}{\printobj{basis[2].dot(basis[2])}} = 0\\
    x_4' &= \dfrac{X\cdot P_1}{|P_1|^2} 
    = \dfrac{\printobj{X.dot(basis[3])}}{\printobj{basis[3].dot(basis[3])}} 
  \end{align*}

  $$X' = \left[4/7, 16/35, 0, -9/15\right]^t$$
\end{question}

\begin{question}{6.13}
  Let $X$ and $Y$ be vectors in $\mathbb{R}^{n} .$ Prove that $|X|^{2}+|Y|^{2}=|X+Y|^{2}$ holds if and
  only if $X \cdot Y=0 .$ Using a diagram, explain why this identity is referred to as
  the Pythagorean theorem.
  
  \ans 

  \begin{align*}
    |X+Y|^{2} &= (X+Y)(X+Y) = X(X+Y) + Y(X+Y) \\
    &= |X|^2 + X\cdot Y + Y\cdot X + |Y|^2 \\
    &= |X|^2 + 2X\cdot Y + |Y|^2
  \end{align*}

  $|X|^2 + 2X\cdot Y + |Y|^2 = |X|^{2}+|Y|^{2}$ if and only if $X \cdot Y=0$

  If we draw a right triangle, where $X$ and $Y$ will be legs of the right triangle.
  We can see that if we apply Pythagorean theorem on the $X$ and $Y$, we will get 
  the equation above. 
\end{question}

\begin{question}{6.14}
  Prove Theorem 6.1 on page $310-$ that is, prove that if $X$ and $Y$ are vectors in
  $\mathbb{R}^{n},$ then

  $$
  -|X||Y| \leq X \cdot Y \leq|X||Y|
  $$

  \ans If $Y = X = 0$ the proof is trivial. Assuming that $X$ and $Y$ are not 0 
  vectors. This inequality is true, since $X \cdot Y = |X| |Y| \cos \theta$
\end{question}

\begin{question}{6.15}
  Prove that if $X \cdot Y=\pm|X||Y|$ where $X$ and $Y$ are nonzero vectors in $\mathbb{R}^{n},$ then
  $X /|X|=\pm Y /|Y| ;$ hence, $X$ and $Y$ are scalar multiples of each other.
  
  \ans Let $U = X/|X| \mp Y/|Y|$

  \begin{align*}
    U\cdot U &= (X/|X| \mp Y/|Y|)^2
    = (\frac{X}{|X|})^2 \mp 2 \frac{X}{|X|} \cdot \frac{Y}{|Y|} + (\frac{Y}{|Y|})^2 \\
    &= 1 \mp 2 \frac{\pm|X||Y|}{|X||Y|} + 1 = 2 - 2 = 0
  \end{align*}

  \begin{align*}
    (X/|X| \mp Y/|Y|)^2 &= 0 \\ 
    X/|X| \mp Y/|Y| &= 0 \\
    X/|X| &= \pm Y/|Y|
  \end{align*}

\end{question}




\end{document}